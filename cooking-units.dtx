% \iffalse  meta-comment
% !TEX encoding = UTF-8 Unicode
% !TEX program  = pdfLaTeX
%
%
%<*internal>
\iffalse
%</internal>
%<*readme>

cooking-units
=============

Overview
--------

While writing a cookery book, units and their format can be quite
troublesome. This contains not only the question of how to print those
units, but also the question of the conversion of units. A cookery
book doesn't contain only SI units, it contains a much more colorful
range of units (kg, dag, g, oz, ... for example) and it may cause
some confusion if too many different units are used, but if you just
write a cookery book for fun you don't think about this ... until
someone (my mother in my case) tells you that it's too confusing.
 
This package solves both problems: It formats your input and you have
the option to convert the units globally or locally using an easy to
use key-value system.

Please note that this package is not intended to be used to print
units (and numbers) physically correct. It should be used for more
light-hearted stuff (like a cookery book for example).

Requirements 
------------

 * expl3 & xparse
 * translations
 * xfrac
 * l3keys2e
 * fmtcount (optional)
 

This file contains
------------------

 * README
 * cooking-units.pdf
 * cooking-units.dtx
 * cooking-units.ins

Changes
-------

See section 'Change History' in the documentation.

Supported Languages
-------------------

* German
* English
* French (suboptimal)

License
-------
This material is subject to the LaTeX Project Public License 1.3c or later. See

  https://ctan.org/license/lppl1.3c

for the details of that license.

%</readme>
%<*internal>
\fi
\def\nameofplainTeX{plain}
\ifx\fmtname\nameofplainTeX\else
  \expandafter\begingroup
\fi
%</internal>
%<*install>
\input l3docstrip.tex
%\keepsilent
\askforoverwritefalse
\preamble
Copyright (C) 2016--2017 by Ben Vitecek (current Maintainer)

This file may be distributed and/or modified under the 
conditions of the LaTeX Project Public License, either 
version 1.3c of this license or (at your option) any later 
version. The latest version of this license is in: 

	https://ctan.org/license/lppl1.3c 

and version 1.3c or later is part of all distributions of 
LaTeX version 2008-05-04 or later. 

This work has the LPPL maintenance status `maintained'.

This work consists of the files README.md, cooking-units.dtx and cooking-units.ins
and the derived file cooking-units.sty.

\endpreamble
%\postamble


%\endpostamble
\usedir{tex/latex/cooking-units}
\generate{
  \file{\jobname.sty}{\from{\jobname.dtx}{package}}
}
%</install>
%<install>\endbatchfile
%<*internal>
\usedir{source/latex/cooking-units}
\generate{
  \file{\jobname.ins}{\from{\jobname.dtx}{install}}
}
\nopreamble\nopostamble 
\usedir{doc/latex/cooking-units}
\generate{
  \file{README.md}{\from{\jobname.dtx}{readme}}
}
\ifx\fmtname\nameofplainTeX
  \expandafter\endbatchfile
\else
  \expandafter\endgroup
\fi
%</internal>
%<*driver> 
\documentclass[
        ngerman,
        french,
        english, %french, naustrian
        onlydoc,
%        check-declarations,
]{l3doc}
\usepackage[utf8]{inputenc}
\usepackage[main=english]{babel}
\usepackage[use-numerals=true]{cooking-units}
\usepackage{tikzsymbols}
\usepackage{listings}
\usepackage{environ}
\usepackage{nicefrac}
\usepackage{cleveref}
\usepackage{mathtools}
\usepackage{siunitx}

\babeltags{myde = ngerman}

\DeclareSIUnit{\ounce}{oz}
\DeclareSIUnit{\oz}{\ounce}
\DeclareSIUnit{\pound}{lb}
\DeclareSIUnit{\lb}{\pound}
\DeclareSIUnit{\stone}{st}
\DeclareSIUnit{\st}{\stone}
\DeclareSIUnit{\stick}{st}
\cusetup{print-numerals=false}

%\EnableCrossrefs


%\newcommand\TODO[1]{\footnote{\textcolor{red}{#1}}}

\ExplSyntaxOn



\clist_new:N \l_cooking_units_list_of_defined_befehlen_clist
\clist_set:Nn \l_cooking_units_list_of_defined_befehlen_clist
  {
    cunum,
    cutext,
    Cutext,
    cuam,
    newcookingunit,
    declarecookingunit,
    providecookingunit,
    cudefinename,
    cudefinesymbol,
    cusetup,
    cudefinekeys,
    cudefinesinglekey,
    cuaddkeys,
    cuaddsinglekeys,
    cuaddtokeys,
    culabel,
    curef,
  }

\NewDocumentCommand \mycs { m }
  {
    \clist_if_in:NnTF \l_cooking_units_list_of_defined_befehlen_clist {#1}
      { \cs{\textcolor{blue}{#1}} }
      { \ERROR }
  }
\newcommand\opt{\texttt}
\newcommand\true{\opt{true}}
\newcommand\false{\opt{false}}

%: \PrintUnit
\NewDocumentCommand \PrintUnit { o m }
  {
    \SaveTranslation \l_tmpa_tl { #2 (cu-unitname) }
    \tl_if_eq:NNF \l_tmpa_tl \q__cooking_units_no_translation
      { \l_tmpa_tl }
    \IfNoValueF {#1} {#1}
    &
    \texttt {#2}
    &
    \tl_use:c { l__cooking_units_default_unit_ #2 _tl }
  }
\NewDocumentCommand \PrintUnitSymbol { m }
  {
    \texttt {#1} & 
    \tl_use:c { l__cooking_units_default_unit_ #1 _tl }
  }




%% Allyourwhitespacearebelongtous (too long)
%% 42
\NewDocumentCommand \dothewhitespace { O{ Miep! } }
  {
    \textcolor{white} {#1} \\
  }
\newcommand\eatthewhitespace{}
\NewEnviron { cuexamplecode }[1][]
  { 
    \begin{center}
    \begin{minipage}{0.4\linewidth}
    \tl_replace_all:Nnn \BODY { @ } { \dothewhitespace }
    \tl_replace_all:Nnn \BODY { \% } { \color{red} \% \ignorespaces }
    \tl_replace_all:Nnn \BODY { \\ } { \\ \color{black} }
    \tl_replace_all:Nnn \BODY { & } { \ignorespaces }
    \BODY
    \end{minipage}
    
    #1
    \int_zero:N \l_tmpa_int
    \begin{minipage}{0.5\linewidth}
    \ttfamily
    \tl_set_rescan:Nno \l_tmpa_tl
      {
        \char_set_catcode_other:N \{ %
        \char_set_catcode_other:N \} %
        \char_set_catcode_other:N \_ %
        \char_set_catcode_other:N \~ %
      }
      { \BODY }
    \tl_map_inline:Nn \l_tmpa_tl
      {
        \token_if_cs:NTF ##1
          { 
            \cs_if_eq:NNTF ##1 \\ 
              { \textbackslash\textbackslash \\  } 
              { 
                \tl_set_rescan:Nnn \l_tmpb_tl
                  {
                    \char_set_catcode_ignore:N \\ %
                    \char_set_catcode_ignore:N \ %
                  }
                  {##1} 
                \textbackslash \textcolor{blue}{\l_tmpb_tl}
              }
          }
          { 
            \bool_lazy_any:nTF
              {
                { \str_if_eq_p:nn {##1} { @ } }
                { \str_if_eq_p:nn {##1} { ! } }
                { \str_if_eq_p:nn {##1} { & } }
              }
              { 
                \str_if_eq:nnTF {##1} { & } 
                  { \space } 
                  { 
                    \str_if_eq:nnTF {##1} { ! }
                      { \ERROR  }
                      { \\ }
                  }
              }
              {
                \token_if_active:NTF ##1 
                  {##1}
                  {
                    \token_if_parameter:NTF ##1
                      { 
                        \int_compare:nNnTF { \l_tmpa_int } = { \c_zero }
                          { 
                            \c_hash_str 
                            \int_incr:N \l_tmpa_int
                          }
                          { \int_zero:N \l_tmpa_int }
                      }
                      { ##1 {} }
                  }
              }
          }
      }
    \end{minipage}
    \end{center}
  }
\ProvideDocumentEnvironment{cuexamplecode}{}{}{}{}


\NewDocumentEnvironment { cuexample } { }
  { 
    \paragraph{Example:} 
%    \par \bigskip \noindent
%    \mbox{}
%    \marginpar{\raggedleft\textbf{Example}}\
%    \ignorespaces
  }
  { }

\NewDocumentEnvironment { cunote } { }
  {
    \paragraph{Note:}
  }{ }


\NewDocumentEnvironment { cuinterest } { m }
  { 
    \par \bigskip
    \clist_set:Nn \l_tmpa_clist {#1}
    \mbox{}\marginpar{\raggedleft\opt{ \clist_use:Nn \l_tmpa_clist { \\ } }}
    \ignorespaces
  }
  { }


%: table
\tl_new:N \l_translation_does_not_exist_tl
\tl_set:Nn \l_translation_does_not_exist_tl { --- }
\tl_new:N \translationdoesnotexisttl
\tl_set_eq:NN \translationdoesnotexisttl \l_translation_does_not_exist_tl
\tl_new:N \mytablecolumns
\tl_set:Nn \mytablecolumns { l l l l l }
\clist_new:N \l_list_of_units_clist
\clist_new:N \l_list_of_textunits_clist
\clist_set:Nn \l_list_of_textunits_clist
  {
    decimal-mark ,
    one (m) ,
    one (f) ,
    one (n) ,
  }
\clist_set:Nx \l_list_of_units_clist
  {
    kg , dag , g , oz , lb , space ,
    C , F , Re , K , space ,
    d , h , min , s , space ,
    m , dm , cm , mm , in , space ,
    l , dl , cl , ml , space ,
    cal , kcal , J , kJ , eV ,  space ,
    pn , EL , TL ,  csp , dsp , ssp , Msp , space ,
    \clist_use:Nn \l_list_of_textunits_clist { , } ,
    THEEND
  }

\cs_new:Npn \__table_midrule:
  {
    \\ \midrule
    \cs_gset:Npn \__table_midrule: { \\ }
  }
\cs_new_eq:NN \__table_default_midrule: \__table_midrule:

\cs_new:Npn \_cook_use:n #1
  { \cs_if_exist_use:cF { __cooking_units_#1 } {\ERROR} }

\NewDocumentCommand \CreateTableForLanguages { s >{\TrimSpaces} m }
  {
    \IfBooleanTF {#1}
      { \clist_gset_eq:NN \l_tmpa_clist \l_list_of_textunits_clist }
      { \clist_gset_eq:NN \l_tmpa_clist \l_list_of_units_clist }
    \cs_gset_eq:NN \__table_midrule: \__table_default_midrule:
    \CreateTableHeader {#2} 
    \CreateTableRow {#2} 
  }

\NewDocumentCommand \CreateTableHeader { m }
  {
     \meta{unit-key} & 
     printed \ unit & unit name & (plural) & gender
  }
\NewDocumentCommand \CreateTableRow { m }
  {
    \clist_map_inline:Nn \l_tmpa_clist
      { 
        \str_case:nnF {##1}
          {
            { space } { \\ \addlinespace \cs_set:Npn \__table_midrule: {} }
            { THEEND } { \\ \bottomrule }
          }{
            \__table_midrule:
            ##1
               & 
                 \clist_if_in:NnTF \l_list_of_textunits_clist {##1}
                   { \l_translation_does_not_exist_tl }
                   { 
                     \_cook_use:n { translate_let:nNxx } {#1} \l_tmpa_tl {##1} \c__cooking_units_postfix_unit_tl 
                     \tl_if_in:NnTF \l_tmpa_tl \q__cooking_units_no_translation
                       { \l_translation_does_not_exist_tl }
                       { \l_tmpa_tl }
                   }
               & 
               \_cook_use:n { translate_let:nNxx } {#1} \l_tmpa_tl {##1} \c__cooking_units_postfix_unitname_tl 
               \tl_if_in:NnTF \l_tmpa_tl \q__cooking_units_no_translation
                 { \l_translation_does_not_exist_tl }
                 { \l_tmpa_tl }
               & 
               \_cook_use:n { translate_let:nNxx } {#1} \l_tmpa_tl {##1} \c__cooking_units_postfix_unitname_tl 
               \_cook_use:n { translate_let:nNxx } {#1} \l_tmpb_tl {##1} \c__cooking_units_postfix_unitname_pl_tl 
               \tl_if_eq:NNF \l_tmpa_tl \l_tmpb_tl
                 { 
                   \tl_if_in:NnTF \l_tmpb_tl \q__cooking_units_no_translation
                     { \l_translation_does_not_exist_tl  }
                     { (\l_tmpb_tl)  }
                 }
               &
               \_cook_use:n { translate_let:nNxx } {#1} \l_tmpa_tl {##1} \c__cooking_units_postfix_gender_tl 
               \tl_if_in:NnTF \l_tmpa_tl \q__cooking_units_no_translation
                 { \l_translation_does_not_exist_tl }
                 { \l_tmpa_tl }
          }
      } 
  }

\NewDocumentCommand \CreateTableForPhrases { >{\TrimSpaces}  m }
  {
    \tl_set:Nn \l_tmpa_tl {#1}
    \__cooking_units_phrase_list_get_for:NNTF \l_tmpa_prop \l_tmpa_tl
      { 
          \__cooking_units_translate_let:VNxx \l_tmpa_tl \l_tmpa_clist 
            { phrase-list-list } \c__cooking_units_postfix_phrase_tl
      }
      { \ERROR }
    \clist_gput_right:Nn \l_tmpa_clist { THEEND }
    \cs_gset_eq:NN \__table_midrule: \__table_default_midrule:
        \begin{tabular}{ l l l l l }
          \toprule 
          \meta{Phrase-key} &  phrase & (plural) & gender 
          \CreatePhraseTableRow {#1} 
        \end{tabular}
  }

\NewDocumentCommand \CreatePhraseTableRow { m }
  {
    \clist_map_inline:Nn \l_tmpa_clist
      { 
        \str_case:nnF {##1}
          {
            { space } { \addlinespace \cs_set:Npn \__table_midrule: {} }
            { THEEND } { \\ \bottomrule }
          }{
            \__table_midrule:
            ##1
               & 
                 \prop_get:NnN \l__cooking_units_phrase_prop {##1} \l_tmpa_tl
                 \tl_gset:NV \l_tmpa_tl \l_tmpa_tl %% Damit global
                 \l_tmpa_tl
               & 
                 \prop_get:NnN \l__cooking_units_phrase_prop {##1-pl} \l_tmpb_tl
                 \tl_if_eq:NNF \l_tmpa_tl \l_tmpb_tl { \l_tmpb_tl }
               &
            \__cooking_units_translate_let:nNxx {#1} 
              \l_tmpa_tl { ##1 -phrase-gender } \c__cooking_units_postfix_gender_tl
               \l_tmpa_tl
          }
      }
  }


\ExplSyntaxOff

\newcommand\cCutext{\mycs{cutext} and \mycs{Cutext}}

\lstset
  {
    basicstyle   = \ttfamily,
    basewidth    = 0.51em,
    gobble       = 3,
    keywordstyle = \color{blue},
    language     = [LaTeX]{TeX},
    commentstyle=\color{red!80!black},
    moretexcs =
      {
        cunum,
        cutext,
        Cutext,
        cuam,
        newcookingunit,
        declarecookingunit,
        providecookingunit,
        cudefinename,
        cudefinesymbol,
        cusetup,
        cudefinekeys,
        cudefinesinglekey,
        cuaddkeys,
        cuaddsinglekeys,
        cuaddtokeys,
        culabel,
        curef,
        cudefinephrase,
      }
  }

\newcookingunit {st}

\cuaddkeys {lb}  
  { 
    {st} { 1/14 }  %% 1 lb are 1/14 st
  }        


\newcookingunit [\ensuremath{ {} ^ { \circ } }\kern-\scriptspace R{\text{\o}}] {Ro}

\cuaddsinglekeys {C}
  {
    {Ro} { #1*21/40 + 7.5 }
  }
\cusetup 
  {
    set-option-for-Ro = { round-to-int = true } 
  }

%: \cusetup
\cusetup
  {
%    convert-to-eV=true ,
%    cutext-range-sign = {as},
%    erase-all-options,
%    set-option-for-K= {},
% set-unknown-message = error ,
%   check-temperature = true,
%   round-precision = 30 ,
  }


\EnableCrossrefs 
\CodelineIndex 
\RecordChanges 
\AtEndDocument { \PrintChanges \PrintIndex }


\begin{document}
\DocInput{\jobname.dtx}
\end{document}
%</driver>
% \fi
%
%
%
%
% \CharacterTable
%  {Upper-case    \A\B\C\D\E\F\G\H\I\J\K\L\M\N\O\P\Q\R\S\T\U\V\W\X\Y\Z
%   Lower-case    \a\b\c\d\e\f\g\h\i\j\k\l\m\n\o\p\q\r\s\t\u\v\w\x\y\z
%   Digits        \0\1\2\3\4\5\6\7\8\9
%   Exclamation   \!     Double quote  \"     Hash (number) \#
%   Dollar        \$     Percent       \%     Ampersand     \&
%   Acute accent  \'     Left paren    \(     Right paren   \)
%   Asterisk      \*     Plus          \+     Comma         \,
%   Minus         \-     Point         \.     Solidus       \/
%   Colon         \:     Semicolon     \;     Less than     \<
%   Equals        \=     Greater than  \>     Question mark \?
%   Commercial at \@     Left bracket  \[     Backslash     \\
%   Right bracket \]     Circumflex    \^     Underscore    \_
%   Grave accent  \`     Left brace    \{     Vertical bar  \|
%   Right brace   \}     Tilde         \~}
%
%
% \iffalse 
%: changes
% \fi
% 
% \changes {2016/06/11} {0.99} {Added the package option to load 'fmtcount'.}
% \changes {2016/08/31} {1.00} {Initial version}
% \changes {2016/08/31} {1.00a} {Fixed calculation: degree Reamur to eV}
% \changes {2016/09/03} {1.01} {Added units 'ssp', 'csp', 'dsp'}
% \changes {2016/09/03} {1.01} {English unit: litre (and only litre) uses the curly l \protect\ensuremath{\protect\ell} now }
% \changes {2016/09/03} {1.01} {British English: 'pinch' is written in full}
% \changes {2016/09/03} {1.01} {Separated Messerspitze and pinch}
% \changes {2016/09/05} {1.02} {Replaced \cs{cufrac} by \cs{cuam}}
% \changes {2016/09/05} {1.02} {New message: 'obsolete-command'}
% \changes {2016/09/09} {1.02a} {Changed name from \cs{@@_parse_fraction_in_input:www} to \cs{@@_parse_mixed_fraction_in_input:www}}
% \changes {2016/09/09} {1.02b} {Changed name from \cs{@@_cunum_parse_range} (and derivatives) to \cs{@@_cutext_parse_range}.}
% \changes {2016/09/09} {1.02c} {In \cs{@@_cutext_default:nnn} it is only
%   checked once if a range is inside.}
% \changes {2016/09/09} {1.02d} {\cs{@@_calculate_input_and_store_in:nN} optimiert durch neue property-key: single.}
% \changes {2016/09/09} {1.02d} {Add 'single' to property list of singlekeys.}
% \changes {2016/09/16} {1.02e} {Only use \cs{phantom} if the argument (for \cs{phantom}) is not empty.}
% \changes {2016/09/09} {1.02i} {Delete 'single' from property lists of singlekeys cause it is not as safe as I thought.}
% \changes {2016/09/09} {1.02f} {Corrected mistake: 'ELektronenvolt' (note uppercase L) to 'Elektronenvolt' in german.}
% \changes {2016/09/26} {1.03} {New option  (and needed macros): add-temperature-to-check.}
% \changes {2016/09/26} {1.03a} {\cs{cuaddsinglekeys} now tests if the unit exists (it didn't before).}
% \changes {2016/09/26} {1.03b} {Replaced \cs{prop_clear_new:c} by \cs{prop_clear:c}.}
% \changes {2016/09/26} {1.03b} {Recalculated all electron volt values for conversion (as 'kg' was wrong before). Let's hope they are correct this time.}
% \changes {2016/09/26} {1.04} {New option: 'round-half'.}
% \changes {2016/10/19} {1.04} {Use \cs{keys_set:nn} only if second argument is not empty.}
% \changes {2016/10/19} {1.04} {'convert-to-eV' now also as optional argument available.}
% \changes {2016/10/19} {1.04} {Option 'load-time-option' now spells 'available' correct.}
% \changes {2016/10/19} {1.04} {Update of documentation.}
% \changes {2016/10/28} {1.10alpha} {Start implementation of \enquote{Change recipe from $n$ to $m$ persons.}.}
% \changes {2016/10/28} {1.10alpha} {\cs{cutext} (and \cs{Cutext}) and \cs{cuam} now parse their input like \cs{cunum}. This is needed as they also need to be changed.}
% \changes {2016/10/29} {1.10alpha} {Tiding code: Now every command is separated into a \enquote{calc} function, a \enquote{print numeric value} and a \enquote{print unit} (if there) function. At least, that's the plan.}
% \changes {2016/10/30} {1.10} {Fractions should now deal correctly with minus signs.}
% \changes {2016/11/07} {1.10} {Finished writing v1.10.}
% \changes {2016/11/13} {1.10} {Change amounts for specific number of persons.}
% \changes {2016/11/13} {1.10} {\cs{cutext}, \cs{Cutext} and \cs{cuam} check their input, allows conversion of units.}
% \changes {2016/11/13} {1.10} {New commands: \cs{culabel} and \cs{curef}.}
% \changes {2016/11/13} {1.10} {New commands: \cs{declarecookingunit} and \cs{providecookingunit}.}
% \changes {2016/11/13} {1.10} {New options: \opt{cutext-to-cunum}, \opt{cutext-change-unit} and \opt{cutext-space}.}
% \changes {2016/11/13} {1.10} {New options: \opt{cuam-version} and \opt{cutext-version}}
% \changes {2016/11/13} {1.10} {New options: \opt{recalculate-amount} and \opt{set-number-of-persons}, \opt{label}, \opt{get-label}, \opt{ref}.}
% \changes {2017/03/10} {1.10a} {\cs{curef} is now defined by \cs{NewExpandableDocumentCommand} instead of the Declare variant.}
% \changes {2017/03/10} {1.11} {Removed \cs{translate} and others from code
%   and replaced them with wrapper-macros.}
% \changes {2017/03/10} {1.11} {Removed things like 'cu-unit' from translate input
%   and placed them into separate tl's.}
% 
% \changes {2017/10/23} {1.20} {Replaced \pkg{translator} by \pkg{translations}.}
% \changes {2017/10/23} {1.20} {Added \enquote{phrases}.}
% \changes {2017/10/23} {1.20} {Added unit \enquote{stick} (of butter).}
% \changes {2017/10/23} {1.20} {|<| is not allowed as a special sign anymore.}
% \changes {2017/10/23} {1.20} {New option: \opt{set-cutext-translation-message}.}
% \changes {2017/10/23} {1.20} {New option: \opt{print-numerals}.}
% \changes {2017/10/23} {1.20} {New option: \opt{use-phrases}.}
% \changes {2017/10/23} {1.20} {New option: \opt{phrase-space}.}
% \changes {2017/10/23} {1.20} {New option: \opt{amount-unit-space}.}
% \changes {2017/10/23} {1.20} {Now checks for ranges if both values can be printed as numerals (if activated) (bug fix).}
% \changes {2017/10/23} {1.20} {Reworked quite a lot of code.}
%
%
%
%
%
%
%
%
%
%%
% \GetFileInfo{\jobname.sty}
% 
%\title{The \pkg{\jobname} package^^A
%\thanks{This document corresponds to \textsf{Benedikt Vitecek}~\fileversion, dated~\filedate.}}
%\author{Ben Vitecek \\ b.vitecek@gmx.at}%
%\date{October 23, 2017}%
%
%
% \maketitle
%
% \begin{documentation}
% \begin{abstract}
%   This package enables user to globally format units, to switch
%   between them and since v1.10 you can also change your recipes for a given
%   number of persons.
%   It should be used for light-hearted things like
%   cookery books (and not e.g. scientific texts).\footnote{I did hide  some 
%   grammatical and spelling errors for easter egg hunters \Winkey.}
%
%   Please read through the section \enquote{Important Changes}
% \end{abstract}
%
% \tableofcontents
%
% \begin{comment}
% \begin{myde}
% \cusetup{print-numerals=true,use-phrases=true}
% \cunum[convert-to-eV=true]{1}{m} \\
% \cunum[convert-to-eV=true]{1}{m} \\
% \cuam{6} Eier \\
% \cuam{12} Eier \\
% \cuam{18} Eier \\
% \cuam{132--168} Eier \\
% \cuam{12--24} Eier \\
% \cuam{12--24.2} Eier \\
% \cuam{24--12} Eier \\
% \cuam{18--6} Eier \\
%  \cuam[recalculate-amount=true, set-number-of-persons=10, label=Hallo*4, ref=Hallo]{24} Eier \\
%  \cuam[recalculate-amount=true, set-number-of-persons=10, ref=Hallo, use-phrases=false]{24} Eier (phrase-false)\\
%  \cuam[recalculate-amount=true, set-number-of-persons=10, ref=Hallo, parse-number=false]{12--24} Eier (no parse)\\
% \cuam{23} Eier \\
% \cuam{60} Eier \\
% \cuam{62} Eier \\
% \cuam{12--60} Eier \\
% \cuam{288} Eier \\
% \end{myde}
%
% \begingroup
% \cusetup{print-numerals=true}
% \cutext[convert-to-eV=true]{1}{m} \\
% \cunum[convert-to-eV=true]{1}{m} \\
% \cuam{6} Eier \\
% \cuam{12} Eier \\
% \cuam{18} Eier \\
%  \cuam[recalculate-amount=true, set-number-of-persons=10, ref=Hallo]{24} Eier \\
%  \cuam[recalculate-amount=true, set-number-of-persons=10, ref=Hallo, use-phrases=false]{24} Eier \\
% \cuam{23} Eier \\
% \cuam{60} Eier \\
% \cuam{62} Eier \\
% \endgroup
% \end{comment}
%
% \section{Introduction}
%
%
% While writing on a cookery book I used -- for reasons whatsoever --
% three different units for weight: kilogram (kg), gram (g) and
% decagram (dag, or older: dkg). Later my mother told me that she
% doesn't like it if a cookery book uses more than two different units
% (for weight in this case).  Happily I hardly used Decagram and
% therefore didn't have many problems changing the units. But, well
% \dots\ I am using \LaTeX\ and changing those units by hand seemed
% not very \LaTeX\-like, so I started writing some code to convert
% units. I expanded the code, rewrote it in \LaTeX3 (which is much
% more pleasant than \LaTeXe) and here it is.
%
%
% \subsection{Important Changes}
%
% \begin{description}
%   \item[Language] I am now using the \pkg{translations} package
%     and I hope it makes things easier. As such, declaring the used language
%     through class options shouldn't be necessary anymore.
%   \item[Phrases] This package now supports the usage of \enquote{phrases}
%     (words used instead of certain integers) (which I think are called \enquote{counting
%     measures} in english, but I am not sure).
%   \item[\cCutext] If no translation is found for a specific language,
%    \cCutext\ are replaced by \mycs{cunum} with a warning is given.
%   \item[Commands] Currently, it seems that allowing \meta{label}
%     to be set by arrow-brackets was not the best idea as it leads to problems
%    if they are made active (e.g. \pkg{babel} and option \opt{spanish}). As such,
%    |<| is not allowed as a \enquote{special-sign} anymore as this package
%   tries to \enquote{fix} this idea (at least make it work). If any problems occur
%   (for this specific case or in general) please feel free to contact me.
% \end{description}
%
%
% \subsection{Supported languages}
% 
% \begin{itemize}
%  \item German
%  \item English
%  \item French (currently suboptimal\footnote{You can only get limited information from the internet.})
% \end{itemize}
% Have another language to add or a correction of an existing one?
% See \cref{sec:bug-feed} for more details. Wanna just check the
% existing translations? See \cref{sec:Translations}.
%
% \section {The Commands}
% \label{sec:commands}
%
%
% This package offers the following commands for unit printing (and
% converting):
%
% \begin{itemize}
%   \item \mycs{cunum}\meta{label}\oarg{options}\marg{amount}\oarg{space}\marg{unit-key}
%   \item \mycs{cutext}\meta{label}\oarg{options}\marg{amount}\marg{unit-key}
%   \item \mycs{Cutext}\meta{label}\oarg{options}\marg{amount}\marg{unit-key}
%   \item \mycs{cuam}\meta{label}\oarg{options} \marg{amount}
%   \item \mycs{cusetup}\marg{options}
% \end{itemize}
%
% Numbers and units are printed using \mycs{cunum}. The numerical part
% can interpret \verb|_| and \texttt{/} as (mixed) fractions and
% \verb|--| as a separator for ranges; to convert units use the option
% \texttt{\meta{old-unit}=\meta{new-unit}}\footnote{New keys can be
% added and defined, see \cref{sec:pred-units} and
% \cref{sec:def-units} for further information.}.  It furthermore
% allows the sign \texttt{?} to be used as a placeholder for not known
% amounts and raises a warning to remind that this amount needs a
% check-up\footnote{You can customize this behavior, see
% \cref{sec:options}}. \oarg{space} adds a space between the number
% and the unit using \lstinline|\phantom|.
%
% For a list of predefined units have a look at \cref{tab:def-units}. 
%
% \meta{label} is explained in \cref{sec:labeletref}.
% 
% 
% \begin{cuexamplecode}
%   \cunum{1}{kg} \\
%   \cunum{2.3}{kg} \\
%   \cunum{2,3}{kg}  \\
%   \cunum{2--3}{kg}  \\
%   \cunum{2.5--3.5}{kg}  \\
%   \cunum[kg=g]{2.5--3.5}{kg}  \\
%   \cunum[C=F]{200}{C}  \\
%   \cunum[C=F]{180--200}{C} \\
%   \cunum{1/2}{m}  \\
%   \cunum{1_1/2}{m}  \\
%   \cunum[m=cm]{1_1/2}{m}  \\
%   \cunum{?}{l}  \\
%   \cunum{50}{dag}  \\
%   \cunum{5}[0]{dag}  \\
%   \cunum{1.1234}{m} 
% \end{cuexamplecode} 
%
% Decimal numbers are automatically rounded to 2 digits after the
% colon, temperatures (\texttt{C}, \texttt{F}, \texttt{K} and
% \texttt{Re}) are automatically rounded to integers.\footnote{You can
% -- of course -- change this behavior, see \cref{sec:options}.}
% 
%
% \cCutext\ print the number and the written name of the unit. Since
% v1.10 it works similar\footnote{One could also say \enquote{exactly
% like}.} to \mycs{cunum}: it allows the conversion between units and
% interprets the numerical part (again |_| and |/| are used for
% (mixed) fractions and |--| for ranges).  Furthermore, if the package
% option \opt{use-numerals} is used, integers below a specific integer
% (by default 13; see \opt{use-numerals-below}) are written out with
% \lstinline|\Cutext| capitalizing the first letter (using package
% \pkg{fmtcount}).
%
%
% \begin{cuexamplecode}
%   \cutext{1}{l} \\
%   \Cutext{1}{l} \\
%   \Cutext{1--2}{l} \\
%   \cutext{12}{l} \\
%   \Cutext{13}{l} 
% \end{cuexamplecode}
% and using package option \opt{use-numerals=true}
% \cusetup{print-numerals=true}
% \begin{cuexamplecode}
%   \cutext{1}{l} \\
%   \Cutext{1}{l} \\
%   \cutext{1--2}{l} \\
%   \cutext{12}{l} \\
%   \Cutext{13}{l} 
% \end{cuexamplecode}
% \cusetup{print-numerals=false}
% Furthermore,
% since v1.10 \cCutext\ also allow their units to be changed (this behavior
% can be altered using \opt{cutext-change-unit}):
% \begin{cuexamplecode}
%   \cusetup{ l=ml } @
%   \cutext{1}{l} \\
%   \Cutext{1}{l} \\
%   \cutext{1--2}{l} \\
%   \cutext{12}{l} \\
%   \Cutext{13}{l} \\
%   \Cutext{?}{l} \\
%   \Cutext{1/2}{l} \\
% \end{cuexamplecode}
%
% 
% \lstinline|\cuam| works like \mycs{cunum}, but without a unit, so
% changing units doesn't affect it. Like \mycs{cunum} \texttt{\_} and
% \texttt{/} are used to imply a (mixed) fraction and |--| is used for ranges.
%
% \begin{cuexamplecode}
%   \cuam{3} \\
%   \cuam{2--3} \\
%   \cuam{2/3} \\
%   \cuam{1_2/3} 
% \end{cuexamplecode}
%
% Furthermore it allows the concept of \enquote{phrases} (replacing a positive integer
% by a word, such as \enquote{12} becoming \enquote{dozen}\footnote{At least I think})
% which can be activated by the option \opt{use-phrases}
% (as I don't know any english phrases, I switched the language to german  for the following examples)
% \begin{myde}
% \begin{cuexamplecode}
%   \cusetup{use-phrases=true} @
%   \cuam{12} \\
%   \cuam{13} \\
%   \cuam{24} \\
%   \cuam{12--24} \\
%   \cuam{12--13}\\
%   \cuam{18} \\
%   \cuam{60} 
% \end{cuexamplecode}
% \end{myde}
%
%
% \section{Label \& refs: Changing the amount of the recipe}
% \label{sec:labeletref}
%
%
% What if you don't want to change units, but the amounts of the recipe because
% you cook not for 4 persons, but for 2 and don't like to do the math? 
% Simple, use the following commands:
%
% \begin{itemize}
%   \item \mycs{culabel}\marg{label}\marg{number of persons}
%   \item \mycs{curef}\marg{label}
% \end{itemize}
%
%
% The first one is the important one: It defines a \meta{label} for a recipe
% which is initially for  \meta{number of persons}. Afterwards \meta{label}
% can be used to tell the commands from \cref{sec:commands} that the given
% amounts are for \meta{number of persons}.
% Each \meta{label} must be unique and an error is raised if a \meta{label}
% is already defined.
%  
% If you would like to print the number of persons this recipe is for, use 
% \mycs{curef}, which is fully expandable.
%
% The following example uses \mycs{culabel} to specify that the recipe is
% initially intended for 2 persons:
% \begin{cuexamplecode}
%   \culabel{recipe}{2} @
%   recipe & for & \curef{recipe} & persons: \\
%   \cunum<recipe>{10--20}{dag} & flour, \\
%   \cunum<recipe>{1/2}{l} & water, \\
%   \cutext[ref=recipe]{10}{g} & nuts, \\
%   \cuam<recipe>{2--3} & eggs, \\
%   \cunum{180}{C} & (\cunum[C=F]{180}{C}) & open & fire
% \end{cuexamplecode}
%
% Now with combination of the option \opt{set-number-of-persons}
% and setting \opt{recalculate-amount} to \opt{true} you can have this recipe
% changed to four persons:
%
% \begin{lstlisting}
%   \culabel{recipe}{2}
%   %% adding options:
%   \cusetup{set-number-of-persons=4,recalculate-amount=true}
% \end{lstlisting}
% \cusetup{set-number-of-persons=4,recalculate-amount=true}
%
% \begin{cuexamplecode}
%   recipe & for & \curef{recipe} & persons: \\
%   \cunum<recipe>{10--20}{dag} & flour, \\
%   \cunum<recipe>{1/2}{l} & water, \\
%   \cutext[ref=recipe]{10}{g} & nuts, \\
%   \cuam<recipe>{2--3} & eggs, \\
%   \cunum{180}{C} & & (\cunum[C=F]{180}{C}) & open & fire
% \end{cuexamplecode}
% \cusetup{recalculate-amount=false}
%
% Note that fractions are automatically evaluated and that only values
% with a \meta{label} are changed (\lstinline|\cunum{180}{C}| for example
% stays the same which also makes sense as the heat should be the same).
%
%
%
%
% \section{Some Interesting options}
% 
% This package has some options which might be of interest and to highlight them, this section exists. All options can be
% found in \cref{sec:options}.
%
%
%
%   \begin{cuinterest}{use-numerals, use-numerals-below, print-numerals} 
%   As seen above, you can use the \emph{package}-option
%   \opt{use-numerals} to print integers used by \cCutext{} below
%   \opt{use-numerals-below} (13 by default)  by \pkg{fmtcount}.
%   You can still decide if numerals should be printed or not with \opt{print-numerals}.
%
%   \begin{cunote}
%     \opt{use-numerals} is a package option as it needs to load
%     \pkg{fmtcount} which is not loaded by default. 
%   \end{cunote}
%
% \end{cuinterest}
%
% \begin{cuinterest}{use-phrases}
%   In (I presume) all languages there exist phrases for a given
%   amount or a number of things (think it is called \enquote{counting measurement}).
%   In German you may say instead of
%   \enquote{12}: \enquote{ein Dutzend}. Using this option you can tell this
%   package to replace predefined integers used in \mycs{cuam} 
%   by phrases for the currently used language
%   (to define new ones, see \cref{sec:phrases})
%
%   Using (for example) language \opt{ngerman} (or \opt{naustrian}, etc.) with
%   package option \opt{use-phrases=true} gives:
%   \begin{myde}
%     \begin{cuexamplecode}
%       \cusetup{use-phrases=true} @
%       \cuam{12} \\
%       \cuam{24} \\
%       \cuam{12--24} \\
%       \cuam{12--13} \\
%       \cuam{18} \\
%       \cuam{60} 
%     \end{cuexamplecode}
%   \end{myde}
% This of course also works with the \emph{package}-option \opt{use-numerals}:
%   \begin{myde}
%     \cusetup{print-numerals=true} 
%     \begin{cuexamplecode}
%       \cusetup{use-phrases=true} @
%       \cuam{12} \\
%       \cuam{24} \\
%       \cuam{12--24} \\
%       \cuam{12--13} \\
%       \cuam{18} \\
%       \cuam{60} 
%     \end{cuexamplecode}
%   \end{myde}
%
%
%   \begin{cunote}
%     Currently only the lower-case variant for \opt{use-numerals} is
%     supported. Furthermore this feature is only available for \mycs{cuam}.
%   \end{cunote}
%
%
%  \end{cuinterest}
%
%
% \section{Predefined units \& some notes}
% \label{sec:pred-units}
%
% \begin{comment}
% In \cref{tab:def-units} and \cref{tab:some-more-units} (and \cref{tab:strange-units})
% you can find
% all predefined units.  In \cref{sec:Translations} all translations available
%  are listed.
% \end{comment}
%
% In \cref{tab:def-units} and you can find all predefined units which can be
% transformed into each other (sorted by group). Other predefined units
% (which cannot be used for transformation) are shown in \cref{tab:some-more-units}.
% \Cref{tab:strange-units} pretty much exists just for fun.
%
%
% \begin{table}[htbp]
%   \centering
%   \caption{^^A
%^^A    The first column shows a list of predefined unit-keys.
%^^A   The column \enquote{default} shows the abbreviation used if
%^^A   for given language no translation is defined. The translations used for \cCutext\
%^^A   are shown in \cref{sec:Translations}.
%^^A   Note that \enquote{electron volt} exists just for fun.
%^^A
%   This table shows all units which  can be transformed into each other, sorted
%   by group. The columns \enquote{default} show the abbreviations used if
%   for given language no translation is defined. The translations used for \cCutext\
%   are shown in \cref{sec:Translations}. 
%   Note that \enquote{electron volt} exists just for fun.
% }
%   \begin{tabular}{ lll @{\hspace{1cm}} lll  }
%       \toprule
%      description & key & default & description & key & default \\ \midrule
%     \PrintUnit {kg} & \PrintUnit{m}  \\
%     \PrintUnit {dag} & \PrintUnit{dm}  \\
%     \PrintUnit {g} & \PrintUnit{cm}  \\
%     \PrintUnit {oz} & \PrintUnit{mm} \\ 
%     \PrintUnit {lb} & \PrintUnit{in} \\ 
%     \PrintUnit [ (of butter)] {stick} \\ \addlinespace
%
%     \PrintUnit {d} & \PrintUnit {l}  \\
%     \PrintUnit {h} & \PrintUnit {dl}  \\
%     \PrintUnit {min} & \PrintUnit {cl} \\
%     \PrintUnit {s} & \PrintUnit{ml} \\ \addlinespace
%
%     \PrintUnit{cal} & \PrintUnit{C} \\
%     \PrintUnit{kcal} & \PrintUnit{F} \\
%     \PrintUnit{J} & \PrintUnit{Re} \\
%     \PrintUnit{kJ} & \PrintUnit{K} \\
%     \PrintUnit{eV} 
%
%    \\  \bottomrule
%   \end{tabular}
%   \label{tab:def-units}
% \end{table}
%
% \begin{table}[htbp]
%   \centering
%   \caption{A (not only) spoonful of (more or less) country and
%   language dependent units.  Please note that sometimes a
%   translation is nearly impossible as a unit
%   (e.g. \enquote{saltspoonful}) may not exist in another language
%   (like german; at least I never heard of it). So please only use
%   units known to you.}
%   \begin{tabular}{ lll }
%     \toprule
%     description & key & symbol \\ \midrule
%     \PrintUnit {pn} \\ \addlinespace
%     \PrintUnit {EL} \\
%     \PrintUnit {TL} \\
%     \PrintUnit {dsp} \\ 
%     \PrintUnit {csp} \\ 
%     \PrintUnit {ssp} \\ \addlinespace
%     \PrintUnit [Messerpsitze (point of a knife)] {Msp} \\
%     \bottomrule
%   \end{tabular}
%   \label{tab:some-more-units}
% \end{table}
% 
% \begin{table}[htbp]
%   \centering
%   \caption{List of (not really) nonsense units (exist just for fun, there will be
%   no support for those units; unless -- of course -- you really want it).}
%   \begin{tabular}{ ll }
%    \toprule
%      unit-key & symbol \\ \midrule
%     \PrintUnitSymbol {eVc-2} \\
%     \PrintUnitSymbol {hbareV-1} \\
%     \PrintUnitSymbol {chbareV-1} \\ 
%     \PrintUnitSymbol {(chbareV-1)3} \\ 
%     \bottomrule
%   \end{tabular}
%   \label{tab:strange-units}
% \end{table}
%
%
%
% \section{Defining units}
% \label{sec:def-units}
%
% New units can be defined using \mycs{declarecookingunit}, 
% \mycs{newcookingunit} and \mycs{providecookingunit}:
%
% \begin{function}{\declarecookingunit,\newcookingunit,\providecookingunit}
%   \begin{syntax}
%     \mycs{declarecookingunit}\oarg{symbol}\marg{unit-key}
%     \mycs{newcookingunit}\oarg{symbol}\marg{new-unit-key}
%     \mycs{providecookingunit}\oarg{symbol}\marg{new-unit-key}
%   \end{syntax}
%   These commands define the unit  \meta{unit-key}. If the 
%   key is not the same as the printed symbol use \oarg{symbol}.
%   Note that \meta{unit-key} should neither contain |/| nor |,|.
%
%   \mycs{newcookingunit} raises an error if the unit is already defined,
%   \mycs{declarecookingunit} creates or (if given) overwrites  \meta{symbol} 
%   and \mycs{providecookingunit} does nothing if the unit is already
%   defined. 
%
%   All units have male gender |m| by default.
%
% Some examples:
%
% \begin{lstlisting}
%   \declarecookingunit{kg}
%   \declarecookingunit{g}
%   \declarecookingunit[Msp.] {Msp}
%   \declarecookingunit[\ensuremath{{}^{\circ}}\kern-\scriptspace C] {C}
% \end{lstlisting}
%
% \begin{cunote}
%  The definition of the printed degree Celsius is
% directly copied and pasted from (a maybe older version of) \pkg{siunitx}
% \end{cunote}
%
%
% \end{function}
%
%
% \section{Defining options to change units}
% \label{sec:defining-opt}
%
% Options (to change units) can be newly defined or added to already existing keys (units)
% using
% \begin{itemize}
%   \item \mycs{cudefinekeys}
%   \item \mycs{cudefinesinglekey}
%   \item \mycs{cuaddkeys}
%   \item \mycs{cuaddsinglekeys}
%   \item \mycs{cuaddtokeys}
% \end{itemize}
% I apologize for the (name) inconsistency between \mycs{cudefinekeys} and
% \mycs{cudefinesinglekey} (although they are named similarly they work different).
%
% \begin{function}{\cudefinekeys,\cudefinesinglekey}
%   \begin{syntax}
%     \mycs{cudefinekeys}\marg{unit-key-1}
%     \ \ \{
%     \ \ \ \ \marg{unit-key-2} \marg{1 unit-key-1 are \dots\ unit-key-2}
%     \ \ \ \ \marg{unit-key-3} \marg{1 unit-key-1 are \dots\ unit-key-3}
%     \ \ \ \ \marg{unit-key-4} \marg{1 unit-key-1 are \dots\ unit-key-4}
%     \ \ \ \ \dots
%     \ \ \}
%    
%     \mycs{cudefinesinglekey}\marg{unit-key-1}
%     \ \ \{
%     \ \ \ \ \marg{unit-key-2} \marg{1 unit-key-2 are \dots\ unit-key-1}
%     \ \ \ \ \marg{unit-key-3} \marg{1 unit-key-3 are \dots\ unit-key-1}
%     \ \ \ \ \dots
%     \ \ \}
%   \end{syntax}
%
%   If you define new units (see \cref{sec:def-units}) and cannot add
%   them to already existing keys you can use
%   \lstinline|\cudefinekeys| bzw.  \lstinline|\cudefinesinglekey| to
%   define new keys.
%
%   \lstinline|\cudefinekeys| takes  \marg{unit-key-1} as a
%   \enquote{basis}, defines a key with the name \meta{unit-key-1} and
%   adds the values \meta{unit-key-1}, \meta{unit-key-2},
%   \meta{unit-key-3}, etc. Furthermore this command also defines the
%   keys \meta{unit-key-2}, \meta{unit-key-3}, etc.  with the same
%   values as \meta{unit-key-1}. Please note that \meta{\dots} has to
%   be a number. 
%  
%   Sometimes it is not that easy and the conversion of one unit into
%   another needs are more complicated formula (see for example
%   temperatures).  If that is the case use
%   \lstinline|\cudefinesinglekey|. As the name says it defines
%   \emph{only} the key \meta{unit-key-1} with the values
%   \meta{unit-key-1}, \meta{unit-key-2}, etc. The advantage of this
%   command is that now \meta{\dots} can be a formula and the
%   numerical input can be placed explicitly using \texttt{\#1}.
%
%
% \end{function}
%
% \begin{cuexample}
%
% This example defines following keys with their respective value:
% \begin{itemize}
% \item the key \texttt{kg} with the values \texttt{kg}, \texttt{dag},
%   \texttt{g} and \texttt{oz}
% \item the key \texttt{dag} with the values \texttt{kg},
%   \texttt{dag}, \texttt{g} and \texttt{oz}
% \item the key \texttt{g} with the values \texttt{kg}, \texttt{dag},
%   \texttt{g} and \texttt{oz}
% \item the key \texttt{oz} with the values \texttt{kg}, \texttt{dag},
%   \texttt{g} and \texttt{oz}
% \item the key \texttt{d} with the values \texttt{d}, \texttt{h},
%   \texttt{min} and \texttt{s}
% \item \dots
% \end{itemize}
%
%
% \begin{align*}
%   \SI{1}{\kg}&= \SI{1}{\kg} &
%   \SI{1}{\kg}&= \SI{100}{\deka\g} &
%   \SI{1}{\kg}&= \SI{1000}{\g}  \\
%   \SI{1}{\kg}&= \SI{35.27399}{\ounce} &
%   \SI{1}{\kg}&= \SI{2.204 622 6}{\pound} 
% \end{align*}
%
% \begin{lstlisting}
%  \cudefinekeys {kg} 
%    {
%      {dag}{ 100 } %% 1 kg are 100 dag
%      {g}  { 1000 } %% 1 kg are 1000 g
%      {oz} { 35.27399 }  %% 1 kg are 35.27399 oz
%      {lb} { 2.204 622 6 } %% 1 kg are  2.204 622 6 lb
%    }
%
%  \cudefinekeys {d}
%    {
%      {h}  { 24 } %% 1 day are 24 hours
%      {min}{ 1440 } %% 1 day are 1440 minutes
%      {s}  { 86400 } %% 1 day are 86400 seconds
%    }
% \end{lstlisting}
%
% To convert degree Fahrenheit to degree Celsius, kelvin and degree R\'eamur one 
% needs the formulas\footnote{See Wikipedia.}
% \begin{align*}
%   T_C &= ( T_F - 32 ) \cdot \frac{5}{9} \\
%   T_K &= ( T_F - 459.67 ) \cdot \frac{5}{9} \\
%   T_{Re} &= ( T_F - 32 ) \cdot \frac{4}{9} 
% \end{align*}
% with $T_F$ being the input temperature in degree Fahrenheit and
% $T_C$ being the same temperature in degree Celsius, etc. Using
% \lstinline|\cudefinesinglekey| the key \texttt{F} with values
% \texttt{C}, \texttt{K} and \texttt{Re} is defined:
% 
% \begin{lstlisting}
%  \cudefinesinglekey {F}
%    {
%      {C}  { ( #1 - 32 ) *  5/9 } %% see formulas above
%      {K}  { ( #1 + 459.67 ) *  5/9 }
%      {Re} { ( #1 - 32 ) * 4/9 } 
%    }
% \end{lstlisting}
%
% This defines the key \texttt{F} with the values \texttt{F}, \texttt{C}, 
% \texttt{K} and \texttt{Re}.
%
% \end{cuexample}
%
%
%
% \begin{function}{\cuaddkeys,\cuaddsinglekeys}
%
%   \begin{syntax}
%     \mycs{cuaddkeys}\marg{unit-key-1}
%     \ \ \{
%     \ \ \ \ \marg{unit-key-2} \marg{1 unit-key-1 are \dots\ unit-key-2}
%     \ \ \ \ \marg{unit-key-3} \marg{1 unit-key-1 are \dots\ unit-key-3}
%     \ \ \ \ \marg{unit-key-4} \marg{1 unit-key-1 are \dots\ unit-key-4}
%     \ \ \ \ \dots
%     \ \ \}
%    
%     \mycs{cuaddsinglekeys}\marg{unit-key-1}
%     \ \ \{
%     \ \ \ \ \marg{unit-key-2} \marg{1 unit-key-2 are \dots\ unit-key-1}
%     \ \ \ \ \marg{unit-key-3} \marg{1 unit-key-3 are \dots\ unit-key-1}
%     \ \ \ \ \dots
%     \ \ \}
%   \end{syntax}
%
%   These commands add \meta{unit-key-2}, etc. to the already defined
%   key \meta{unit-key-1}.
%
%   \lstinline|\cuaddkeys| takes the already defined key
%   \marg{unit-key-1} as a \enquote{basis}, and adds
%   \meta{unit-key-2}, \meta{unit-key-3}, etc. to its
%   values. Furthermore it adds those new values to other keys linked
%   to \meta{unit-key-1} and defines the new keys \meta{unit-key-2},
%   etc. with the same values as \meta{unit-key-1}.
%  
%  
%   If the conversion is more complicated use
%   \lstinline|\cuaddsinglekeys|.  It adds \meta{unit-key-2}, etc. as
%   values to \meta{unit-key-1}.  The numerical input can be placed
%   using \texttt{\#1} (see \lstinline|\cudefinesinglekey|).  This
%   command neither defines new keys nor does it add values to other
%   keys than \meta{unit-key-1}.
%
%
% \end{function}
%
%
% \begin{cuexample}
%   Suppose you are British (I am sorry, I can't think of another
%   reason to use those units) and you want to implement 'stone' (yes,
%   I was surprised myself that such a unit exists, but it even
%   appears in a Sherlock Holmes story). You exactly know that
%   \SI{1}{\st} equals \SI{14}{\lb}, well \dots\ now you have two
%   choices. \lstinline|\cuaddkeys| or \lstinline|\cuaddtokeys| (use
%   the one best fitting). This example uses the first, the next the
%   latter one.
% \begin{lstlisting}
%  \newcookingunit{st} %% defining new unit 'stone'
%  \cuaddkeys{lb}  %% adding st to lb (could also add to kg, dag and oz)
%    { 
%      {st} { 1/14 }  %% 1 lb are 1/14 st as 14 lb are 1 st
%    }        
% \end{lstlisting}
% \begin{cuexamplecode}
% \cunum[lb=st]{1}{lb} \\
% \cunum[st=lb]{1}{st} \\
% \cunum[st=g]{1}{st} \\
% \cunum[st=kg]{1}{st} \\
% \cunum[kg=st]{1}{kg} \\
% \cunum[st=kg]{16}{st} 
% \end{cuexamplecode}
% \end{cuexample}
%
%
% \begin{cuexample}
%   Now you want to add degree R{\o}mer and convert Celsius to degree
%   R{\o}mer:
% \begin{equation*}
%   T_{R\text\o} = T_C * \frac{21}{40} + 7.5
% \end{equation*}
%
% \begin{lstlisting}
%  %% defining new unit 'degree R{\o}mer'
%  \newcookingunit [\ensuremath{ {} ^ { \circ } }\kern-\scriptspace R{\o}] {Ro} 
%  \cuaddsinglekeys {C} %% adds value 'Ro' to 'C'.
%    {
%      {Ro} { #1 * 21/40 + 7.5 }
%    }
%  \cusetup %% round to integer automatically
%    {
%      set-option-for-Ro = { round-to-int = true }
%    }
% \end{lstlisting}
% \begin{cuexamplecode}
% \cunum{10}{C} \\
% \cunum[C=Ro]{10}{C} 
% \end{cuexamplecode}
% \end{cuexample}
%
%
%
%
%
%
%
%
%
%
%
% \begin{function}{\cuaddtokeys}
%   \begin{syntax}
%     \mycs{cuaddtokeys} \marg{unit-key-1} \marg{unit-key-2} \marg{1 unit-key-2 are \dots\ unit-key-1}
%   \end{syntax}
%   Works similar to \mycs{cuaddkeys} regarding the definition of keys.
% \end{function}
%
% \begin{cuexample}
% Continuing the example from before, this time with \lstinline|\cuaddtokeys|:
% \begin{lstlisting}
%  \newcookingunit{st} %% defining (again) new unit 'stone'
%  \cuaddtokeys {lb} {st} { 14 }  %% 1 st are 14 lb
% \end{lstlisting}
% \begin{cuexamplecode}
% \cunum[lb=st]{1}{lb} \\
% \cunum[st=lb]{1}{st} \\
% \cunum[st=g]{1}{st} \\
% \cunum[st=kg]{1}{st} \\
% \cunum[kg=st]{1}{kg} \\
% \cunum[st=kg]{16}{st} 
% \end{cuexamplecode}
% \end{cuexample}
%
% \begin{comment}
% \begin{cuexample}
% Implementation of \texttt{stick}.  From https://cooking.stackexchange.com/questions/784/translating-cooking-terms-between-us-uk-au-ca-nz
% we know that 
% \[ \SI{1}{\stick} = \SI{0.25}{\lb} \]
%
% There are two ways to implement \texttt{stick}:
% either use \mycs{cuaddkeys} or \mycs{cuaddtokeys}.
%
% The easier way (in this case) is \mycs{cuaddtokeys} which is pretty straight forward:
% \begin{lstlisting}
%   \cuaddtokeys { lb } { st } { 0.25 }
% \end{lstlisting}
% \dots\ and finished
% 
% \end{cuexample}
% \end{comment}
%
%
%
%
%
%
%
%
%
%
%
%
%
%
% \section{Language support}
% \label{sec:language}
%
% Unit names and symbols depend on the language.  To change the
% name depending on the language you can use \lstinline|\cudefinename|
% and to only change symbols use \lstinline|\cudefinesymbol|.
%
% \begin{function}{decimal-mark, one(m),one(f),one(n)}
%   Those are special keys (as they cannot be used as units). Not only
%   are printed units language depending, but as is the decimal mark
%   (\enquote{.} or \enquote{,}). To set the decimal mark use
%   \opt{decimal-mark} (see examples below).
%
%   Furthermore if you are using the package-option \opt{use-numerals}
%   you may also use the keys \opt{one(m)}, \opt{one(f)} and
%   \opt{one(n)}. If you use this option, integers below a certain
%   value (see option \opt{use-numerals-below}) are written-out. The
%   only problem is the written-out \enquote{1} mostly depends on the
%   gender of the following word (e.g. \enquote{ein Baum} (m),
%   \enquote{eine Pflanze} (f) and \enquote{ein Auto} (n)). To set the
%   written-out 1 to be correct with the gender of the used unit, use
%   these keys (see also examples below)
% \end{function}
%
% \begin{function}{\cudefinename}
%   \begin{syntax}
%     \mycs{cudefinename}\marg{Language}
%      \ \ \{
%      \ \ \ \ \marg{unit-key-1} \oarg{symbol-1} \marg{singular-1} \oarg{plural-1} \meta{gender}
%      \ \ \ \ \marg{unit-key-2} \oarg{symbol-2} \marg{singular-2} \oarg{plural-2} \meta{gender}
%      \ \ \ \ \dots
%      \ \ \}
%   \end{syntax}
%   
%   This command defines the names (and optionally the symbol) of the
%   units printed in \lstinline|\cutext| and \lstinline|\Cutext|
%   (and \lstinline|\cunum| regarding the symbol) for the specific
%   \meta{Language}. For details regarding \meta{language} see the
%   \pkg{translations} documentation.
%
%   If the plural form of the name differs from the singular form use
%   \oarg{plural} to specify the plural form, if no \oarg{plural} is
%   given the plural will be set equal to its singular. The singular form
%   is only used if the number in \lstinline|\cutext| and
%   \lstinline|\Cutext| is equal to 1.
%
%   \meta{gender} can be \texttt{m} (maskulin), \texttt{f} (feminin) or
%   \texttt{n} (neutrum). If not given \texttt{m} is used as default.
% \end{function}
%
% \begin{lstlisting}
%  \cudefinename {English}
%    {
%      {kg}  {kilogramme}
%      {oz}  {ounce} 
%      {h}   {hour} [hours]
%      {C}   {degree\space Celsius} [degrees\space Celsius]
%      {decimal-marker} {.}
%      {one(m)} {one}
%      {one(f)} {one}
%      {one(n)} {one}
%   }
% \end{lstlisting}
% \begin{lstlisting}
%  \cudefinename {German}
%    {
%      {kg}  {Kilogramm} <n>
%      {oz}  {Unze} <f>
%      {d}   {Tag} [Tage]
%      {h}   {Stunde} [Stunden] <f>
%      {C}   {Grad\space Celsius}
%      {decimal-marker} {,}
%      {one(m)} {ein}
%      {one(f)} {eine}
%      {one(n)} {ein}
%    }
% \end{lstlisting}
%
%
% \begin{function}{\cudefinesymbol}
%
%   \begin{syntax}
%     \mycs{cudefinesymbol}\marg{Language}
%      \ \ \{
%      \ \ \ \ \marg{unit-key-1} \marg{symbol-1} 
%      \ \ \ \ \marg{unit-key-2} \marg{symbol-2} 
%      \ \ \ \ \dots
%      \ \ \}
%   \end{syntax}
%   
%   This command defines the symbols of the units printed in
%   \lstinline|\cunum| for the specific \meta{language}.  It works
%   similar as \lstinline|\cudefinename|, but only the symbols (and no
%   names) can be set.   For details regarding \meta{language} see
%   the \pkg{translations} documentation.
%
% \end{function}
%
%
% \begin{lstlisting}
%  \cudefinesymbol {English}
%    {
%      {decimal-mark} {.}
%      {one(m)} {one}
%      {one(f)} {one}
%      {one(n)} {one}
%    }
%  \cudefinesymbol {German}
%    {
%      {decimal-mark} {,}
%      {one(m)} {ein}
%      {one(f)} {eine}
%      {one(n)} {ein}
%    }
%  \cudefinesymbol {French}
%    {
%      {l} {L}
%      {dl} {dL}
%      {cl} {cL}
%      {ml} {mL}
%      {decimal-mark} {.}
%      {one(m)} {un}
%      {one(f)} {une}
%      {one(n)} {un}
%    }
% \end{lstlisting}
%
% \begin{cuexample}
%  Imagine that instead of the abbreviation \enquote{dag} for \enquote{decagramme}
%  you want to use \enquote{ducks} (because \dots\ I don't know). You can easily
%  do this via
%  \begin{lstlisting}
%  \cudefinesymbol {English}
%    {
%      {dag} {ducks}
%    }
%  \end{lstlisting}
%  \cudefinesymbol {English}
%    {
%      {dag} {ducks}
%    }
% As you can see it may be a bit suboptimal as there is no plural version allowed.
% You do it anyway and end up with:
%  \begin{cuexamplecode}
%    \cunum{12}{dag} & weed\\
%    \cunum{3}{dag} & nuts\\
%    \cunum{10}{dag} & duckmeat
%  \end{cuexamplecode}
%
%
% \end{cuexample}
%
%
%
%
%
%
% \subsection{Phrases}
% \label{sec:phrases}
%
% Each language has synonyms for certain (integer) numbers. This package
% supports those phrases and they can be implemented with the following
% command and used by \mycs{cuam}:
%
% \begin{function}{\cudefinephrase}
%   \begin{syntax}
%     \mycs{cudefinesymbol}\marg{Language}
%      \ \ \{
%      \    \marg{integer-1}   \marg{phrase-1} \oarg{phrase-1-plural} \meta{gender-1}
%      \    \marg{integer-2} * \marg{phrase-2} \oarg{phrase-2-plural} \meta{gender-2}
%      \    \dots
%      \ \ \}
%   \end{syntax}
%   
%   This command pairs for a given \marg{Language} the number \marg{integer-1}
%   with \marg{phrase-1} (plural and gender). The package then checks if the amount
%   given in \mycs{cuam} is either this number or a \emph{multiple} of it.
%
%   If the behavior of checking for a multiple is not wanted, you can use the optional
%   star \texttt{*} for a given \marg{integer}
%
%   \meta{gender} can be \opt{m}, \opt{f} or \opt{n}. It is \opt{m} by default.
%   
%   Afterwards the numbers are ordered from highest to lowest so that
%   the phrase with the highest number is used (if used at all).
%
%   Furthermore, it chooses star (|*|) phrases over non-star phrases.   
%
%   \begin{cunote}
%     Numbers with the optional star |*| are stored as negative numbers. 
%   \end{cunote}
% \end{function}
%
% \begin{cuexample}
%   The following example creates some phrases for the language
%   \enquote{German}:
%   
% \cudefinephrase {German}
%    {
%      { 60 }  {Schock} <n>
%      { 6  }* {halbes\  Dutzend} <n>
%    }
%
%
% \begin{lstlisting}
%  \cudefinephrase {German}
%    {
%      { 12 }  {Dutzend} <n> %% implemented by default
%      { 60 }  {Schock} <n>
%      { 6  }* {halbes\ Dutzend} <n>
%    }
% \end{lstlisting}
%   Let's just use them (activating the german language):
%   \begin{myde}
%   \begin{cuexamplecode}
%     \cusetup{use-phrases=true} @
%     \cuam{12} \\
%     \cuam{24} \\
%     \cuam{60} \\
%     \cuam{120} \\
%     \cuam{6} \\
%     \cuam{18} 
%   \end{cuexamplecode}
%   \end{myde}
%
% As you can see, \enquote{Schock} ($60$) is preferred over \enquote{Dutzend} ($12$)
% as it linked to the higher number. Furthermore, for $6$ the phrase \enquote{halbes
% Dutzend}  (half a dozen) is used, but because it is a star version it is \emph{not}
% used for $18$.
%
% \end{cuexample}
% 
%
% \section{Options}
% \label{sec:options}
%
% Options in \pkg{cooking-units} can mostly be set globally using
% \lstinline|\cusetup| or locally using the optional argument of the
% respective command (but \emph{not} as a package option).  The only
% exception is the option given in \cref{sec:load-time-options} which
% needs to be used as a package option.
%
% \subsection{Load time options}
% \label{sec:load-time-options}
% 
% \begin{function}{use-numerals}
%   \begin{syntax}
%     \cs{usepackage}[use-numerals=\meta{true/false}]\{cooking-units\}
%   \end{syntax}
%   If set to \true\ loads package \pkg{fmtcount} and uses 
%   \cs{numberstringnum}  for \mycs{cutext}
%   and \cs{Numberstringnum} for \mycs{Cutext} to write-out numbers below
%   \opt{use-numerals-below} (13 by default), integers above are
%   printed as numbers. 
%   You can decide to not print any numerals by setting \opt{print-numerals}
%   to \false.
%
%   \begin{cunote}
%     \opt{use-numerals} is a package option as it needs to load
%     \pkg{fmtcount} which is not loaded by default.
%   \end{cunote}
%
%   \begin{cunote}
%   Please note the keys \opt{one(m)}, \opt{one(f)}
%   and \opt{one(n)} to change the printed \enquote{one} (as
%   \enquote{one} is in many languages dependent on the gender of the
%   following word. E.g in German:  Masculine: ein Baum,
%   Feminin: eine Pflanze, Neutrum: ein Auto).
%   \end{cunote}
%
% \end{function}
% 
% \cusetup{print-numerals=true}
% \begin{cuexamplecode}
%   \cutext{1}{kg} \\
%   \Cutext{1}{kg} \\
%   \cutext{2}{kg} \\
%   \Cutext{2}{kg} \\
%   \cutext{12}{kg} \\
%   \cutext{13}{kg} \\
%   \cutext{13}{kg} \\
%   \Cutext{14}{kg} 
% \end{cuexamplecode}
% \cusetup{print-numerals=false}
% 
% \subsection{Normal options}
%
% Options in this subsection can only be set as local options or using
% \lstinline|\cusetup|, but \emph{not}  as load time options.
% 
% \begin{function}{\cusetup}
%   Options can be set using \mycs{cusetup}\Arg{options}.
% \end{function}
% 
% \subsubsection{Unit Specific options} 
%
% \begin{function}{unit}
%   \begin{syntax}
%     \meta{unit-key-1} = \meta{unit-key-2}
%  \end{syntax}
%  Change unit \meta{unit-key-1} to \meta{unit-key-2} (see
%  \cref{sec:defining-opt} to define new options).
% \end{function}
% 
% \begin{function}{set-option-for-<unit-key>,add-option-for-<unit-key>,erase-all-options}
%   \begin{syntax}
%     set-option-for-\meta{unit-key} = \meta{key1=value1,\dots}
%     add-option-for-\meta{unit-key} = \meta{key1=value1,\dots}
%    erase-all-options
%  \end{syntax}
%  Sets and adds \meta{key1=value1,\dots} to a specific
%  \meta{unit-key},
%  \texttt{erase-all-options} is used to erase all options for all
%  \meta{unit-key}s.
%  
%  You may want to attach some options to a special
%  \meta{unit-key}. Those options are automatically activated if (and
%  only if) the specific \meta{unit-key} is used (or changed into this
%  unit). Setting options overwrites old options. Adding options, well
%  \dots\ adds the options to the old ones.
%  
%  The following rounds the values to integers for \texttt{F}, \texttt{C}, 
%  \texttt{K} and \texttt{Re}.
%  
% \end{function}
% 
% \begin{lstlisting}
%  \cusetup 
%    { 
%      set-option-for-F  = { round-to-int = true } ,
%      set-option-for-C  = { round-to-int = true } ,
%      set-option-for-K  = { round-to-int = true } ,
%      set-option-for-Re = { round-to-int = true } 
%    }
% \end{lstlisting}
%
% You can \enquote{delete} the  options by setting
% an empty value for a specific \meta{unit-key} (or use
% \opt{erase-all-options} to erase all options for all \meta{unit-key}s)
%
% \subsubsection{Command behavior}
%
% \begin{function}{cutext-to-cunum}
%   \begin{syntax}
%     cutext-to-cunum = \meta{true/false}
%   \end{syntax}
%   Want to get rid of all \cCutext? Set this option to \opt{true} and all
%   \cCutext\ are changed into \mycs{cunum}.
% \end{function}
% \begin{cuexamplecode}
%   \cutext{1}{kg} \\
%   \Cutext{2}{kg} \\
%   \cutext{1/2}{kg} \\
%   \cutext{?}{kg} \\
%   \cutext[kg=g]{1--2}{kg} \\
%   \cusetup{cutext-to-cunum = true } @
%   \cutext{1}{kg} \\
%   \Cutext{2}{kg} \\
%   \cutext{1/2}{kg} \\
%   \cutext{?}{kg} \\
%   \cutext[kg=g]{1--2}{kg}
% \end{cuexamplecode}
%
%
% \begin{function}{cutext-change-unit}
%   \begin{syntax}
%     cutext-change-unit = \meta{true/false}
%   \end{syntax}
%   Set this option to \opt{true} if you do \emph{not} want the units of \cCutext\
%  to be changed. Set to \true\ by default
% \end{function}
% \begin{cuexamplecode}
%   \cutext[kg=g]{1}{kg} \\
%   \cutext[kg=g]{1/2}{kg} \\
%   \cutext[kg=g]{1--2}{kg} \\
%   \cusetup{cutext-change-unit=false } @
%   \cutext[kg=g]{1}{kg} \\
%   \cutext[kg=g]{1/2}{kg} \\
%   \cutext[kg=g]{1--2}{kg}
% \end{cuexamplecode}
%
% \begin{function}{cuam-version,cutext-version}
%   \begin{syntax}
%     cuam-version = \meta{old/new}
%     cutext-version = \meta{old/new}
%   \end{syntax}
%   Since v1.10 this package also parses and checks the input of
%   \cCutext\ and \mycs{cuam}. If you want to restore the old
%   behavior, set this option to \opt{old}, but note that then you can
%   neither change the amounts for a given number of persons nor
%   change the unit of \cCutext. Both of them are set to \opt{new} by
%   default.
% \end{function}
%
% \subsubsection{Input and Outputs}
%
% \begin{function}{set-special-sign,add-special-sign}
%   \begin{syntax}
%     set-special-sign = \meta{character(s)}
%     add-special-sign = \meta{character(s)}
%   \end{syntax}
%   Allows \meta{character(s)} to be used in the first mandatory argument of 
%  \mycs{cunum}, \mycs{cuam}, \cCutext{} without raising an error (you can customize this
%  behavior, see \opt{set-unknown-message}). By default it is set to |?|.
%  Please note that the sign |<| is not allowed as a special sign.
% \end{function}
%
% \begin{cuexamplecode}
%   \cunum{?}{kg} \\
%   \cunum[g=kg]{10?--20?}{kg} \\
%   \cusetup{add-special-sign={xX} } @
%   \cunum{x}{kg} \\
%   \cunum{X--?}{kg} \\
%   \cusetup{set-special-sign={} } @
%   \cunum{1}{kg} \\
%   \cunum{1--2}{kg} 
% \end{cuexamplecode}
%
% \begin{function}{set-unknown-message}
%   \begin{syntax}
%     set-unknown-message = \meta{error/warning/none}
%   \end{syntax}
%   Using a special sign (\texttt{?} by default) causes a warning to
%   be raised. Set this option to \opt{error} if you want an error (as
%   an extra emphasis), \opt{warning} if you want a warning (default)
%   and \opt{none} if you don't want to know anything about it.
% \end{function}
%
% \begin{function}{set-cutext-translation-message}
%   \begin{syntax}
%     set-cutext-translation-message = \meta{error/warning/none}
%   \end{syntax}
%   If a translation for \cCutext\ is not available the commands are replaced
%   by \mycs{cunum}. Currently -- if this is happening -- a warning is shown,
%  you may change the behavior of the message (error, warning or not showing at all)
%  using this option.
% \end{function}
%
% \begin{function}{print-numerals}
%   \begin{syntax}
%     print-numerals = \meta{true/false}
%   \end{syntax}
%   If the package option \opt{use-numerals} is set to \true\ you can
%   deactivate the printing of numerals by setting \opt{print-numerals} to 
%   \false\ and activate them by setting it to \true.
%
%  Note that this option is automatically set to \opt{true} if \opt{use-numerals} is used.
% \end{function}
% \cusetup{ print-numerals=true }
% \begin{cuexamplecode}
%   \cutext{1}{kg} \\
%   \cutext{2}{kg} \\
%   \cutext{12}{kg} \\
%   \cutext{13}{kg} \\
%   \cusetup{ print-numerals=false } @
%   \cutext{1}{kg} \\
%   \cutext{2}{kg} \\
%   \cutext{12}{kg} \\
%   \cutext{13}{kg} \\
% \end{cuexamplecode}
% \cusetup{ print-numerals=false }
%
% \begin{function}{use-numerals-below}
%   \begin{syntax}
%     use-numerals-below = \meta{integer}
%   \end{syntax}
%   Only usable if the package option \opt{use-numerals} is
%   active. Prints the name of the numbers for integers used in
%   \mycs{cutext} and \mycs{Cutext} smaller than
%   \meta{integer}. \meta{integer} is by default 13. Package
%   \pkg{fmtcount} is used for this purpose.
% \end{function}
% \cusetup{ print-numerals=true }
% \begin{cuexamplecode}
%   \cutext{1}{kg} \\
%   \cutext{2}{kg} \\
%   \cutext{12}{kg} \\
%   \cutext{13}{kg} \\
%   \cusetup{ use-numerals-below=10 } @
%   \cutext{1}{kg} \\
%   \cutext{2}{kg} \\
%   \cutext{12}{kg} \\
%   \cutext{13}{kg} \\
%   \cusetup{ use-numerals-below=0 } @
%   \cutext{1}{kg} \\
%   \cutext{2}{kg} \\
%   \cutext{12}{kg} \\
%   \cutext{13}{kg} \\
%   \cusetup{ use-numerals-below=12001 } @
%   \cutext[kg=g]{1}{kg} \\
%   \cutext[kg=g]{2}{kg} \\
%   \cutext[kg=g]{12}{kg} \\
%   \cutext[kg=g]{13}{kg} \\
% \end{cuexamplecode}
% \cusetup{ print-numerals=false } 
%
%
% \begin{function}{parse-number}
%   \begin{syntax}
%     parse-number = \meta{true/false}
%   \end{syntax}
%   If set to \texttt{false} prints the number of \lstinline!\cunum!,
%   \lstinline!\cutext!, \lstinline!\Cutext! and \mycs{cuam} as they
%   are (after some \dots\ well \dots\ parsing due to
%   \enquote{\lstinline|_|}). Is set to \texttt{true} by default.
% \end{function}
%
% \begin{cuexamplecode}
%   \cusetup{ parse-number=false } @
%   \cunum[kg=g]{1}{kg} \\
%   \cunum{1--2}{kg} \\
%   \cunum{1----------2}{kg} \\
%   \cunum{1.2}{kg} \\
%   \cunum[kg=g]{1,2}{kg} \\
%   \cunum{1/2}{kg} \\
%   \cunum{1_2/3}{kg} \\
%   \cunum{1/2_3}{kg} \\
%   \cunum{qwertzuiop}{kg} \\
%   \cutext{1}{kg} \\
%   \cutext{100}{kg} \\
%   \cutext{gjfak}{kg} \\
%   \cutext[kg=g]{12}{kg} \\
%   \cuam{1----------2} \\
%   \cuam{1,2} \\
%   \cuam{1_1/2} \\
%   \cuam{kwflk} \\
% \end{cuexamplecode}
%
%
%
% \begin{function}{range-sign}
%   \begin{syntax}
%     range-sign = \meta{string}
%     cunum-range-sign = \meta{string}
%     cutext-range-sign = \meta{string}
%   \end{syntax}
%   The second sets the \emph{printed} range sign used in \mycs{cunum}
%   (and \mycs{cuam}) to \meta{string}, the third sets the printed
%   range sign used in \cCutext{} to \meta{string}.
%   Using the first option sets the
%   range signs for both \mycs{cunum} (and \mycs{cuam}) and
%   \mycs{cutext}/\mycs{Cutext} to \meta{string}.
%
%  The default for \meta{string} is \texttt{-{}-} (for both).
% \end{function}
%
% \begin{cuexamplecode}
%   \cusetup{cunum-range-sign={~to~}} @
%   \cunum{1--2}{kg} \\
%   \cuam{1--2} \\
%   \cutext{1--2}{kg} \\
%   \Cutext{1--2}{kg} 
% \end{cuexamplecode}
% \begin{cuexamplecode}
%   \cusetup{cutext-range-sign={~to~}} @
%   \cunum{1--2}{kg} \\
%   \cuam{1--2} \\
%   \cutext{1--2}{kg} \\
%   \Cutext{1--2}{kg} 
% \end{cuexamplecode}
% \begin{cuexamplecode}
%   \cusetup{range-sign={~to~}} @
%   \cunum{1--2}{kg} \\
%   \cuam{1--2} \\
%   \cutext{1--2}{kg} \\
%   \Cutext{1--2}{kg} 
% \end{cuexamplecode}
%
% \begin{function} {use-phrases}
%   \begin{syntax}
%     use-phrases = \meta{true/false}
%   \end{syntax}
%  Setting this option to \true\ replaces certain integers (see \cref{sec:phrases} for
%  more information) with their phrase counterpart. This option is set to \false\ by default.
% \end{function}
%
% \begin{myde}
% For the German langauge:
% \begin{cuexamplecode}
%   \cuam{12} \\
%   \cuam{12--24} \\
%   \cuam{36} \\
%   \cusetup{use-phrases=true} @
%   \cuam{12} \\
%   \cuam{12--24} \\
%   \cuam{36} \\
%   \cusetup{use-phrases=true, print-numerals=true} @
%   \cuam{12} \\
%   \cuam{12--24} \\
%   \cuam{36} \\
% \end{cuexamplecode}
% \end{myde}
%
%
% \subsubsection{Rounding options}
%
% \begin{function}{round-precision}
%  \begin{syntax}
%    round-precision = \meta{integer}
%  \end{syntax}
%  Rounds the amount automatically to \meta{integer} digits after the
%  colon.  Note that units like \texttt{C}, \texttt{F}, \texttt{K} and
%  \texttt{Re} are still rounded to integers due to
%  \texttt{set-option-for-<unit-key>}.
% \end{function}
%
%  
% \begin{cuexamplecode}
%   \cusetup{round-precision= 5} @
%   \cunum{1.23456789}{kg} \\
%   \cunum[g=kg]{12.587}{g} \\
%   \cunum{194}{kg} \\
%   \cunum[C=F]{200--210}{C} \\
%   \cunum[K=C]{0.0012}{K} \\
%   \cusetup{round-precision= 1} @
%   \cunum{1.23456789}{kg} \\
%   \cunum{12.58}{kg} \\
%   \cunum[g=kg]{194}{g} \\
%   \cunum[C=F]{200--210}{C} \\
%   \cunum[K=C]{0.0012}{K} 
% \end{cuexamplecode}
%
% \begin{cunote}
% \begin{cuexamplecode}
%   \cusetup{erase-all-options} @
%   \cusetup{set-option-for-C={round-precision=-1}} @
%   \cusetup{set-option-for-F={round-precision=-1}} @
%   \cunum[K=C]{0.0012}{K} \\
%   \cunum{185}{C} \\
%   \cunum[C=F]{180--200}{C} \\
%  \end{cuexamplecode}
% \end{cunote}
%
% \begin{function}{round-to-int}
%   \begin{syntax}
%     round-to-int = \meta{true/false}
%   \end{syntax}
%   Rounds the amount to an integer if set \texttt{true}.
% \end{function}
%
%
% \begin{cuexamplecode}
%   \cusetup{round-to-int=true} @
%   \cunum{1.23456789}{kg} \\
%   \cunum{12.58}{kg} \\
%   \cunum[g=kg]{194--294}{g} \\
%   \cunum[kg=g]{1.23456789}{kg}
% \end{cuexamplecode}
%
% \begin{function}{round-half}
%   \begin{syntax}
%     round-half = \meta{default/commercial}
%   \end{syntax} 
%   This option is only important for half-way numbers
%   (e.g. $0.005$). By setting it to \opt{default} the value will be
%   rounded to the nearest even number.  Setting it to
%   \opt{commercial} rounds the value away from zero.
%
% It is set to \opt{default} by \dots\ default.
%
%  \begin{cunote}
%    \opt{default} actually refers to the fact that it is the default
%    rounding algorithm used by \cs{fp\_eval:n \{ round( ) \}}
%    without a third argument.
%  \end{cunote}
%
% \end{function}
%
%
% \begin{cuexamplecode}
%   \cusetup{round-half=default} @
%   \cunum{0.005}{kg} \\
%   \cunum{-0.005}{kg} \\
%   \cunum{1.245}{kg} \\
%   \cusetup{round-half=commercial} @
%   \cunum{0.005}{kg} \\
%   \cunum{-0.005}{kg} \\
%   \cunum{1.245}{kg} @
% \end{cuexamplecode}
%
%
%
% \subsubsection{Fractions}
%
% \begin{function}{eval-fraction}
%  \begin{syntax}
%    eval-fraction = \meta{true/false}
%  \end{syntax}
%   This option takes \texttt{true} or \texttt{false} as values.
%   If set to \texttt{true} fractions are evaluated. Please note that divisions
%  through zero are not allowed.
% \end{function}
%
% \begin{cuexamplecode}
%   \cusetup{eval-fraction=true} @
%   \cunum{1/3}{kg} \\
%   \cunum{1/2}{kg} \\
%   \cunum[kg=g]{1/2}{kg} \\
%   \cunum{1_1/2}{kg} \\
%   \cunum[kg=g]{1_1/2}{kg} \\
%   \cunum[kg=g]{-1_1/2}{kg} \\
% \end{cuexamplecode}
%
% \begin{function}{fraction-command}
%   \begin{syntax}
%     fraction-command = \meta{\cs{command}}
%   \end{syntax}
%   Sets the command used for printing fractions equal to \meta{\cs{command}}.
%   \meta{\cs{command}} has to take two arguments. By default
%   it is equal to |\sfrac| from \pkg{xfrac}.
%
%   Please note that the amount is \emph{not} printed inside a math
%   environment by default.
% \end{function}
% 
% \begin{cuexamplecode}
%   \newcommand\myfrac[2]{#1/#2} @
%   \cusetup{fraction-command=\myfrac}  @
%   \cuam{1/8} \\
%   \cunum{1/2}{kg} \\
%   \cunum{4/5}{C} \\
%   \cunum{1_2/3}{kg} \\
%   \cusetup{fraction-command=\nicefrac}  @
%   \cuam{1/8} \\
%   \cunum{1/2}{kg} \\
%   \cunum{4/5}{C} \\
%   \cunum{1_2/3}{kg} 
% \end{cuexamplecode}
%
% \begin{function}{fraction-inline}
%   \begin{syntax}
%     fraction-inline = \meta{input containing \#1 and \#2}
%   \end{syntax}
%   Similar to \texttt{fraction-command} only that you don't have to 
%   define a command  to alter the output of the fraction.
% \end{function}
% 
% \begin{cuexamplecode}
%   \cusetup{fraction-inline={#1/#2} }  @
%   \cuam{1/8} \\
%   \cunum{1/2}{kg} \\
%   \cunum{4/5}{C} \\
%   \cunum{1_2/3}{kg} \\
%   \cusetup{fraction-inline={\nicefrac{#2}{#1}} }  @
%   \cuam{1/8} \\
%   \cunum{1/2}{kg} \\
%   \cunum{4/5}{C} \\
%   \cunum{1_2/3}{kg} 
% \end{cuexamplecode}
%
% \subsubsection{Spaces}
%
% \begin{function}{mixed-fraction-space}
%   \begin{syntax}
%     mixed-fraction-space = \meta{length}
%   \end{syntax}
%   Sets the length between the fraction and the number in a mixed-fraction,
%   default is \texttt{0.1em} (because I said so; if someone has some literature
%   or sources to look up the space, please let me know).
% \end{function}
% \begin{cuexamplecode}
%   \cuam{1_2/3} \\
%   \cunum{1_2/3}{kg} \\
%   \cunum{10_2/3}{kg} \\
%   \cusetup{mixed-fraction-space=1em } @
%   \cuam{1_2/3} \\
%   \cunum{1_2/3}{kg} \\
%   \cunum{10_2/3}{kg} \\
%   \cusetup{mixed-fraction-space=0em } @
%   \cuam{1_2/3} \\
%   \cunum{1_2/3}{kg} \\
%   \cunum{10_2/3}{kg} 
% \end{cuexamplecode}
%
% \begin{function}{cutext-space}
%   \begin{syntax}
%     cutext-space = \meta{string}
%   \end{syntax}
%   \meta{string} is inserted between the numeral part and the unit part
%   when using \cCutext. By default it is set to \cs{space}. Use this option
%  if you want to e.g.\ insert an unbreakable space.
% \end{function}
% \begin{cuexamplecode}
%   \cutext{1}{kg}\\
%   \Cutext{10}{kg}\\ 
%   \cusetup { cutext-space = ~ } @
%   \cutext{1}{kg}\\
%   \Cutext{10}{kg}\\
%   \cusetup { cutext-space = {} } @
%   \cutext{1}{kg}\\
%   \Cutext{10}{kg}\\
%   \cusetup { cutext-space = {qwe} } @
%   \cutext{1}{kg}\\
%   \Cutext{10}{kg}\\
% \end{cuexamplecode}
%
%
% \begin{function}{phrase-space}
%   \begin{syntax}
%     phrase-space = \meta{string}
%   \end{syntax}
%   \meta{string} is inserted between the numeral part and the phrase part
%   while using \mycs{cuam}. By default it is set to \cs{space}. Use this option
%  if you want to e.g.\ insert an unbreakable space.
% \end{function}
%  \cusetup{use-phrases=true}
% \begin{myde}
% (Switching to german)
% \begin{cuexamplecode}
%   \cuam{12}\\
%   \cuam{144}\\
%   \cusetup { phrase-space = ~ } @
%   \cuam{12}\\
%   \cuam{144}\\
%   \cusetup { phrase-space = {} } @
%   \cuam{12}\\
%   \cuam{144}\\
%   \cusetup { phrase-space = {qwe} } @
%   \cuam{12}\\
%   \cuam{144}\\
% \end{cuexamplecode}
% \end{myde}
%  \cusetup{use-phrases=false}
%
% \begin{function}{amount-unit-space}
%   \begin{syntax}
%     amount-unit-space = \meta{string}
%   \end{syntax}
%   Change the spacing for \mycs{cunum} between the printed amount(s) and
%   the unit. The default value is |\thinspace|.
% \end{function}
%
% \begin{cuexamplecode}
%   \selectlanguage{ngerman} @
%   \cunum{1}{kg}\\
%   \cunum{1/2}{kg}\\
%   \cunum{1--2}{kg}\\
%   \cusetup { amount-unit-space = {\hspace{1em}} } @
%   \cunum{1}{kg}\\
%   \cunum{1/2}{kg}\\
%   \cunum{1--2}{kg}\\
%   \cusetup { amount-unit-space = {} } @
%   \cunum{1}{kg}\\
%   \cunum{1/2}{kg}\\
%   \cunum{1--2}{kg}\\
%   \cusetup { amount-unit-space = {qwe} } @
%   \cunum{1}{kg}\\
%   \cunum{1/2}{kg}\\
%   \cunum{1--2}{kg}\\
% \end{cuexamplecode}
%
% \subsubsection{label \& refs}
%
% \begin{function}{recalculate-amount}
%   \begin{syntax}
%     recalculate-amount = \meta{true/false}
%   \end{syntax}
%   Set this option to \opt{true} if you want to change your recipes to the given
%   number of people set by \opt{set-number-of-persons}. Note that only those
%   values who have a label are changed.
% \end{function}
%
% \begin{function}{set-number-of-persons}
%   \begin{syntax}
%     set-number-of-persons = \meta{integer}
%   \end{syntax}
%   With this option you can determine the number of people your recipes
%   are. Note that this option only has an effect on those who have a \meta{label}
%   given. It is set to \opt{4} by default.
% \end{function}
% \begin{cuexamplecode}
%   \culabel{anotherrecipe}{2} @
%   \curef{anotherrecipe}~persons \\ 
%   \cunum<anotherrecipe>{1}{kg}\\
%   \cuam<anotherrecipe>{1}\\
%   \cutext<anotherrecipe>{1}{kg}\\
%   \curef{anotherrecipe}~persons \\ @
%   \cusetup{ recalculate-amount = true } @
%   \curef{anotherrecipe}~persons \\ 
%   \cunum<anotherrecipe>{1}{kg}\\
%   \cuam<anotherrecipe>{1}\\
%   \cutext<anotherrecipe>{1}{kg}\\
%   \Cutext[ref=anotherrecipe]{10}{kg}\\ @
%   \cusetup { set-number-of-persons = 3 } @
%   \curef{anotherrecipe}~persons \\ 
%   \cunum<anotherrecipe>{1}{kg}\\
%   \cuam<anotherrecipe>{1}\\
%   \cutext<anotherrecipe>{1}{kg}\\
%   \Cutext[ref=anotherrecipe]{10}{kg}\\ @
%   \cusetup { set-number-of-persons = 2 } @
%   \curef{anotherrecipe}~persons \\ 
%   \cunum<anotherrecipe>{1}{kg}\\
%   \cuam<anotherrecipe>{1}\\
%   \cutext<anotherrecipe>{1}{kg}\\
%   \Cutext[ref=anotherrecipe]{10}{kg}\\ @
%   \cusetup { set-number-of-persons = 1 } @
%   \curef{anotherrecipe}~person \\ 
%   \cunum<anotherrecipe>{1}{kg}\\
%   \cuam<anotherrecipe>{1}\\
%   \cutext<anotherrecipe>{1}{kg}\\
%   \Cutext[ref=anotherrecipe]{10}{kg}\\ 
% \end{cuexamplecode}
% 
%
% \begin{function}{label}
%   \begin{syntax}
%     label = \meta{string}*\meta{integer}
%   \end{syntax}
%   The key-value version of \mycs{culabel}. It defines the label \meta{string}
%   which is originally for \meta{integer} people. Please note that the \texttt{*}
%   is mandatory as it separates the string from the integer. Note that each
%   label is defined globally and must be unique.
% \end{function}
% \begin{cuexamplecode}
%   \cusetup{label=Toast*1} @
%   \curef{Toast}~person \\
%   \cuam<Toast>{2} \\
%   \cunum<Toast>{2}{dag} \\
%   \cusetup { recalculate-amount = true } @
%   \curef{Toast}~persons \\
%   \cuam<Toast>{2} \\
%   \cunum<Toast>{2}{dag}
% \end{cuexamplecode}
%
%
% \begin{function}{get-label}
%   \begin{syntax}
%     get-label = \meta{label}
%   \end{syntax}
%   The key-value version of \mycs{curef}. Note that this key doesn't save the value
%   inside a macro but rather prints it directly into the document.
% \end{function}
% \begin{cuexamplecode}
%   \culabel{Schinken}{3} @
%   \cusetup{get-label=Schinken} \\
%   \curef{Schinken} \\
%   \cusetup { recalculate-amount = true } @
%   \cusetup{get-label=Schinken} \\
%   \curef{Schinken} \\
% \end{cuexamplecode}
%
%
% \begin{function}{ref}
%   \begin{syntax}
%     ref = \meta{label}
%   \end{syntax}
%   Instead of using the first optional arguments of the commands in \cref{sec:commands}
%   you may use this option. It requires a valid value and throws an error if \meta{label}
%   is not defined.
% \end{function}
% \begin{cuexamplecode}
%   \culabel{Kaese}{3} @
%   \cunum<Kaese>[m=dm]{1}{m} \\
%   \cunum[ref=Kaese,m=dm]{1}{m} \\
%   \cusetup { recalculate-amount = true } @
%   \cunum<Kaese>[m=dm]{1}{m} \\
%   \cunum[ref=Kaese,m=dm]{1}{m} 
% \end{cuexamplecode}
%
% \subsection{Weird options}
%
% \begin{function}{check-temperature}
%   \begin{syntax}
%     check-temperature = \meta{true/false}
%   \end{syntax}
%   Checks if the used temperature is below  absolute zero.
%  Currently \texttt{C}, \texttt{F}, \texttt{K} and \texttt{Re} are supported. 
%  While \lstinline|\cunum{0}{K}| is ok, \lstinline!\cunum{-1}{K}! raises an error, 
%  same for the others. Is set to \texttt{false} by default. To add new units
%  see \opt{add-temperature-to-check}.
% \end{function}
%
% \begin{function}{add-temperature-to-check}
%   \begin{syntax}
%     add-temperature-to-check = 
%     \ \ \{ 
%     \ \ \ \ \meta{unit-key-1} = \meta{minimum-value-1} ,  
%     \ \ \ \ \meta{unit-key-2} = \meta{minimum-value-2} ,  
%     \ \ \ \ \dots
%     \ \ \}
%   \end{syntax}
%   This option adds  \meta{unit-key-1} and so on to the list of units to be checked
%   if \opt{check-temperature} is active. The argument can be a comma-separated
%   list of \meta{unit-key} = \meta{minimum-value}. This sets the allowed minimum
%   value of \meta{unit-key} to \meta{minimum-value}. 
%  
% \end{function}
%
%   For example, this package implements the allowed minimum values for the
%  temperatures \texttt{C}, \texttt{F}, \texttt{K} and \texttt{Re} to be checked if
%  \opt{check-temperature} is active using:
% \begin{lstlisting}
%   \cusetup
%     {
%       add-temperature-to-check =
%         {
%           K = 0,
%           C = -273.15 ,
%           F = -459.67 ,
%           Re = -218.52
%         } 
%     }
% \end{lstlisting}
%
% If you want to add a new value, for example degree R{\o}mer
% (which has be defined in another example) you can write:
% \begin{lstlisting}
%   \cusetup
%     {
%       add-temperature-to-check = { Ro = -135.90375 }
%     }
% \end{lstlisting}
%
%
%
%
%
% \begin{function}{convert-to-eV}
%   \begin{syntax}
%     convert-to-eV = \meta{true/false}
%   \end{syntax}
%   Converts (nearly) every unit in \cref{tab:def-units}
%   to electron volt or the respective derivative. Note that this 
%   option is:
%   a) experimental and probably will forever be and b) just a joke, you are not 
%   supposed to use this units in a cookery book (and as you see this package
%   doesn't support the arrangement of such huge numbers). Also you may
%   want to check the values if you really want to use them, just to be sure
%   (I've checked them several times and hope they are finally correct, but mistakes
%   happen).
%
% \end{function}
%
% \begin{cuexamplecode}[\hfill]
%   \cusetup{ convert-to-eV=true } @
%   \cunum{1}{kg} \\
%   \cunum{1}{l} \\
%   \cunum{1}{J} \\
%   \cunum{1}{m} \\
%   \cunum{1}{C} \\
%   \cunum{1}{s}
% \end{cuexamplecode}
%
%
%
%
%
% 
% \section{Bugs \& Feedback}
% \label{sec:bug-feed}
%
% Bug reports  are always welcome. If you are sending a bug report 
% please include a minimal working example showing the bug and a short
% description. If you use mail please add  \pkg{cooking-units} 
% to the e-mail header. GMX has the habit of putting e-mails into the 
% spam account and adding  \pkg{cooking-units} to the header makes
% it easier to recognize those e-mails. 
%
% Feedback and requests (commands, units, etc.) are also  welcome.  Please
% also add (if possible) an example of the desired output into the
% minimal example (and -- if by mail -- add \pkg{cooking-units} to the
% header).
%
% Furthermore, as you can see I am not able to speak too many
% languages (german and english to be precise; I managed to add french
% with the help of the internet, which is not optimal) so if you are
% able to speak a language not yet implemented and would like to help
% you can send me  the translations known to you. A list of all units
% (and their current translations) is given in \cref{sec:Translations}.
% 
%
%
%
%
%
%
% \clearpage
% \appendix
%
% \section{Translations} 
% \label{sec:Translations}
% This section contains the list of available translations. Each table
% shows the available translations regarding the unit symbol, the unit name
% (printed if \mycs{cutext} or \mycs{Cutext} is used)  and
% the plural form (if different from the singular form). A second table
% shows the translations used for phrases (if given).
%
% If a translation is not available a \enquote{\translationdoesnotexisttl} is
% shown.
%
% \small
%
% \newpage
% \subsection{English}
% \begin{tabular}{ l l l l l }
%   \toprule
%   \CreateTableForLanguages{ English }
% \end{tabular}
%
%
% \newpage
% \subsection{american}
%^^A Only differences from \enquote{English} are defined.
%
% \begin{tabular}{ l l l l l }
%   \toprule
%   \CreateTableForLanguages{ american  }
% \end{tabular}
%
% \newpage
% \subsection{German}
% \begin{tabular}{ l l l l l }
%   \toprule
%   \CreateTableForLanguages{ German }
% \end{tabular}
%
% \bigskip
%
%   \CreateTableForPhrases { German }
%
% \bigskip
% \begin{minipage}{\textwidth}
% Some further phrases, just to write them down (they are not implemented, 
% as they are barely used).
%
% \begin{tabular} { l l @{\space} l l l l }
% \toprule
%    \meta{number}  & name &  \multicolumn{2}{c}{Note} & (plural) & gender \\ \midrule
%  60 & Schock & (5 Dutzend, & $12*5$)  & & n \\
%  144 & Gros & (12 Dutzend, & $12*12$)  && n \\
%  1728 & Gro{\ss}gros  &(12 Gro{\ss}, & $12*144$) && n \\
% \bottomrule
% \end{tabular}
% \end{minipage}
%
% Note that Gro{\ss}gros has other (probably more common) synonyms.
%
% \newpage
% \subsection{French}
% \begin{tabular}{ l l l l l }
%   \toprule
%   \CreateTableForLanguages{ French }
% \end{tabular}
%
%
% \clearpage
%
% \begin{comment}
%^^A \section{Notes and other stuff}
%^^A This section contains stuff which might or might not be interesting.
%^^A
%^^A \subsection{More or less FAQ}
%^^A
%^^A \num{624150912.6 +- 3.9 e+10}
%^^A \num{6.241509126 +- 0.000000039 e+18}
%^^A
%^^A \num{8.6173303 +- 00000.0000050 e-5}
%^^A
%^^A \subsection{Calculation}
%^^A This subsection deals with some regarding the conversion to certain units.
%^^A Mostly to (I like to call them) \enquote{electron volt units} (or natural units).
%^^A Let's begin.
%^^A
%^^A We start with the easiest one: conversion of joule into electron volt.
%^^A Easy because Wikipedia and nearly every physics book knows the answer. 
%^^A And according to Wikipedia
%^^A \begin{align}
%^^A  \SI{1}{\eV} &= \SI{1.6021766208(98)e-19}{\joule} \\
%^^A  \SI{1}{\joule} &= \SI{6.241509126 +- 0.000000039 e+18}{\eV}
%^^A \end{align}
%^^A Adding this is no problem: 
%^^A \begin{lstlisting}
%^^A   \cuaddtokeys {J} {eV} { 1.6021766208e-19 }
%^^A \end{lstlisting}
%^^A
%^^A
%^^A Great, next Temperature (kelvin) to electron volt. Also no problem at all
%^^A because joule and kelvin are related by the boltzmann constant $k_B$ with
%^^A \begin{equation}
%^^A   energy (\si{\joule}) = k_B \cdot temperature (\si{\kelvin})
%^^A \end{equation}
%^^A as $[k_B]= \sfrac{\si{\joule}}{\si{\kelvin}}$ and 
%^^A $k_B= \SI{1.38064852(79)e-23}{\joule\per\kelvin}$. Now using the power
%^^A of (more or less basic, but somehow annoying) mathematics we get:
%^^A \begin{align}
%^^A  [k_B] &= \SI{1.38064852(79)e-23}{\joule\per\kelvin} = \\
%^^A &= \num{1.38064852(79)e-23} \cdot \SI{6.241509126 +- 0.000000039 e+18 }{\eV\per\K} = \\
%^^A &= \SI{8.6173303 +- 0.0000050 e-5 }{\eV\per\K} 
%^^A \end{align}
%
%
%^^A \section{Notes and other stuff}
%^^A \subsection{Some details to electron volts}
%^^A This section explains how I calculated the values for conversion into electron volt
%^^A or the respective derivative. Let's begin with some constants (from Wikipedia), note that
%^^A \begin{equation*}
%^^A   \num{1.23456(78)e-9} = \num[separate-uncertainty = true]{1.23456(78)e-9}
%^^A \end{equation*}
%^^A just for info.
%^^A
%^^A \begin{align}
%^^A   \SI{1}{\eV} &= \SI{1.6021766208(98)e-19}{\J} \\
%^^A   \hbar &= \SI{1.054571800(13)e-34}{\J\s} \\
%^^A   k_B &= \SI{1.38064852(79)e-23}{\J\per\K} = \SI{8.6173303(50)e-5}{\eV\per\K}\\
%^^A   c &= \SI{299792458}{\m\per\s} 
%^^A \end{align}
%^^A
%^^A Note that all those and following values are rounded to the second relevant figure of the 
%^^A uncertainty of the value.
%^^A
%^^A Well then, next are temperatures.
%^^A
%^^A Temperatures are fairly easy. Just convert the temperature of your choice to kelvin
%^^A and multiplicate it with $k_B$:
%^^A \begin{align}
%^^A   \SI{1}{\K} * k_B &= \SI{8.6173303(50)e-5}{\eV}
%^^A \end{align}
%^^A
%^^A Next is more complicated. To compute the resulting unceartanity I using
%^^A \enquote{Gauß'sches Fehlerfortpflanzungsgesetz} (I don't know if this has 
%^^A a translation). Anyway.  We know that
%^^A \begin{align*}
%^^A    \frac{\SI{1}{\eV}}{c^2} &= \frac{\SI{1.6021766208e-19}{\J}}{(\SI{299792458}{\m\per\s})^2}
%^^A    = \frac{\SI{1.6021766208e-19}{\kg\m\squared\per\s\squared}}{\SI{8,98755178736818E16}{\m\squared\per\s\squared}} = \\
%^^A &\frac{\SI{1.6021766208e-3}{\kg}}{\num{8,98755178736818}} =
%^^A \SI{1782661906,94091624922088}{\kg}
%^^A \end{align*}
%^^A
%^^A Great, now we have to calucalte the unceartinity:
%^^A \begin{align}
%^^A   f(\si{\eV},c) &= \frac{\si{\eV}}{c^2} \\
%^^A   df(\si\eV,c) &= \sqrt{ (\frac{\partial f}{\partial \si\eV})^2 * (\Delta \si\eV)^2 +  (\frac{\partial f}{\partial c})^2  * (\Delta c)^2 } = \\
%^^A &= \sqrt{  (\frac{1}{c^2})^2 * (\Delta \si\eV)^2 + (-2\frac{eV}{c^{-3}})^2  * (\Delta c)^2 }  
%^^A \end{align}
%^^A
%^^A Now we know that $\Delta c = 0$:
%^^A \begin{align}
%^^A   df(\si\eV,c) &= \sqrt{  (\frac{1}{c^2})^2 * (\Delta \si\eV)^2 }  = 
%^^A \sqrt{ \frac{1}{(\SI{299792458}{\m\per\s} )^2} * (\SI{0.0000000098e-19}{\J})^2 } \\
%^^A &= 
%^^A \end{align}
% \end{comment}
%
%
%
%
%
%
%
%
%
%
% \end{documentation}
%  
% \begin{implementation}
%
% \section{Implementation}
%
% \subsection{Beginning}
%
%    \begin{macrocode}
%<@@=cooking_units>
%    \end{macrocode}
%
%    \begin{macrocode}
%<*package> 
%    \end{macrocode}
%
%
%
% Na dann, auf gehts!
%
%    \begin{macrocode}
\@ifpackageloaded {xparse} 
  { } 
  { \RequirePackage {xparse} }
\@ifpackageloaded {expl3} 
  { } 
  { \RequirePackage {expl3} }
%    \end{macrocode}
%
%: Package
%    \begin{macrocode}
\ProvidesExplPackage
  {cooking-units}
  {2017/10/23}
  {1.20}
  {Ein Paket fuer Kocheinheiten}
%    \end{macrocode}
%
% Checking if \pkg{expl3} is up-to-date, otherwise abort the loading of the package.
%    \begin{macrocode}
\@ifpackagelater { expl3 } { 2017/09/18 }
  { }
  {
    \PackageError { cooking-units } { Support~package~expl3~too~old }
      {
        You~need~to~update~your~installation~of~the~bundles~'l3kernel'~and~
        'l3packages'.\MessageBreak
        Loading~cooking-units~will~abort!
      }
    \tex_endinput:D
  }
%    \end{macrocode}
% Loading some needed packages.
%    \begin{macrocode}
\@ifpackageloaded { translations } { } { \RequirePackage { translations } }
\@ifpackageloaded { xfrac } { } { \RequirePackage { xfrac } }
\@ifpackageloaded { l3keys2e } { } { \RequirePackage { l3keys2e } }
%    \end{macrocode}
%
% Checking if \pkg{translations} is up-to-date, otherwise abort the loading of the package.
%    \begin{macrocode}
\@ifpackagelater { translations } { 2017/08/31 }
  { }
  {
    \PackageError { cooking-units } { Support~package~translations~too~old }
      {
        You~need~to~update~your~installation~of~the~package~'translations'.\MessageBreak
        Loading~cooking-units~will~abort!
      }
    \tex_endinput:D
  }
%    \end{macrocode}
%
% Define the only load-time option for this package. If it is set,
% load package \pkg{fmtcount}.
%    \begin{macrocode}
\bool_new:N \g_@@_opt_numeral_bool
\keys_define:nn { cooking-units }
  {
    use-numerals .bool_gset:N = \g_@@_opt_numeral_bool ,
    use-numerals .default:n = { false },
  }
%    \end{macrocode}
% Now process the package options \dots
%    \begin{macrocode}
\ProcessKeysOptions { cooking-units }
\bool_if:NT \g_@@_opt_numeral_bool
  {
    \@ifpackageloaded { fmtcount } { } { \RequirePackage { fmtcount } }
  }
%    \end{macrocode}
% \dots\ and redefine the package option such that it cannot be used elsewhere.
%    \begin{macrocode}
\keys_define:nn { cooking-units }
  {
    use-numerals .code:n = { \msg_error:nnn { cooking-units } { load-time-option } { fmtcount } } ,
  }
%    \end{macrocode}
% 
%
%
% \subsection {Defining Variables}
%
% \begin{macro}
%  {
%  \tl_replace_all:NVn,
%  \tl_replace_once:NnV ,  \tl_replace_once:NVn ,
%  \tl_if_in:nVTF ,
%  \tl_if_in:NVTF ,
%  \tl_if_in:NVT ,
%  \fp_compare:cNnT ,
%  \fp_eval:c ,
%  \prop_get:cVc ,
%  \int_abs:c ,
%  }
% Some variations of commands we will need later.
%    \begin{macrocode}
\cs_generate_variant:Nn \tl_replace_all:Nnn { NVn }
\cs_generate_variant:Nn \tl_replace_once:Nnn { NnV, NVn }
\cs_generate_variant:Nn \tl_if_in:nnTF { nVTF }
\cs_generate_variant:Nn \tl_if_in:NnTF { NVTF  }
\cs_generate_variant:Nn \tl_if_in:NnT { NVT }
\cs_generate_variant:Nn \fp_compare:nNnT { cNnT }
\cs_generate_variant:Nn \fp_eval:n { c }
\cs_generate_variant:Nn \prop_get:cVN { cVc }
\cs_generate_variant:Nn \int_abs:n { c }
\cs_generate_variant:Nn \tl_show:n { x , f }
%    \end{macrocode}
% \end{macro}
%
% \begin{macro}{\@@_frac:nn}
% This command is used to print the fractions and can be changed
% accordingly.
%    \begin{macrocode}
\cs_new_eq:NN \@@_frac:nn \sfrac
%    \end{macrocode}
% \end{macro}
%
%
% \begin{macro}{\@@_print_numeral:n , \@@_print_Numeral:n}
% This command is used to print the fractions and can be changed
% accordingly.
%    \begin{macrocode}
\cs_new:Npn \@@_print_numeral:n #1 {}
\cs_new:Npn \@@_print_Numeral:n #1 {}
\bool_if:NT \g_@@_opt_numeral_bool
  {
    \cs_set_eq:NN \@@_print_numeral:n \numberstringnum
    \cs_set_eq:NN \@@_print_Numeral:n \Numberstringnum
  }
%    \end{macrocode}
% \end{macro}
%
%
% \begin{macro}{ \l_@@_change_unit_prop }
% Conversions of units are stored within this property list.
% If someone requests that \texttt{kg} should be changed into \texttt{g},
% \texttt{kg} is stored as a key with the value \texttt{g}. If someone
% then uses the unit \texttt{kg} the value \texttt{g} is restored and  the unit
% is changed accordingly.
%    \begin{macrocode}
\prop_new:N \l_@@_change_unit_prop
%    \end{macrocode}
% \end{macro}
%
%
% \begin{macro}
%  {
% \l_@@_number_tmpa_tl,
% \l_@@_number_tmpb_tl,
% \l_@@_tmpa_tl,
% \l_@@_tmpb_tl,
% \l_@@_mixed_fraction_tl,
% \l_@@_given_unit_tl,
% \l_@@_option_unit_tl,
% \l_@@_language_tl,
% \l_@@_cunum_range_sign_tl,
% \l_@@_cutext_range_sign_tl,
% \l_@@_value_unit_space_tl,
% \l_@@_input_digits_tl ,
% \l_@@_input_decimal_mark_tl,
% \l_@@_input_value_signs_tl,
% \l_@@_input_allowed_special_signs_tl ,
% \c_@@_input_str_hash_one_tl ,
% \l_@@_input_range_sign_tl,
% \l_@@_input_times_persons_sign,
% \l_@@_cutext_space_tl,
% \l_@@_cuphrase_space_tl,
% \l_@@_translation_tmpa_tl,
% \l_@@_cutext_last_value_tl,
% \l_@@_phrase_phrase_tl ,
%  }
% Quite a lot of tl's.
%    \begin{macrocode}
\tl_new:N \l_@@_number_tmpa_tl
\tl_new:N \l_@@_number_tmpb_tl
\tl_new:N \l_@@_tmpa_tl
\tl_new:N \l_@@_tmpb_tl
\tl_new:N \l_@@_mixed_fraction_tl
\tl_new:N \l_@@_given_unit_tl
\tl_new:N \l_@@_option_unit_tl
\tl_new:N \l_@@_language_tl
\tl_new:N \l_@@_cunum_range_sign_tl
\tl_new:N \l_@@_cutext_range_sign_tl
\tl_new:N \l_@@_value_unit_space_tl
\tl_new:N \l_@@_input_digits_tl 
\tl_new:N \l_@@_input_decimal_mark_tl
\tl_new:N \l_@@_input_value_signs_tl
\tl_new:N \l_@@_input_allowed_special_signs_tl 
\tl_new:N \c_@@_input_str_hash_one_tl 
\tl_new:N \l_@@_input_range_sign_tl
\tl_new:N \l_@@_input_times_persons_sign
\tl_new:N \l_@@_cutext_space_tl
\tl_new:N \l_@@_cuphrase_space_tl
\tl_new:N \l_@@_translation_tmpa_tl
\tl_new:N \l_@@_cutext_last_value_tl
\tl_new:N \l_@@_phantom_tl
\tl_new:N \l_@@_phrase_phrase_tl
\tl_new:N \l_@@_unit_key_not_allowed_tl
%    \end{macrocode}
% \end{macro}
%
%
% Setting some token lists to their default value.
% \lstinline|str_hash_one_tl| is used for defining single keys.
% (You will see, I didn't have a better idea)
%    \begin{macrocode}
\tl_set:Nn \l_@@_input_digits_tl { 0123456789 }
\tl_set:Nn \l_@@_input_times_persons_sign { * }
\tl_set:Nn \l_@@_input_range_sign_tl { -- }
\tl_set:Nn \l_@@_input_decimal_mark_tl { . , }
\tl_set:Nn \l_@@_input_value_signs_tl { + - }
\tl_set:Nn \l_@@_input_allowed_special_signs_tl { ? }
\tl_set_rescan:Nnn \c_@@_input_str_hash_one_tl
  { \char_set_catcode_letter:N \# } {#1}
%    \end{macrocode}
%    \begin{macrocode}
\tl_set:Nn \l_@@_cunum_range_sign_tl { -- }
\tl_set:Nn \l_@@_cutext_range_sign_tl { -- }
\tl_set:Nn \l_@@_value_unit_space_tl { \thinspace }
\tl_set:Nn \l_@@_cutext_space_tl { \space }
\tl_set:Nn \l_@@_cuphrase_space_tl { \space }
\tl_set:Nn \l_@@_unit_key_not_allowed_tl { , / }
%    \end{macrocode}
%
% Flat out stolen from \pkg{siunitx}
%    \begin{macrocode}
\AtBeginDocument {
  \cs_if_free:cT { T@TS1 }
    {
      \DeclareFontEncoding { TS1 } { } { }
      \DeclareFontSubstitution { TS1 } { cmr } { m } { n }
    }
}
%    \end{macrocode}
%    \begin{macrocode}
\DeclareTextSymbolDefault \c_@@_minus_tl { TS1 }
\DeclareTextSymbol \c_@@_minus_tl { TS1 } { 61 }
\AtBeginDocument {
  \@ifpackageloaded { fontspec }
    {
      \@ifpackageloaded { eulervm }
        { }
        {
          \int_const:Nn \c_@@_minus_int { 8722 }
          \tl_set:Nn \c_@@_minus_tl
            { \tex_char:D \c_@@_minus_int }
        }
    }
    { }
}
%    \end{macrocode}
%
% \begin{macro}{ \l_@@_mixed_frac_dim }
%   The dimension between the fraction and the mixed fraction part is
%   stored within this macro. There is no real reason why I have
%   chosen this distance to be 0.1em, I just thought that it looks
%   best.  But if someone has some ideas of how large this distance
%   should be I am happy to listen.
%    \begin{macrocode}
\dim_new:N \l_@@_mixed_frac_dim
\dim_set:Nn \l_@@_mixed_frac_dim { 0.1 em }
%    \end{macrocode}
% \end{macro}
%
%
% \begin{macro}{ \l_@@_significant_figures_int }
%   Stores the round-precision inside. Not sure if 'significant
%   figures' is the correct therm for this. .
%    \begin{macrocode}
\int_new:N \l_@@_significant_figures_int
\int_set:Nn \l_@@_significant_figures_int { 2 }
%    \end{macrocode}
% \end{macro}
%
% \begin{macro}{ \l_@@_print_numerals_below_int }
%   Used if option 'use-numerals' is active. Uses numerals for
%   integers smaller than this number. I learned this number
%   at school.
%    \begin{macrocode}
\int_new:N \l_@@_print_numerals_below_int
\int_set:Nn \l_@@_print_numerals_below_int { 13 }
%    \end{macrocode}
% \end{macro}
%
%
% \begin{macro}{ \l_@@_local_number_of_persons_int }
% Each recipe defined by \cs{culabel} defines a counter to
% store the number of persons the recipe is for. For calculation
% the value is retrived and stored inside this temporal counter.
%    \begin{macrocode}
\int_new:N \l_@@_local_number_of_persons_int
%    \end{macrocode}
% \end{macro}
%
% \begin{macro}{ \l_@@_wanted_number_of_persons_int }
%   Not only the number of persons are recipe is for is needed
%   for calculation, but also the number of persons you want
% the recipe to be. This information is stored here.
%    \begin{macrocode}
\int_new:N \l_@@_wanted_number_of_persons_int
\int_set:Nn \l_@@_wanted_number_of_persons_int { 4 }
%    \end{macrocode}
% \end{macro}
%
%    \begin{macrocode}
\int_new:N \l_@@_phrase_number_tl
%    \end{macrocode}
%
%
% \begin{macro}{ \l_@@_list_of_defined_keys_clist , \g_@@_list_of_defined_units_clist }
%   Sequence of defined units and keys. Units are defined
%  globally as they create new commands, keys do not do that
% (I think). Could be my mistake.
%    \begin{macrocode}
\seq_new:N \l_@@_list_of_defined_keys_seq
\seq_new:N \g_@@_list_of_defined_units_seq
%    \end{macrocode}
%\end{macro}
%
%
% \begin{macro} { \g_@@_allowed_unit_phrases_tl }
%   \changes {2016/06/12} {New: \cs{g_@@_allowed_unit_phrases_tl}}
% Used for the keys 'one(m)', 'one(f)', etc. Those are special keys
% which cannot be used as units, but are processed by the commands in
% the language section as such.
%    \begin{macrocode}
\clist_new:N \g_@@_allowed_special_keys_clist
%    \end{macrocode}
% \end{macro}
%
% \begin{macro} { \l_@@_phrase_numbers_clist }
% Inside this list the numbers for which a phrase is defined
% is stored in. As this is language specific, the list is stored
% inside a language-sensitive command and retrived when needed.
%    \begin{macrocode}
\clist_new:N \l_@@_phrase_numbers_clist 
%    \end{macrocode}
% \end{macro}
%
% \begin{macro} { \l_@@_temperatures_to_check_seq }
%   \changes {1.02h} {New.}
% Stores units which should be tested if \opt{check-temperature} equals true.
%    \begin{macrocode}
\seq_new:N \l_@@_temperatures_to_check_seq
%    \end{macrocode}
% \end{macro}
%
% \begin{macro} { \l_@@_phrase_prop }
% Stores the number and he respective phrase. For example
% if \enquote{12} has the phrase \enquote{Dutzend}, this
% key-value pair is stored inside.
%    \begin{macrocode}
\prop_new:N \l_@@_phrase_prop
%    \end{macrocode}
% \end{macro}
%
% \begin{macro}
%   {
% \l_@@_minus_bool,
% \l_@@_round_number_bool,
% \l_@@_error_bool,
% \l_@@_eval_fractions_bool,
% \l_@@_draft_bool,
% \l_@@_round_to_int_bool,
% \l_@@_special_sign_bool,
% \l_@@_single_key_bool,
% \l_@@_check_temperature_bool,
% \l_@@_convert_to_eV_bool,
% \l_@@_cutext_uppercase_word_bool,
% \l_@@_error_for_unknown_value_bool,
% \l_@@_using_cutext_bool,
% \l_@@_cuam_old_bool,
% \l_@@_calc_for_persons_bool,
% \l_@@_calc_because_ref_was_given_bool,
% \l_@@_calc_persons_bool,
% \l_@@_cutext_to_cunum_bool,
% \l_@@_cutext_old_bool,
% \l_@@_cutext_change_unit_bool,
% \l_@@_round_commercial_bool,
% \l_@@_use_phrases_bool,
% \l_@@_lokal_phrase_bool
%   }
% Some booleans we need later.
%    \begin{macrocode}
\bool_new:N \l_@@_minus_bool
\bool_new:N \l_@@_round_number_bool
\bool_new:N \l_@@_error_bool
\bool_new:N \l_@@_eval_fractions_bool
\bool_new:N \l_@@_draft_bool
\bool_new:N \l_@@_round_to_int_bool
\bool_new:N \l_@@_special_sign_bool
\bool_new:N \l_@@_single_key_bool
\bool_new:N \l_@@_check_temperature_bool
\bool_new:N \l_@@_convert_to_eV_bool
\bool_new:N \l_@@_cutext_uppercase_word_bool
\bool_new:N \l_@@_error_for_unknown_value_bool
\bool_new:N \l_@@_using_cutext_bool
\bool_new:N \l_@@_cuam_old_bool
\bool_new:N \l_@@_calc_for_persons_bool
\bool_new:N \l_@@_calc_because_ref_was_given_bool
\bool_new:N \l_@@_calc_persons_bool
\bool_new:N \l_@@_cutext_to_cunum_bool
\bool_new:N \l_@@_cutext_old_bool
\bool_new:N \l_@@_cutext_change_unit_bool
\bool_new:N \l_@@_round_commercial_bool
\bool_new:N \l_@@_use_phrases_bool
\bool_new:N \l_@@_lokal_phrase_bool
\bool_new:N \l_@@_print_numeral_bool
\bool_new:N \l_@@_lokal_numeral_bool
\bool_new:N \l_@@_significant_figures_positive_bool
%    \end{macrocode}
%
% While rewriting the code I searched for those 
% booleans a lot.
%    \begin{macrocode}
\bool_new:N \l_@@_range_in_input_bool
\bool_new:N \l_@@_fraction_in_input_bool
\bool_new:N \l_@@_decimal_in_input_bool
%    \end{macrocode}
% \end{macro}
%
%
%
% Setting some of them to \texttt{true} or another boolean respectively.
%    \begin{macrocode}
\bool_set_true:N \l_@@_cutext_change_unit_bool
\bool_set_eq:NN \l_@@_print_numeral_bool \g_@@_opt_numeral_bool
%    \end{macrocode}
%
%
% \begin{macro}{ \q_@@_range }
% Replacing the sign \enquote{-{}-} with \lstinline|\q_@@_range| for testing.
%    \begin{macrocode}
\quark_new:N \q_@@_range
%    \end{macrocode}
%\end{macro}
%
% \begin{macro}{ \q_@@_no_translation }
% Note the spelling mistake in \enquote{available}.
%    \begin{macrocode}
\quark_new:N \q_@@_no_translation
%    \end{macrocode}
%\end{macro}
%
%
% \begin{macro}{
%    \l_@@_tmpa_fp ,
%    \l_@@_tmpa_clist ,
%    \l_@@_tmpa_prop ,
%    \l_@@_tmpb_prop ,
%    \l_@@_tmpa_seq ,
%  }
% Some temporal stores which are used thoughout the code.
%    \begin{macrocode}
\fp_new:N \l_@@_tmpa_fp
\clist_new:N \l_@@_tmpa_clist
\prop_new:N \l_@@_tmpa_prop
\prop_new:N \l_@@_tmpb_prop
\seq_new:N \l_@@_tmpa_seq
\int_new:N \l_@@_tmpa_int
\int_new:N \l_@@_tmpb_int
%    \end{macrocode}
%\end{macro}
%
%
%
% \subsection {Keys}
%
% Let's define some keys.
%    \begin{macrocode}
\keys_define:nn { cooking-units }
  {
%    \end{macrocode}
%
%\begin{macro}{ eval-fraction }
% If set to true the fractions are evaluated.
%    \begin{macrocode}
    eval-fraction .bool_set:N = \l_@@_eval_fractions_bool ,
    eval-fraction .default:n = { false } ,
%    \end{macrocode}
%\end{macro}
%
%\begin{macro}{ round-precision }
%  Setting the round-precision. Setting those two at once to 
%  not calculate it every time.
%    \begin{macrocode}
    round-precision .code:n = 
      {
        \int_set:Nn \l_@@_significant_figures_int {#1}
        \int_compare:nNnT {#1} < { 0 }
          { \bool_set_false:N \l_@@_significant_figures_positive_bool }
          { \bool_set_true:N \l_@@_significant_figures_positive_bool }
      } ,
    round-precision .default:n = { 2 } ,
%    \end{macrocode}
%\end{macro}
%
%\begin{macro}{ round-to-int }
%  Rounding the results to an integer.
%    \begin{macrocode}
    round-to-int .bool_set:N = \l_@@_round_to_int_bool ,
    round-to-int .default:n = { false } ,
%    \end{macrocode}
%\end{macro}
%
%\begin{macro}{ range-sign }
%  Setting the printed range sign and make a difference between cunum
%  and c(C)utext.
%    \begin{macrocode}
    range-sign .meta:n = 
      {  
        cunum-range-sign = {#1} ,
        cutext-range-sign = {#1}
      } ,
    range-sign .default:n = { -- } ,
    cunum-range-sign .tl_set:N = \l_@@_cunum_range_sign_tl ,
    cunum-range-sign .default:n = { -- } ,
    cutext-range-sign .tl_set:N = \l_@@_cutext_range_sign_tl ,
    cutext-range-sign .default:n = { -- } ,
%    \end{macrocode}
%\end{macro}
%
%\begin{macro}{ amount-unit-space }
%  Setting the space between the value and the printed unit.
%    \begin{macrocode}
    amount-unit-space .tl_set:N = \l_@@_value_unit_space_tl ,
    amount-unit-space .default:n = { \thinspace } ,
%    \end{macrocode}
%\end{macro}
%
%\begin{macro}{ fraction-command }
%  Setting the fraction command
%    \begin{macrocode}
    fraction-command .code:n = { \cs_set_eq:NN \@@_frac:nn #1 } ,
    fraction-command .default:n = { \sfrac } ,
%    \end{macrocode}
%\end{macro}
%
%\begin{macro}{ fraction-inline }
% Setting the code inline.
%    \begin{macrocode}
    fraction-inline .code:n = { \cs_set:Npn \@@_frac:nn ##1##2 {#1} } ,
    fraction-inline .default:n = { \sfrac {#1} {#2} } ,
%    \end{macrocode}
%\end{macro}
%
%\begin{macro}{ mixed-fraction-space }
% Setting the space between the mixed fraction part and the fraction.
%    \begin{macrocode}
    mixed-fraction-space .dim_set:N = \l_@@_mixed_frac_dim ,
    mixed-fraction-space .default:n = { 0.1 em } ,
%    \end{macrocode}
%\end{macro}
%
%\begin{macro}{ parse-number }
% Parse the input? If no the input is printed as is (after some safety parsing).
%    \begin{macrocode}
    parse-number .bool_set_inverse:N = \l_@@_draft_bool , 
    parse-number .default:n= { true } ,
%    \end{macrocode}
%\end{macro}
%
%\begin{macro}{ add-special-sign }
% Adding a (some) special sign(s) which is (are)  allowed in the input.
%    \begin{macrocode}
    add-special-sign .code:n = 
      { 
        \str_if_eq:nnTF {#1} { < }
          { \msg_error:nn { cooking-units } { <-not-allowed-as-special-sign } }
          { \tl_put_right:Nn  \l_@@_input_allowed_special_signs_tl {#1} }
      } ,
    add-special-sign .default:n = { } ,
%    \end{macrocode}
%\end{macro}
%
%\begin{macro}{ set-special-sign }
% Doing the same as above but also overrides the old signs.
%    \begin{macrocode}
    set-special-sign .code:n = 
      { 
        \str_if_eq:nnTF {#1} { < }
          { \msg_error:nn { cooking-units } { <-not-allowed-as-special-sign } }
          { \tl_set:Nn  \l_@@_input_allowed_special_signs_tl {#1} }
      } ,
%    \end{macrocode}
%\end{macro}
%
%
% \begin{macro}{ input-range-sign }
% Don't wanna use \verb|--| as a range sperator in \mycs{cunum}? 
% Use this option.
%    \begin{macrocode}
    input-range-sign .tl_set:N = \l_@@_input_range_sign_tl ,
    input-range-sign .default:n = { -- } ,
%    \end{macrocode}
%\end{macro}
%
%
%\begin{macro}{ check-temperature }
%  Weird option. Checking the temperature, if the temperature is below the absolute zero temperature
%  it raises an error.
%    \begin{macrocode}
    check-temperature .bool_set:N = \l_@@_check_temperature_bool ,
    check-temperature .default:n = { true } ,
%    \end{macrocode}
%
%\begin{macro}{ add-temperature-to-check }
%  Adds a temperature to check for \texttt{check-temperature}.
% It uses the \cs{keyval\_parse:NNn} command as this
% macro is used to parse keys (which is what I need).
%    \begin{macrocode}
    add-temperature-to-check .code:n = 
      { 
        \keyval_parse:NNn
          \@@_temperature_to_check_print_error:n
          \@@_temperatures_to_check_define:nn
          {#1}
      },
    temperature-to-check .value_required:n = { true } ,
%    \end{macrocode}
% \end{macro} 
%
%\begin{macro}{ convert-to-eV }
%  Another weird option, converts pretty much any unit defined by this
%  package to electron volt or the respective derivative.
% As this is a unit transfrmation, it needs to be inside the
% group.
%    \begin{macrocode}
    convert-to-eV .bool_set:N = \l_@@_convert_to_eV_bool ,
    convert-to-eV .default:n = { true } ,
    convert-to-eV .groups:n = { change-unit } ,
%    \end{macrocode}
%\end{macro}
%
% \begin{macro}{ use-numerals-below }
%   Use numerals if the intger is below the integer set by this option.
%    \begin{macrocode}
    use-numerals-below .int_set:N = \l_@@_print_numerals_below_int ,
    use-numerals-below .default:n = { 13 } ,
%    \end{macrocode}
%\end{macro}
%
%
% \begin{macro}{ set-unknown-message }
%   Sets the message for a special-sign to error, warning or none.
%    \begin{macrocode}
    set-unknown-message .choices:nn =
      { error , warning , none }
      {
        \msg_redirect_name:nnn { cooking-units } { amount-not-known } 
          { \l_keys_choice_tl }
      } ,
    set-unknown-message .default:n = { warning } ,
%    \end{macrocode}
%\end{macro}
%
% \begin{macro}{ set-unknown-message }
%   Sets the message for a special-sign to error, warning or none.
%    \begin{macrocode}
    set-cutext-translation-message .choices:nn =
      { error , warning , none }
      {
        \msg_redirect_name:nnn { cooking-units } { cutext-no-translation-available } 
          { \l_keys_choice_tl }
      } ,
    set-cutext-translation-message .default:n = { warning } ,
%    \end{macrocode}
%\end{macro}
%
%
%\begin{macro}{ erase-all-options }
%  Erasing all preset options.
%    \begin{macrocode}
    erase-all-options .code:n = 
      { 
        \seq_map_inline:Nn \g_@@_list_of_defined_units_seq
          {
            \clist_clear:c { l_@@_predefined_option_##1_clist }
          }
        },
%    \end{macrocode}
% \end{macro}
%
%\begin{macro}{ erase-options-for }
%  Erasing options.
%    \begin{macrocode}
    erase-options-for .code:n = 
      { 
        \keyval_parse:NNn 
          \@@_erase_options_for:n
          \@@_erase_options_for_error:nn
          {#1}
      },
%    \end{macrocode}
% \end{macro}
%
%\begin{macro}{ set-options-for }
%  Setting options.
%    \begin{macrocode}
    set-options-for .code:n = 
      { 
        \keyval_parse:NNn 
          \@@_set_options_for:n
          \@@_set_options_for:nn
          {#1}
      },
%    \end{macrocode}
% \end{macro}
%
%\begin{macro}{ add-options-for }
%  Setting options.
%    \begin{macrocode}
    add-options-for .code:n = 
      { 
        \keyval_parse:NNn 
          \@@_add_options_for:n
          \@@_add_options_for:nn
          {#1}
      },
%    \end{macrocode}
% \end{macro}
%
% \begin{macro}{ round-half }
%  Choosing between \enquote{normal} rounding to even and commercial rounding and sets the boolean accordingly.
%    \begin{macrocode}
    round-half .choices:nn =
      { default , commercial }
      {
        \int_case:nn { \l_keys_choice_int }
          {
            { 1 } { \bool_set_false:N \l_@@_round_commercial_bool }
            { 2 } { \bool_set_true:N \l_@@_round_commercial_bool }
          }
      },
    round-half .default:n = { default },
%    \end{macrocode}
% \end{macro}
%
% \begin{macro}{ set-number-of-persons }
% Setting the number of persons the recipe should be for.
%    \begin{macrocode}
    set-number-of-persons .int_set:N = \l_@@_wanted_number_of_persons_int ,
    set-number-of-persons .default:n = { 4 } ,
%    \end{macrocode}
% \end{macro}
%
% \begin{macro}{ label }
% Defines a label \dots\ is \mycs{culabel} as a key.
%    \begin{macrocode}
    label .code:n = 
      { 
        \@@_label_and_persons:n {#1} 
      } ,
    label .value_required:n = { true } ,
%    \end{macrocode}
% \end{macro}
%
% \begin{macro}{ get-label }
% \mycs{curef} as a key.
%    \begin{macrocode}
    get-label .code:n = 
      { 
        \@@_curef:n {#1}
      } ,
    label .value_required:n = { true } ,
%    \end{macrocode}
% \end{macro}
%
% \begin{macro}{ ref }
% The <> option for keys.
%    \begin{macrocode}
    ref .code:n = 
      { 
        \@@_reference_label_and_persons:n {#1} 
      } ,
    ref .value_required:n = { true } ,
%    \end{macrocode}
% \end{macro}
%
% \begin{macro}{ cuam-version  , cutext-version}
% Some keys with horrible option names. Reverts 
% the respective command back to its older state (pre v1.10).
%    \begin{macrocode}
    cuam-version .choices:nn = 
      { new , old }
      {
        \int_case:nn { \l_keys_choice_int }
          {
            { 1 } { \bool_set_false:N \l_@@_cuam_old_bool }
            { 2 } { \bool_set_true:N \l_@@_cuam_old_bool }
          }
      } ,
    cuam-version .default:n = { new } ,
    cutext-version .choices:nn = 
      { new , old }
      {
        \int_case:nn { \l_keys_choice_int }
          {
            { 1 } { \bool_set_false:N \l_@@_cutext_old_bool } 
            { 2 } { \bool_set_true:N \l_@@_cutext_old_bool }
          }
      } ,
    cutext-version .default:n = { new } ,
%    \end{macrocode}
% \end{macro}
%
% \begin{macro}{ recalculate-amount }
% Setting the number of persons your recipes should be for
% is not enough; it is also needed to tell the package to
% recalculate the amounts.
%    \begin{macrocode}
    recalculate-amount .bool_set:N = \l_@@_calc_for_persons_bool ,
    recalculate-amount .default:n = { false } ,
%    \end{macrocode}
% \end{macro}
%
% \begin{macro}{ cutext-to-cunum }
% Don't want any \mycs{cutext} (or \mycs{Cutext}) in your
% document? Use this option!
%    \begin{macrocode}
    cutext-to-cunum .bool_set:N = \l_@@_cutext_to_cunum_bool ,
    cutext-to-cunum .default:n = { false } ,
    cutext-to-cunum .groups:n = { command } ,
%    \end{macrocode}
% \end{macro}
%
% \begin{macro}{ cutext-space }
% The space used in \mycs{cutext} between the number (or numeral)
% and unit.
%    \begin{macrocode}
    cutext-space .tl_set:N = \l_@@_cutext_space_tl ,
    cutext-space .default:n = { \space } ,
%    \end{macrocode}
% \end{macro}
%
% \begin{macro}{ phrase-space }
% Same as before, but for phrases
%    \begin{macrocode}
    phrase-space .tl_set:N = \l_@@_cuphrase_space_tl ,
    phrase-space .default:n = { \space } ,
%    \end{macrocode}
% \end{macro}
%
% \begin{macro}{ cutext-change-unit }
% Do not wanna change units in \mycs{cutext}? Use this option.
%    \begin{macrocode}
    cutext-change-unit .bool_set:N = \l_@@_cutext_change_unit_bool ,
    cutext-change-unit .default:n = { true } ,
%    \end{macrocode}
% \end{macro}
%
%
% \begin{macro}{ use-phrases }
% Do not wanna use phrases in \mycs{cuam}? Use this option!
%    \begin{macrocode}
    use-phrases .bool_set:N = \l_@@_use_phrases_bool ,
    use-phrases .default:n = { true } ,
%    \end{macrocode}
% \end{macro}
%
%
%
% \begin{macro}{ print-numerals }
% A not very good name, but I couldn't think of a better name.
%    \begin{macrocode}
    print-numerals .bool_set:N = \l_@@_print_numeral_bool ,
    print-numerals .default:n = { true } ,
%    \end{macrocode}
% \end{macro}
%
%
% \begin{macro}{ numeral-function , Numeral-function }
%
%    \begin{macrocode}
    numeral-function .code:n = { \cs_set_eq:NN \@@_print_numeral:n #1 } ,
    Numeral-function .code:n = { \cs_set_eq:NN \@@_print_Numeral:n #1 } ,
%    \end{macrocode}
% \end{macro}
%
%
%
% Ending the definition of keys.
%    \begin{macrocode}
  }
%    \end{macrocode}
% \end{macro}
%  
%
%
% \subsection {Messages}
%
% Defining messages. 
%
% \begin{macro} {Messages}
% I do not allow fractions and ranges  in the same input. Maybe I will
% change this.
%    \begin{macrocode}
\msg_new:nnnn { cooking-units } { fraction-not-allowed-with-range }
  { 
    'You' \ cannot \ use \ '/' \ ( and  \ '_' )\ in \ combination \
    with \ '\l_@@_input_range_sign_tl' \ in \  '#1'.
  }{ 
    You \ cannot \ use \ fractions \ with \ a \ range.
    \msg_see_documentation_text:n { cooking-units } 
  }
%    \end{macrocode}
%
% Do not allow a \_ without a /.
%    \begin{macrocode}
\msg_new:nnnn { cooking-units } { missing-slash }
  { 
    You \ cannot \ use \ '_'  \ without \
    '/' \ in \  '#1'.
  }{ 
    You \ cannot \ have \ a \ mixed \ fraction \ ('_') \ without \
    a \ normal \ fraction \ ('/').
    \msg_see_documentation_text:n { cooking-units } 
  }
%    \end{macrocode}
% Error message if unit is not known to this package.
%    \begin{macrocode}
\msg_new:nnnn { cooking-units } { unknown-unit }
  { 
    The \ unit \ '#1' \ is \ not \ defined. \ Use \ 
    \newcookingunit \ to \ define \ new \ units.
  }{ 
    Define \ units \ before \ using \ or \ check \ if \ the \
    unit-key \ is \ written \ correctly.
    \msg_see_documentation_text:n { cooking-units } 
  }
%    \end{macrocode}
% Error if unit is already defined.
%    \begin{macrocode}
\msg_new:nnnn { cooking-units } { unit-already-defined } 
  { The\ unit \ '#1' \ is \ already \ defined. } 
  { 
    The \ unit-key \ is \ already \ defined. \ Please \ use \ a \ different \
    key \ for \ a \ new \ unit.
    \msg_see_documentation_text:n {cooking-units} 
  }
%    \end{macrocode}
% 
% Missing argument in \lstinline|\cudefinesymbols| (et all).
%    \begin{macrocode}
\msg_new:nnnn { cooking-units } { missing-argument } 
  { There \ is \ an \ missing \ argument. } 
  { 
    You \ probably \ have \ forgotten \ a \ curly-brace \ pair.
    \msg_see_documentation_text:n {cooking-units} 
  }
%    \end{macrocode}
% If fractions are evaluated division by zero is not allowed.
%    \begin{macrocode}
\msg_new:nnnn { cooking-units } { Division-by-zero } 
  { Division\ by \ zero \ is \ not \ allowed. } 
  { See \ a \ math \ book \ of \ your \ choice \ or \ for \ example \ Wikipedia. }
%    \end{macrocode}
% Showing the not allowed token in the input. Hope this helps.
%    \begin{macrocode}
\msg_new:nnnn { cooking-units } { Token-not-allowed } 
  { The \ token \ '#1' \ is \ not \ allowed. } 
  { 
    The \ command \ accepts \ only \ a \ fixed \ number \ of \ tokens. 
    \msg_see_documentation_text:n {cooking-units} 
  }
%    \end{macrocode}
% A second decimal sign is not allowed (No na net).
%    \begin{macrocode}
\msg_new:nnnn { cooking-units } { Second-decimal-sign-not-allowed } 
  { A \ second \ decimal \ sign \ is \ not \ allowed. } 
  { 
    Perhaps \ you \ didn't \ type \ it \ correctly.
    \msg_see_documentation_text:n {cooking-units} 
  }
%    \end{macrocode}
% Error message for an undefined key.
%    \begin{macrocode}
\msg_new:nnnn { cooking-units } { Key-not-defined } 
  { 
    The \ key \ '#1' \ is \ not \ defined. \ Use \ \cudefinekeys or \
    \cudefinesinglekey to \ define \ keys. 
  } 
  { 
    This \ key \ is \ not \ defined, \ perhaps \ you  \ misspelled \ it.
    \msg_see_documentation_text:n {cooking-units} 
  }
%    \end{macrocode}
%
% If the temperature is too low print this error message.
% Now prints all units for which a zero-point is defined.
%    \begin{macrocode}
\msg_new:nnnn { cooking-units } { Temperature-too-low } 
  { 
    The \ temperature \ '#1' \ is \ too \ low.\ It \ cannot \ be \
    below \ the \ absolute \ zero - point \ of \ '#2'. \ Note \ that \ 
    the \ temperatures \ 
    '\seq_use:Nnnn \l_@@_temperatures_to_check_seq 
      { ', ~ ' } { ', ~ ' } { ' ~ and ~ ' }' \
    are \ rounded \ to \ integers. 
    \\\\
    You \ can \ disable \ the \ option \ 'check-temperature' \ to \
    disable \ this \ error.
  } 
  { See \ for \ example \ Wikipedia. }
%    \end{macrocode}
% If for an unit-key the value is wrong the following error message is shown.
%    \begin{macrocode}
\msg_new:nnnn { cooking-units } { key-choice-unknown } 
  { 
    The \ key \ '#1' \ only \ accepts \ only \ 
    '#3' \ as \ a \ set \ of \ choices \ and \ '#2' \ is \ non \ of \ these.
  } 
  { 
    The \ key \ accepts \ accepts\ only \ a \ fixed \ set \ of \ choices. \
    You \ can \ add \ new \ choices \ via \ \cuaddkeys, \ \cuaddsinglekeys \
    and \ \cuaddtokeys.
    \msg_see_documentation_text:n {cooking-units} 
  }
%    \end{macrocode}
%
%\changes {v0.98a} {2016/06/11} {New option: 'value-not-known'.}
%\changes {v0.99} {2016/06/12} {Renamed option: 'amount-not-known'.}
% An info message for unknown messages.
%    \begin{macrocode}
\msg_new:nnnn { cooking-units } { amount-not-known } 
  { 
    The \ amount \ of \ #1 \ is \ not \ known \ at \ line \ 
    \msg_line_number: .
  } 
  {
    You \ used\  a\  special\  sign\  indicating\  that\  the\  true\  amount\  of\  the\ 
    specific\  ingredient\  is\  (was) \ not\  known\  to\  you. This\  message\ 
    reminds\  you\  about\  that\  fact.
    \msg_see_documentation_text:n {cooking-units}
  }
%    \end{macrocode}
%
% If a load time option is not used as a package option,  this
% message is shown.
%    \begin{macrocode}
\msg_new:nnnn { cooking-units } { load-time-option } 
  { 
    The \ option \ '#1' \ is \ only \ available \ as \ a \ load-time-option. \
    Please \ set \ it \ as \ a \ package \ option.
  } 
  { 
    You \ cannot \ set \ this \ option \ using \ \cusetup \
    but \ only \ as \ a \ package \ option.
    \msg_see_documentation_text:n { cooking-units } 
  }
%    \end{macrocode}
%
% Messages for obsolete commands.
%    \begin{macrocode}
\msg_new:nnnn { cooking-units } { obsolete-command } 
  { 
    Command \ #1 is \ obsolete. \ Please \ use \ #2 instead.
  } 
  { 
    Don't \ use \ this \ old \ command \ ... 
    \msg_see_documentation_text:n { cooking-units } 
  }
%    \end{macrocode}
%
% Error message if a new temperature to check is defined and no
% minimal value is given.
%    \begin{macrocode}
\msg_new:nnnn { cooking-units } { No-Value-given } 
  { 
    Please \ input \ a \ number \ to \ check \ for \ 
    'check-temperature'.
  } 
  { 
    A \ minimum \ value \ is \ needed \ for \ testing \ if \
    'check-temperature' \ is \ active.
    \msg_see_documentation_text:n { cooking-units } 
  }
%    \end{macrocode}
%
% Error message if a zero-point temperature is already defined.
%    \begin{macrocode}
\msg_new:nnnn { cooking-units } { Minimum-already-defined } 
  { 
    A \ minimum \ for \ '#1' \ has \ already \ been \  defined.
  } 
  { 
    You \ cannot \ redefine \ it.
    \msg_see_documentation_text:n { cooking-units } 
  }
%    \end{macrocode}
% 
% Using the key version of \mycs{culabel} one needs to give the number of
% people the recipe is for after a \enquote{*}.
%    \begin{macrocode}
\msg_new:nnnn { cooking-units } { Number-of-persons-missing } 
  { 
    Please \ add \ the \ number \ of \ persons \ this \ recipe \ is \ for \ in \ '#1'. \
    Note \ that\ the \ number \ must \ be \ given \ after \ a \ '*'.
  } 
  { 
    Write \ 'Schweinsbraten*4' \ to \ create \ the \ label \ 'Schweinsbraten' \
    which \ is \ initially \ for \ 4 \ persons.
    \msg_see_documentation_text:n { cooking-units } 
  }
%    \end{macrocode}
%
% The number of persons, must be an integer \dots\ 
%    \begin{macrocode}
\msg_new:nnnn { cooking-units } { Number-of-persons-is-not-an-integer } 
  { 
    The \ number \ of \ persons \ the \ recipe \ is \ for \ must \ be \ an \
    integer. \ '#1' \ is \ not \ allowed.
  } 
  { 
    The \ number \ '#1' \ is \ not \ allowed.
    \msg_see_documentation_text:n { cooking-units } 
  }
%    \end{macrocode}
%
% Each label defined by \mycs{culabel} needs to be new.
%    \begin{macrocode}
\msg_new:nnnn { cooking-units } { label-already-defined } 
  { 
    The \ label \ '#1' \ has \ already \ been \ defined.
  } 
  { 
    Each \ label \ must \ be \ unique.
    \msg_see_documentation_text:n { cooking-units } 
  }
%    \end{macrocode}
%
% Message if a label is not defined.
%    \begin{macrocode}
\msg_new:nnnn { cooking-units } { label-not-defined } 
  { 
    The \ label \ is \ not \ defined. \ Please \ note \ that \ a \ label
    \ has \ to \ defined \ before \ it \ is \ referenced.
  } 
  { 
    Define \ the \ label \ before \ using \ it.
    \msg_see_documentation_text:n { cooking-units } 
  }
%    \end{macrocode}
%
% If an unit is already defined and redefined by an \mycs{declarecookingunit}.
%    \begin{macrocode}
\msg_new:nnnn { cooking-units } { redefine-unit } 
  { 
    The \ unit \ '#1' \ is \ redefined \ by \ \declarecookingunit at \ 
    \msg_line_context: . 
  } 
  { 
    \msg_see_documentation_text:n { cooking-units } 
  }
%    \end{macrocode}
%
% A \enquote{phrase} must be an integer.
%    \begin{macrocode}
\msg_new:nnnn { cooking-units } { phrase-unit-not-an-integer } 
  { 
    A \ phrase \ must \ be \ an \ integer,  \ '#1' \ is \ not \ allowed.
  } 
  { 
    You \ can \ only \ use \ integers.
    \msg_see_documentation_text:n { cooking-units } 
  }
%    \end{macrocode}
%
% A translation for \cs{cutext} or \cs{Cutext} is not available. For this case
% -- instead of printing the keyname (see  \pkg{translations}) -- we fall back
% to \cs{cunum}.
%    \begin{macrocode}
\msg_new:nnnn { cooking-units } { cutext-no-translation-available } %% ToDo
  { 
    For \ the \ unit \ '#1' \ there \ exists \ no \ translation \ to \ be \ used \
    for \ \cutext and \ \Cutext. \
    You \ can \ define \ new \ translations \ for \ a \ given \ language \ using \
    \cudefinename.
  } 
  { 
    \cunum \ is \ used \ instead.
    \msg_see_documentation_text:n { cooking-units } 
  }
%    \end{macrocode}
%
%
%    \begin{macrocode}
\msg_new:nnnn { cooking-units } { Translation-not-available } %% ToDo
  { 
    The \ translation \ for \ #1 \ does \ not \ exist. 
    Please \ define \ it \ using \ \cudefinename.
  } 
  { 
    And \ you \ may \ send \ me \ the \ translation \ as \ it \ is \
    not \ available \ yet.
    \msg_see_documentation_text:n { cooking-units } 
  }
%    \end{macrocode}
%
%
%    \begin{macrocode}
\msg_new:nnnn { cooking-units } { <-not-allowed-as-special-sign } 
  { 
    Currently \ (and \ probably \ forever) \ the \ sign \ '<' \
    is \ not \ allowed \ to \ be \ used \ as \ a \ special \ sign.
  } 
  { 
    I \ apologize \ for \ the \ inconvenience.
    \msg_see_documentation_text:n { cooking-units } 
  }
%    \end{macrocode}
%
%
%    \begin{macrocode}
\msg_new:nnnn { cooking-units } { unknown-gender } 
  { 
    '#1' \ is \ not \ allowed \ to \ be \ used \ as \ a \ gender-specification.
    Only \ 'm', \ 'f' \ or \ 'n' \ are \ allowed
  }{ 
    Please \ remove \ spaces \ if \ there \ are \ some.
    \msg_see_documentation_text:n { cooking-units } 
  }
%    \end{macrocode}
% \end{macro}
%
% \section {Helper Macros}
%
% The name says it all.
%
%\begin{macro}
%  {
%    \@@_error_if_unit_not_defined:n,
%    \@@_error_if_unit_not_defined:V,
%  }
% A little helper macro. Checks if the unit is defined, if not raise an error.
%    \begin{macrocode}
\cs_new:Npn \@@_error_if_unit_not_defined:n #1
  {
    \seq_if_in:NnF \g_@@_list_of_defined_units_seq {#1} 
      { \msg_error:nnn { cooking-units } { unknown-unit } {#1} }
  }
\cs_generate_variant:Nn \@@_error_if_unit_not_defined:n { V }
%    \end{macrocode}
%\end{macro}
%
%
%
%
%
%
% \begin{macro}{ \@@_if_integer:n }
%   Checking if the input consists only of numbers. 
%    \begin{macrocode}
\prg_new_protected_conditional:Npnn \@@_if_integer:n #1 { TF , F , T }
  {
    \tl_map_inline:nn {#1} 
      {
        \tl_if_in:NnF \l_@@_input_digits_tl {##1}
          { 
            \tl_map_break:n { \use_iii:nnn }
          }
      }
    \use_i:nn \prg_return_true: \prg_return_false:
  }
\cs_generate_variant:Nn \@@_if_integer:nTF { V }
\cs_generate_variant:Nn \@@_if_integer:nT { V }
\cs_generate_variant:Nn \@@_if_integer:nF { V }
%    \end{macrocode}
% \end{macro}
%
% \begin{macro}{ \@@_if_integer:n }
%   Checking if the input consists only of numbers. 
%    \begin{macrocode}
\cs_new:Npn \@@_bool_set_if_integers:NNN #1#2#3
  {
    \@@_if_integer:VTF #2
      { \bool_set_true:N #1 }
      { \bool_set_false:N #1 }
    \bool_lazy_and:nnT 
      { \l_@@_range_in_input_bool } {#1}
      { 
        \@@_if_integer:VF #3
          { \bool_set_false:N #1 }
      } 
  }
%    \end{macrocode}
% \end{macro}
%
%
%
% \begin{macro}[TF]{ \@@_fp_if_equal_one:n}
% \begin{macro}[TF]{ \@@_int_if_equal_one:n}f
%    \begin{macrocode}
\prg_new_conditional:Npnn \@@_fp_if_equal_one:n #1 { TF }
  {
    \fp_compare:nNnTF {#1} = { \c_one_fp }
      { \prg_return_true: }
      { \prg_return_false: }
  }
\prg_new_conditional:Npnn \@@_int_if_equal_one:n #1 { TF }
  {
    \int_compare:nNnTF {#1} = { \c_one }
      { \prg_return_true: }
      { \prg_return_false: }
  }
%    \end{macrocode}
% \end{macro} \end{macro}
%
%
%
% \begin{macro}{ \@@_check_if_correct_gender_input:n }
%    \begin{macrocode}
\cs_new:Npn \@@_check_if_correct_gender_input:n #1
  {
    \str_case:nnF {#1}
      {
        { m } {}
        { f } {}
        { n } {}
      } { \msg_error:nnn { cooking-units } { unknown-gender } {#1} } 
  }
%    \end{macrocode}
% \end{macro}
%
%
%
% \begin{macro}{\@@_convert_to_eV:}
%   Still work in progress. Will probably forever be. Changes all
%   predefined units into natural units (except for those who cannot
%   be changed, like Msp. for example). 
%
%    \begin{macrocode}
\seq_new:N \g_@@_natural_units_seq
\prop_new:N \g_@@_convert_to_eV_remember_prop
\seq_set_split:Nnn \g_@@_natural_units_seq { , }
  {
    eV ,
    eVc-2 ,
    hbareV-1 ,
    chbareV-1 ,
    (chbareV-1)3 ,
  }
\cs_new:Npn \@@_keys_set:nnn #1#2#3
  {
    \keys_set:nn {#1} { #2 = #3 }
  }
\cs_generate_variant:Nn \@@_keys_set:nnn { nVV , nVn }
\cs_new:Npn \@@_convert_to_eV:
  {
    \prop_if_exist:cT { l_@@_cukeys_ \l_@@_given_unit_tl _prop }
      {
        \prop_get:NVNTF \g_@@_convert_to_eV_remember_prop 
          \l_@@_given_unit_tl \l_@@_tmpa_tl
          {
            \@@_keys_set:nVV { cooking-units } \l_@@_given_unit_tl \l_@@_tmpa_tl 
          }{
            \prop_get:cnNT { l_@@_cukeys_ \l_@@_given_unit_tl _prop } { Liste } \l_@@_tmpa_seq
              {
                \seq_map_inline:Nn \g_@@_natural_units_seq
                  {
                    \seq_if_in:NnT \l_@@_tmpa_seq {##1}
                      {
                        \prop_gput:NVn \g_@@_convert_to_eV_remember_prop
                          \l_@@_given_unit_tl {##1}
                        \@@_keys_set:nVn { cooking-units } \l_@@_given_unit_tl {##1}
                        \seq_map_break:
                      }
                  }
              }
          }
      }
  }
%    \end{macrocode}
%\end{macro}
%
%
%
%
% \begin{macro}{\@@_tl_set_fp_and_eval:Nn}
% I am using this construction often enough, so \dots\ yeah.
%    \begin{macrocode}
\cs_new:Npn \@@_tl_set_fp_and_eval:Nn #1#2
  { \tl_set:Nx #1 { \fp_eval:n {#2} } }
\cs_generate_variant:Nn \@@_tl_set_fp_and_eval:Nn { Nc }
%    \end{macrocode}
% \end{macro}
%
% \begin{macro}{\@@_temperature_to_check_print_error:n,\@@_temperatures_to_check_define:nn}
% Commands used in the option \texttt{temperature-to-check}.
% If no value is given (first command) it raises an error. If a value
% is given the  unit is added to a check-list and a new 
% constant is defined.
%    \begin{macrocode}
\cs_new:Npn \@@_temperature_to_check_print_error:n #1
  {
    \@@_error_if_unit_not_defined:n {#1}
    \msg_error:nn { cooking-units } { No-Value-given }
  }
\cs_new:Npn \@@_temperatures_to_check_define:nn #1#2
  {
    \@@_error_if_unit_not_defined:n {#1}
    \fp_if_exist:cTF { c_@@_ #1 _min_fp }
      { \msg_warning:nnn { cooking-units } { Minimum-already-defined } {#1} }
      {
        \seq_put_right:Nn \l_@@_temperatures_to_check_seq {#1}
        \fp_const:cn { c_@@_ #1 _min_fp } {#2}      
      }
  }
%    \end{macrocode}
% \end{macro}
%
% \begin{macro}{ \@@_erase_options_for:n , \@@_erase_options_for_error:nn }
%    \begin{macrocode}
\cs_new:Npn \@@_erase_options_for:n #1
  {
    \@@_error_if_unit_not_defined:n {#1}
    \clist_clear:c { l_@@_predefined_option_##1_clist }
  }
\cs_new:Npn \@@_erase_options_for_error:nn #1#2
  {
    \@@_error_if_unit_not_defined:n {#1}
    \msg_error:nn { cooking-units } { Value-not-allowed }
  }
%    \end{macrocode}
% \end{macro}
%
% \begin{macro}{ \@@_set_options_for:n , \@@_set_options_for:nn }
%    \begin{macrocode}
\cs_new:Npn \@@_set_options_for:n #1
  {
    \@@_error_if_unit_not_defined:n {#1}
    \msg_error:nn { cooking-units } { No-Value-given }
  }
\cs_new:Npn \@@_set_options_for:nn #1#2
  {
    \@@_error_if_unit_not_defined:n {#1}
    \clist_set:cn { l_@@_predefined_option_#1_clist } {#2}
  }
%    \end{macrocode}
% \end{macro}
%
% \begin{macro}{ \@@_add_options_for:n , \@@_add_options_for:nn }
%    \begin{macrocode}
\cs_new:Npn \@@_add_options_for:n #1
  {
    \@@_error_if_unit_not_defined:n {#1}
    \msg_error:nn { cooking-units } { No-Value-given }
  }
\cs_new:Npn \@@_add_options_for:nn #1#2
  {
    \@@_error_if_unit_not_defined:n {#1}
    \clist_put_right:cn { l_@@_predefined_option_#1_clist } {#2}
  }
%    \end{macrocode}
% \end{macro}
%
% \begin{macro}{\@@_rounding_function:n}
% I think this definition is somewhat stupid, but I don't have a better idea.
%    \begin{macrocode}
\cs_new:Npn \@@_rounding_function:n #1 
  {
    #1
      \bool_if:NTF \l_@@_round_commercial_bool
        { 
          ,
          \bool_if:NTF \l_@@_round_to_int_bool 
            { \c_zero_fp } 
            { \l_@@_significant_figures_int } 
          , #1 
        }
        { \bool_if:NF \l_@@_round_to_int_bool { , \l_@@_significant_figures_int } }
  }
%    \end{macrocode}
% \end{macro}
%
% \begin{macro}{\@@_label_and_persons:n}
%    \begin{macrocode}
\cs_new:Npn \@@_label_and_persons:n #1 
  { 
    \tl_if_in:nnTF {#1} { * }
      {
        \@@_label_and_persons_times_persons:ww #1 \q_stop
      }{
        \msg_error:nnn { cooking-units} { Number-of-persons-missing } {#1}
      }
  }
%    \end{macrocode}
% \end{macro}
%
%    \begin{macrocode}
\cs_new:Npn \@@_label_and_persons_times_persons:ww #1 * #2 \q_stop
  { 
    \@@_culabel:nn {#1} {#2}
  }
%    \end{macrocode}
%
% \begin{macro}{ \@@_culabel:nn , \@@_curef:n }
%    \begin{macrocode}
\cs_new:Npn \@@_culabel:nn #1#2
  { 
    \int_if_exist:cTF { l_@@_number_of_persons_ #1 _int }
      {
        \msg_error:nnn { cooking-units } { label-already-defined } {#1}
      }{
        \@@_if_integer:nTF {#2}
          {
            \int_new:c { l_@@_number_of_persons_ #1 _int }
            \int_gset:cn { l_@@_number_of_persons_ #1 _int } {#2}
          }
          { \msg_error:nnn { cooking-units } { Number-of-persons-is-not-an-integer } {#2} }      
      }
  }
%    \end{macrocode}
% \end{macro}
%
% \begin{macro}{ \@@_reference_label_and_persons:n }
%    \begin{macrocode}
\cs_new:Npn \@@_reference_label_and_persons:n #1
  { 
    \int_if_exist:cTF { l_@@_number_of_persons_ #1 _int }
      { 
        \bool_set_true:N \l_@@_calc_because_ref_was_given_bool
        \int_set_eq:Nc \l_@@_local_number_of_persons_int
          { l_@@_number_of_persons_ #1 _int }
      }
      { \msg_error:nnn { cooking-units } { label-not-defined } {#1} }
  }
%    \end{macrocode}
% \end{macro}
%
% \begin{macro}{ \@@_curef:n }
%    \begin{macrocode}
\cs_new:Npn \@@_curef:n #1
  { 
    \int_if_exist:cTF { l_@@_number_of_persons_ #1 _int }
      {
        \bool_if:NTF \l_@@_calc_for_persons_bool
          { \int_use:N \l_@@_wanted_number_of_persons_int }
          { \int_use:c { l_@@_number_of_persons_ #1 _int } }
      }{ \msg_error:nnn { cooking-units } { label-not-defined } {#1} }
  }
%    \end{macrocode}
% \end{macro}
%
% \begin{macro}{ \culabel , \curef  }
%    \begin{macrocode}
\NewDocumentCommand \culabel { m m } { \@@_culabel:nn {#1} {#2} }
\NewExpandableDocumentCommand \curef { m } { \@@_curef:n {#1} }
%    \end{macrocode}
% \end{macro}
%
%
%
%
% \begin{macro}{ \@@_tl_if_in_remove_and_set_bool:NnN }
% Can probably optimize this as the sign is always at the first place
% (this is done after cheking the input, so the asumption is safe), but yeah \dots
%    \begin{macrocode}
\cs_new:Npn \@@_tl_if_in_remove_and_set_bool:NnN #1#2#3
  {
    \tl_if_in:NnTF #1 {#2}
      {
        \bool_set_true:N #3
        \tl_remove_once:Nn #1 {#2}
      }
      { \bool_set_false:N #3 }
  }
%    \end{macrocode}
% \end{macro}
%
% \begin{macro}{ \@@_tl_if_in_remove_and_reverse_bool:NnN }
%    \begin{macrocode}
\cs_new:Npn \@@_tl_if_in_remove_and_reverse_bool:NnN #1#2#3
  {
    \tl_if_in:NnT #1 {#2}
      {
        \bool_if:NTF #3
          { \bool_set_false:N #3 }
          { \bool_set_true:N #3 }
        \tl_remove_once:Nn #1 {#2}
      }
  }
%    \end{macrocode}
% \end{macro}
%
%
%
% \subsection{Language Macros}
%
%    \begin{macrocode}
\tl_const:Nn \c_@@_postfix_unit_tl { ( cu-unit ) }
\tl_const:Nn \c_@@_postfix_unit_pl_tl { ( cu-unit-pl ) }
\tl_const:Nn \c_@@_postfix_unitname_tl { ( cu-unitname ) }
\tl_const:Nn \c_@@_postfix_unitname_pl_tl { ( cu-unitname-pl ) }
\tl_const:Nn \c_@@_postfix_gender_tl { ( cu-unitgender ) }
\tl_const:Nn \c_@@_postfix_phrase_tl { ( cu-unitphrase ) }
%    \end{macrocode}
%
% \begin{macro}{ \@@_deftranslation:nn }
% \begin{macro}{ \@@_deftranslation:xxn , \@@_deftranslation:Vnv}
%    \begin{macrocode}
\cs_new:Npn \@@_deftranslation_base:nnn #1#2#3
  {
    \declaretranslationfallback { #1 #2 } {#3}
  }
\cs_new:Npn \@@_deftranslation_to:nnnn #1#2#3#4
  {
    \declaretranslation {#1} { #2 #3 } {#4}
  }
\cs_generate_variant:Nn \@@_deftranslation_base:nnn { xx , xxV }
\cs_generate_variant:Nn \@@_deftranslation_to:nnnn { Vxxv, VxxV , Vxxn, Vxxx,Vxxo,Vxxf }
%    \end{macrocode}
% \end{macro}
% \end{macro}
%
%
%
% \begin{macro}{\@@_newtranslation:xxn }
% \begin{macro}{\@@_newtranslation:nxxn }
%    \begin{macrocode}
\cs_new:Npn \@@_newtranslation_base:nnn #1#2#3
  {
    \definetranslationfallback { #1 #2 } {#3}
  }
\cs_new:Npn \@@_newtranslation_to:nnnn #1#2#3#4
  {
    \definetranslation {#1} { #2 #3 } {#4}
  }
\cs_generate_variant:Nn \@@_newtranslation_base:nnn { nVn }
\cs_generate_variant:Nn \@@_newtranslation_to:nnnn { nnVn }
%    \end{macrocode}
% \end{macro}
% \end{macro}
%
%
% \begin{macro}{\@@_translate:nn , \@@_translate:xx }
% \begin{macro}{\@@_translate_let:Nxx , \@@_translate_let:Nxx }
%    \begin{macrocode}
\cs_new:Npn \@@_translate:nn #1#2
  {
    \GetTranslation { #1 #2 }
  }
\cs_new:Npn \@@_translate_let:Nnn #1#2#3
  {
    \SaveTranslation {#1} { #2 #3 }
  }
\cs_new:Npn \@@_translate_let:nNnn #1#2#3#4
  {
    \SaveTranslationFor {#2} {#1} { #3 #4 }
  }
\cs_generate_variant:Nn \@@_translate:nn { xx  }
\cs_generate_variant:Nn \@@_translate_let:Nnn { Nxx }
\cs_generate_variant:Nn \@@_translate_let:nNnn { VNxx, nNxx }
%    \end{macrocode}
% \end{macro}
% \end{macro}
%
% \begin{macro}[TF]{\@@_unitname_get:Nn,\@@_unitname_get:Nx }
%    \begin{macrocode}
\prg_new_conditional:Npnn \@@_unitname_get:Nn #1#2 { F }
  {
    \@@_translate_let:Nxx #1 {#2} \c_@@_postfix_unitname_tl
    \tl_if_eq:NNTF #1 \q_@@_no_translation
      { \prg_return_false: }
      { \prg_return_true: }
  }
\cs_generate_variant:Nn \@@_unitname_get:NnF { Nx }
%    \end{macrocode}
% \end{macro}
%
% \begin{macro}{\@@_translate_one_to_and_check_existance:Nx }
%    \begin{macrocode}
\cs_new:Npn \@@_translate_one_to_and_check_existance:Nx #1#2
  {
    \@@_translate_let:Nxx #1 {#2} \c_@@_postfix_gender_tl
    \@@_unitname_get:NxF #1 { one (#1) }
      { 
        \@@_translate_let:Nxx #1 {#2} \c_@@_postfix_gender_tl
        \msg_error:nnn { cooking-units } { Translation-not-available } { one (#1) }
      }
  }
%    \end{macrocode}
% \end{macro}
%
% \begin{macro}{ decimal-mark, one(m), one(f), one(n) }
%   Defining the translation for the decimal-mark. Note that those
%   'phrases' are stored inside
%   \verb|\g_@@_allowed_special_keys_clist|.
%   Furthermore some translations are defined.
%    \begin{macrocode}
\clist_gset:Nn \g_@@_allowed_special_keys_clist
  {
    decimal-mark ,
    one (m) ,
    one (f) ,
    one (n)
  }
%    \end{macrocode}
% \end{macro}
%    \begin{macrocode}
\@@_newtranslation_base:nVn { decimal-mark } \c_@@_postfix_unitname_tl { . }
\@@_newtranslation_base:nVn { decimal-mark } \c_@@_postfix_unitname_pl_tl { \q_@@_no_translation }
\@@_newtranslation_to:nnVn { German } { decimal-mark } \c_@@_postfix_unitname_tl { , }
%    \end{macrocode}
% Note that the plural versions just exist for completing the set.
%    \begin{macrocode}
\@@_newtranslation_base:nVn { one (m) } \c_@@_postfix_unitname_tl { \q_@@_no_translation }
\@@_newtranslation_base:nVn { one (f) } \c_@@_postfix_unitname_tl { \q_@@_no_translation }
\@@_newtranslation_base:nVn { one (n) } \c_@@_postfix_unitname_tl { \q_@@_no_translation }
\@@_newtranslation_base:nVn { one (m) } \c_@@_postfix_unitname_pl_tl { \q_@@_no_translation }
\@@_newtranslation_base:nVn { one (f) } \c_@@_postfix_unitname_pl_tl { \q_@@_no_translation }
\@@_newtranslation_base:nVn { one (n) } \c_@@_postfix_unitname_pl_tl { \q_@@_no_translation }
%    \end{macrocode}
%    \begin{macrocode}
\@@_newtranslation_to:nnVn { English } { one (m) } \c_@@_postfix_unitname_tl { one }
\@@_newtranslation_to:nnVn { English } { one (f) } \c_@@_postfix_unitname_tl { one }
\@@_newtranslation_to:nnVn { English } { one (n) } \c_@@_postfix_unitname_tl { one }
%    \end{macrocode}
%    \begin{macrocode}
\@@_newtranslation_to:nnVn { German } { one (m) } \c_@@_postfix_unitname_tl { ein }
\@@_newtranslation_to:nnVn { German } { one (f) } \c_@@_postfix_unitname_tl { eine }
\@@_newtranslation_to:nnVn { German } { one (n) } \c_@@_postfix_unitname_tl { ein }
%    \end{macrocode}
%
%
% \section{General}
%
% This section collects commands which are used by all four main commands
% (\mycs{cunum}, \cCutext, \mycs{cuam}), hence the name.
% 
%
% \begin{macro}{\@@_initialise_default:n , 
%     \@@_initialise_unit_change:n , 
%     \@@_initialise_after_unit_change:nn 
%   }
%   Functions shared by all initialization functions:
%    \begin{macrocode}
\cs_new_protected:Npn \@@_initialise_default:n #1
  {
    \tl_set:Nn \l_@@_given_unit_tl {#1}
    \@@_error_if_unit_not_defined:V \l_@@_given_unit_tl
    \bool_set_false:N \l_@@_special_sign_bool
    \bool_set_false:N \l_@@_error_bool
    \bool_set_false:N \l_@@_range_in_input_bool
    \bool_set_false:N \l_@@_fraction_in_input_bool
  }
%    \end{macrocode}
% Some units have options added to them. To get those options it is first
% needed to know which unit will be used at all. Therefore units will
% be changed first and afterwards the other options are processed.
%    \begin{macrocode}
\cs_new_protected:Npn \@@_initialise_unit_change:n #1
  {
    \tl_if_empty:nF {#1}
      {
        \keys_set_groups:nnn { cooking-units } { change-unit } {#1}  
      }
    \bool_if:NT \l_@@_convert_to_eV_bool { \@@_convert_to_eV: }
    \prop_get:NVNF \l_@@_change_unit_prop \l_@@_given_unit_tl \l_@@_option_unit_tl
      {
        \tl_set_eq:NN \l_@@_option_unit_tl \l_@@_given_unit_tl
      }
  }
%    \end{macrocode}
%    \begin{macrocode}
\cs_new_protected:Npn \@@_initialise_after_unit_change:nn #1#2
  {
    \IfNoValueF {#1}
      { \@@_reference_label_and_persons:n {#1} }
    \clist_if_empty:cF { l_@@_predefined_option_ \l_@@_option_unit_tl _clist }
      {
        \keys_set_filter:nnv 
          { cooking-units } 
          { change-unit } 
          { l_@@_predefined_option_ \l_@@_option_unit_tl _clist }
      }
    \tl_if_empty:nF {#2}
      { \keys_set_filter:nnn { cooking-units } { change-unit } {#2} }
    \bool_lazy_and:nnTF
      { \l_@@_calc_because_ref_was_given_bool } { \l_@@_calc_for_persons_bool }
      { \bool_set_true:N \l_@@_calc_persons_bool }
      { \bool_set_false:N \l_@@_calc_persons_bool }
    \bool_lazy_and:nnTF
      { \g_@@_opt_numeral_bool } { \l_@@_print_numeral_bool }
      { \bool_set_true:N \l_@@_lokal_numeral_bool }
      { \bool_set_false:N \l_@@_lokal_numeral_bool }
  }
%    \end{macrocode}
% \end{macro}
%
%
% \begin{macro}{ \@@_print_input:N , \@@_print_correct_unit: , 
%   \@@_post_process_input:NN , \@@_pre_process_input:NN}
%   Wrapper macro for printing the (not)calculated output. Note that if no
%   calculation happens in \mycs{cutext} (and \mycs{Cutext}) \cs{l_@@_option_unit_tl}
%   is set to \cs{l_@@_given_unit_tl} (the unit given in the second argument of 
%   \mycs{cutext} or \mycs{Cutext}) by default.
%    \begin{macrocode}
\cs_new_protected:Npn \@@_print_input:N #1 { }
\cs_new_protected:Npn \@@_print_correct_unit: { }
\cs_new_protected:Npn \@@_do_not_process_input:NN #1#2 { }
\cs_new_eq:NN \@@_pre_process_input:NN \@@_do_not_process_input:NN
\cs_new_eq:NN \@@_post_process_input:NN \@@_do_not_process_input:NN
%    \end{macrocode}
% \end{macro}
%
% \begin{macro} { \@@_set_process_and_print_for_cunum: ,
%    \@@_set_process_and_print_for_cutext: ,
%    \@@_set_process_and_print_for_cuam:
%  }
%  It is handy to set the \enquote{print} and \enquote{process}
%  command all in one go. It's also less messy to use.
%    \begin{macrocode}
\cs_new_protected:Npn \@@_set_process_and_print_for_cunum: 
  { 
    \cs_set_eq:NN \@@_print_input:N \@@_print_numerical_input:N
    \cs_set_eq:NN \@@_print_correct_unit: \@@_cunum_print_correct_unit:
    \cs_set_eq:NN \@@_pre_process_input:NN \@@_do_not_process_input:NN
    \cs_set_eq:NN \@@_post_process_input:NN \@@_do_not_process_input:NN
  }
\cs_new_protected:Npn \@@_set_process_and_print_for_cutext: 
  { 
    \cs_set_eq:NN \@@_print_input:N \@@_cutext_print_input:N
    \cs_set_eq:NN \@@_print_correct_unit: \@@_cutext_print_correct_unitname:
    \cs_set_eq:NN \@@_pre_process_input:NN \@@_cutext_pre_process_input:NN
    \cs_set_eq:NN \@@_post_process_input:NN \@@_cutext_post_process_input:NN
  }
\cs_new_protected:Npn \@@_set_process_and_print_for_cuam: 
  { 
    \cs_set_eq:NN \@@_print_input:N \@@_cuam_print_numerical_input:N
    \cs_set_eq:NN \@@_print_correct_unit: \@@_cuam_print_correct_unitphrase:
    \cs_set_eq:NN \@@_pre_process_input:NN \@@_do_not_process_input:NN
    \cs_set_eq:NN \@@_post_process_input:NN \@@_cuam_post_process_input:NN
  }
%    \end{macrocode}
% \end{macro}
%
%
%
%
%
% \begin{macro}{ \@@_print_numerical_input:N }
%   Prints the nummerical output (if it is not a fraction).
%
%   Changed \lstinline|\l_@@_tmpa_tl| by |\l_@@_translation_tmpa_tl|, it's better that way.
%    \begin{macrocode}
\cs_new_protected:Npn \@@_print_numerical_input:N #1
  {
    \tl_if_in:NnT #1 { . }
      {
        \@@_translate_let:Nxx \l_@@_translation_tmpa_tl
          { decimal-mark } \c_@@_postfix_unitname_tl
        \tl_replace_once:Nnn #1 { . } { \l_@@_translation_tmpa_tl }
      }
    \tl_if_in:NnT #1 { - }
      { \tl_replace_once:NnV #1 { - } \c_@@_minus_tl }
    #1
  }
%    \end{macrocode}
% \end{macro}
%
%
% \begin{macro}{ \@@_print_fractions:nnn }
%   The name of this function is the name of the game: It prints fractions.
%   Furthermore it sets the boolean \lstinline|fraction_in_input_bool| to true
%   so that the correct unit is printed. Instead of \lstinline|\kern| I used the
%   \lstinline|\hbox_to_wd:nn| command (I don't have a better idea).
%
%  Since v.1.10(alpha) it also prints the minus sign (hopefully correct).
%    \begin{macrocode}
\cs_new:Npn \@@_print_fractions:nnn #1#2#3
  {
    \bool_if:NT \l_@@_minus_bool { \c_@@_minus_tl }
    \tl_if_empty:nF {#1} 
      { 
        #1 
        \hbox_to_wd:nn { \l_@@_mixed_frac_dim } { } 
      }
    \@@_frac:nn {#2} {#3}
  }
%    \end{macrocode}
% \end{macro}
%
%
%
%
%
%
% \subsection {Parsing and checking numbers}
%
% This subsection contains macros helping an checking the input.
%
% \begin{macro}{ \@@_parse_input:n }
%   Rescans the input to get rid of spaces and  to make \_ and ? inactive
%   (french with babel makes ? active and changes the definition of it).
%   \lstinline|--| is replaced by \lstinline|\q_@@_range| and the input is parsed
%   (if not empty). Afterwards the units are printed.
%    \begin{macrocode}
\cs_new:Npn \@@_parse_input:n #1
  {
    \tl_set_rescan:Nnn \l_@@_tmpa_tl
      {
        \char_set_catcode_letter:N \_ %
        \char_set_catcode_ignore:N \ %
        \char_set_catcode_other:N ? %
      } {#1}
    \bool_lazy_or:nnTF
      { \l_@@_draft_bool }
      { \tl_if_empty_p:N \l_@@_tmpa_tl }
      {
        \l_@@_tmpa_tl
      }{
        \tl_if_in:NVT \l_@@_tmpa_tl \l_@@_input_range_sign_tl
          {
            \tl_replace_once:NVn \l_@@_tmpa_tl \l_@@_input_range_sign_tl { \q_@@_range } 
            \bool_set_true:N \l_@@_range_in_input_bool            
          }
        \@@_split_input:V \l_@@_tmpa_tl
      }
  }
%    \end{macrocode}
% \end{macro}
%
%
% \begin{macro}{ \@@_split_input:n, \@@_split_input:V }
% Splits the input into   fractions or ranges (\lstinline|q_@@_range|).
% Depending on the input on of the four macros are used.
%    \begin{macrocode}
\cs_new:Npn \@@_split_input:n #1
  {
    \bool_if:NTF \l_@@_range_in_input_bool
      { \@@_parse_range_in_input:ww #1 \q_stop }
      { 
        \tl_if_in:nnTF {#1} { / }
          {
            \bool_lazy_or:nnTF
              { \l_@@_calc_persons_bool }{ \l_@@_eval_fractions_bool }
              { \bool_set_false:N \l_@@_fraction_in_input_bool }
              { \bool_set_true:N \l_@@_fraction_in_input_bool }
            \tl_if_in:nnTF {#1} { _ }
              { \@@_parse_mixed_fraction_in_input:www #1 \q_stop }
              { \@@_parse_fraction_in_input:ww #1 \q_stop }
          }
          { \@@_parse_number_in_input:n {#1} }
      }
  }
\cs_generate_variant:Nn \@@_split_input:n { V }
%    \end{macrocode}
% \end{macro}
%
% \begin{macro}{ \@@_parse_input_and_safe_in:nN }
%   Used to check the input. \lstinline|\l_@@_tmpa_tl| is cleared
%   at the beginning. At first it checks if the first token 
%   is a sign or not. The parsed input is stored into \lstinline|\l_@@_tmpa_tl|.
%    \begin{macrocode}
\cs_new:Npn \@@_parse_input_and_safe_in:nN #1 #2
  {
    \tl_clear:N \l_@@_tmpa_tl
    \bool_set_false:N \l_@@_decimal_in_input_bool
    \@@_parse_sign_and_loop:Nw #1 \q_stop
    \bool_if:NTF \l_@@_error_bool
      { \tl_set:Nn #2 {#1} }
      { \tl_set_eq:NN #2 \l_@@_tmpa_tl }
  }  
%    \end{macrocode}
% \end{macro}
%
% \begin{macro}{ \@@_parse_sign_and_loop:Nw }
%   This function seperates the input into two parts with the first
%   part being the first token. This token is checked whetever or not
%   it is a sign.  If it is a sign it is put into
%   \lstinline|\l_@@_tmpa_tl|, if not it is checked normally by
%   \lstinline|\@@_parse_input_loop_aux:N|.  \#2 (the other
%   tokens) is then given to \lstinline|\tl_map_function:nN|. This
%   also works if the input only conists of one token with \#2 being
%   empty (hopefully).
%    \begin{macrocode}
\cs_new:Npn \@@_parse_sign_and_loop:Nw #1#2 \q_stop
  {
    \tl_if_in:NnTF \l_@@_input_value_signs_tl {#1}
      { \tl_put_right:Nn \l_@@_tmpa_tl {#1} }
      { 
        \tl_map_function:nN {#1} \@@_parse_input_loop_aux:N
      }
    \bool_if:NF \l_@@_error_bool
      { \tl_map_function:nN {#2} \@@_parse_input_loop_aux:N }
  }
%    \end{macrocode}
% \end{macro}
%
% \begin{macro}{ \@@_parse_input_loop_aux:N }
%   Parses the input. It also is more or less copied from
%   \texttt{siunitx}.  Checks if the input consists only of numbers,
%   one decimal-sign and the allowed special input. If an allowed
%   token is found the boolean \lstinline|\l_@@_special_sign_bool| is
%   set to true.
%    \begin{macrocode}
\cs_new:Npn \@@_parse_input_loop_aux:N #1
  {
    \tl_if_in:NnTF \l_@@_input_digits_tl {#1}
      { \tl_put_right:Nn \l_@@_tmpa_tl {#1} }
      {
        \tl_if_in:NnTF \l_@@_input_decimal_mark_tl {#1}
          { 
            \bool_if:NT \l_@@_decimal_in_input_bool 
              { 
                \msg_error:nn { cooking-units } 
                  { Second-decimal-sign-not-allowed } 
              }
            \bool_set_true:N \l_@@_decimal_in_input_bool 
            \tl_put_right:Nn \l_@@_tmpa_tl { . }
          }{
            \tl_if_in:NnTF \l_@@_input_allowed_special_signs_tl {#1}
              { 
                \bool_set_true:N \l_@@_special_sign_bool 
                \tl_put_right:Nn \l_@@_tmpa_tl {#1}
              }{ 
                \bool_if:NTF \l_@@_range_in_input_bool
                  {
                    \tl_if_in:nnTF { / _ } {#1}
                      { \msg_error:nnn { cooking-units } { fraction-not-allowed-with-range } {#1} }
                      { \msg_error:nnn { cooking-units } { Token-not-allowed } {#1} }
                  }{
                    \tl_if_in:nnTF { _ } {#1}
                      { \msg_error:nnn { cooking-units } { missing-slash } {#1} }
                      { \msg_error:nnn { cooking-units } { Token-not-allowed } {#1} }
                   }
                \bool_set_true:N \l_@@_error_bool
                \tl_map_break:
              }
          }
      }
  }
%    \end{macrocode}
% \end{macro}
%
%
% \begin{macro}{\@@_parse_number_in_input:n}
%   If no  fractions and ranges are used this macro activated.
%   It just checks the input, stores the checked (and a bit
%   changed) input into a macro. This macro is given to another
%   function to calculate the input and print it.
%    \begin{macrocode}
\cs_new:Npn \@@_parse_number_in_input:n #1
  {
    \@@_parse_input_and_safe_in:nN {#1} \l_@@_number_tmpa_tl
    \@@_process_and_print_number_in_input:N \l_@@_number_tmpa_tl
  }
\cs_new:Npn \@@_process_and_print_number_in_input:N #1
  {
    \@@_pre_process_input:NN #1 \q_no_value
    \@@_calculate_and_store_in:N #1 
    \@@_post_process_input:NN #1 \q_no_value
    \@@_print_input:N #1
  }
%    \end{macrocode}
% \end{macro}
%
% \begin{macro}{ \@@_parse_range_in_input:ww }
%   Is used if a \lstinline|\q_@@_range| is found inside the input.
%   Seperates the input, checks it and prints it (after calculating
%   it). Furthermore a boolean is set to true, is used to check for
%   errors (ergo if fractions are used).
%    \begin{macrocode}
\cs_new:Npn \@@_parse_range_in_input:ww #1 \q_@@_range #2 \q_stop
  {
    \@@_parse_input_and_safe_in:nN {#1} \l_@@_number_tmpa_tl
    \@@_parse_input_and_safe_in:nN {#2} \l_@@_number_tmpb_tl    
    \@@_pre_process_input:NN \l_@@_number_tmpa_tl \l_@@_number_tmpb_tl
    \@@_calculate_and_store_in:N \l_@@_number_tmpa_tl 
    \@@_calculate_and_store_in:N \l_@@_number_tmpb_tl 
    \@@_post_process_input:NN \l_@@_number_tmpa_tl \l_@@_number_tmpb_tl 
    \@@_print_input:N \l_@@_number_tmpa_tl
    \bool_if:NTF \l_@@_using_cutext_bool
      { \tl_use:N \l_@@_cutext_range_sign_tl }
      { \tl_use:N \l_@@_cunum_range_sign_tl }
    \@@_print_input:N \l_@@_number_tmpb_tl
  }
%    \end{macrocode}
% \end{macro}
%
% \begin{macro}{ \@@_parse_fraction_in_input:ww }
%   If a \lstinline|/| (but no \lstinline|_|) is found inside the input.
%   Well \dots\ does the same as the functions before.
%   If fractions should be evaluated the input is \dots\ well, 
%   evaluated and printed. Otherwise the input is given to 
%   another function which prints the fractions. Note that the empty
%   argument in \lstinline|\@@_print_fractions:nnn| indicates a
%   \enquote{normal} fraction.
%    \begin{macrocode}
\cs_new:Npn \@@_parse_fraction_in_input:ww #1/#2 \q_stop
  {
    \@@_parse_input_and_safe_in:nN {#1} \l_@@_number_tmpa_tl
    \@@_parse_input_and_safe_in:nN {#2} \l_@@_number_tmpb_tl
    \bool_lazy_or:nnTF
      { \l_@@_fraction_in_input_bool }
      { \l_@@_special_sign_bool }
      {
        \@@_tl_if_in_remove_and_set_bool:NnN \l_@@_number_tmpa_tl { - } \l_@@_minus_bool
        \@@_tl_if_in_remove_and_reverse_bool:NnN \l_@@_number_tmpb_tl { - } \l_@@_minus_bool
        \@@_print_fractions:nnn 
          { } 
          { \l_@@_number_tmpa_tl } 
          { \l_@@_number_tmpb_tl } 
      }{
        \@@_tl_set_fp_and_eval:Nn \l_@@_number_tmpa_tl 
          { \l_@@_number_tmpa_tl / \l_@@_number_tmpb_tl }
        \@@_process_and_print_number_in_input:N \l_@@_number_tmpa_tl
      }
  }
%    \end{macrocode}
% \end{macro}
%
% \begin{macro}{ \@@_parse_mixed_fraction_in_input:www }
%   The same procedure as last function? The same procedure as every 
%   function!
%   
%   If it should be evaluated it is important to check if the mixed-fraction
%   part is positive or negative. That's why the minuses are removed \emph{before}
%   checking if there is a fraction (in opposite to the command before).
%
%   \begin{align}
%     \cuam{1_2/3} &=  1 + 2/3 \\
%     \cuam{-1_2/3} &= -1 - 2/3
%   \end{align}
%   
%    \begin{macrocode}
\cs_new:Npn \@@_parse_mixed_fraction_in_input:www #1_#2/#3 \q_stop
  {
    \@@_parse_input_and_safe_in:nN {#1} \l_@@_mixed_fraction_tl
    \@@_parse_input_and_safe_in:nN {#2} \l_@@_number_tmpa_tl
    \@@_parse_input_and_safe_in:nN {#3} \l_@@_number_tmpb_tl
    \@@_tl_if_in_remove_and_set_bool:NnN \l_@@_mixed_fraction_tl { - } \l_@@_minus_bool
    \@@_tl_if_in_remove_and_reverse_bool:NnN \l_@@_number_tmpa_tl { - } \l_@@_minus_bool
    \@@_tl_if_in_remove_and_reverse_bool:NnN \l_@@_number_tmpb_tl { - } \l_@@_minus_bool
    \bool_lazy_or:nnTF
      { \l_@@_fraction_in_input_bool }
      { \l_@@_special_sign_bool }
      {
        \@@_print_fractions:nnn
          { \l_@@_mixed_fraction_tl } 
          { \l_@@_number_tmpa_tl } 
          { \l_@@_number_tmpb_tl }
      }{
        \@@_tl_set_fp_and_eval:Nn \l_@@_number_tmpa_tl
          {
            \bool_if:NTF \l_@@_minus_bool
              { - \l_@@_mixed_fraction_tl - }
              { \l_@@_mixed_fraction_tl + }
            \l_@@_number_tmpa_tl / \l_@@_number_tmpb_tl 
          }
        \@@_process_and_print_number_in_input:N \l_@@_number_tmpa_tl
      }
  }
%    \end{macrocode}
% \end{macro}
%
%
%
%
%
%
% \subsection {Formatiere \& Calculiere}
%
% \begin{macro}{ \@@_calculate_and_store_in:N }
%   
%   \changes {v0.98a} {2016/06/11} {Added warning to show if an unknown value is used.}
%   
%   After parsing the input, it is given to this function. If a not
%   allowed token is found (and \lstinline|\l_@@_error_bool| is set to
%   true) it just prints the input.  Otherwise it checks if a allowed sign was
%   found. If so it is just stored inside \lstinline|\l_@@_tmpa_tl|,
%   otherwise it is calulcated and stored in \lstinline|\l_@@_tmpa_tl|. The input
%   \#1 can be either a token or a number.
%    \begin{macrocode}
\cs_new:Npn \@@_calculate_and_store_in:N #1
  {
    \bool_if:NF \l_@@_error_bool
      {
        \bool_if:NTF \l_@@_special_sign_bool
          { 
            \tl_set_eq:NN \l_@@_tmpa_tl #1
            \msg_warning:nnx { cooking-units } { amount-not-known } \l_@@_tmpa_tl 
          }{
            \bool_lazy_and:nnTF
              { \l_@@_using_cutext_bool } { ! \l_@@_cutext_change_unit_bool }
              { \tl_set_eq:NN \l_@@_tmpa_tl #1 } 
              { \@@_calculate_input_and_store_in:nN {#1} \l_@@_tmpa_tl }
            \bool_if:NT \l_@@_calc_persons_bool
              { 
                \@@_calc_for_number_of_persons_and_store_in:NN 
                  \l_@@_tmpa_tl \l_@@_tmpa_tl 
              }
            \@@_round_calculated_input:NV \l_@@_tmpa_tl \l_@@_tmpa_tl
            \@@_check_temperature_limit:N \l_@@_tmpa_tl
          }
        \tl_set_eq:NN #1 \l_@@_tmpa_tl
      }
  }
%    \end{macrocode}
% \end{macro}
%
% \begin{macro}{ \@@_calculate_input_and_store_in:nN }
% Well \dots\ this function calculates the input. First checks if the
% wanted conversion contains a \lstinline|\l_@@_tmpa_fp| (ergo a \#1 in
% the key definition). If true the conversion token (which already has
% \lstinline|\l_@@_tmpa_fp|  in its input) is executed, else the input
% number is multiplicated with the conversion token.
%    \begin{macrocode}
\cs_new:Npn \@@_calculate_input_and_store_in:nN #1#2
  {
    \fp_set:Nn \l_@@_tmpa_fp {#1}
    \tl_if_in:cnTF { l_@@_tmpa_ \l_@@_given_unit_tl _ tl } { \l_@@_tmpa_fp }
      { 
        \@@_tl_set_fp_and_eval:Nc #2 { l_@@_tmpa_ \l_@@_given_unit_tl  _ tl }
      }{ 
        \@@_tl_set_fp_and_eval:Nn #2 
          { \l_@@_tmpa_fp * \tl_use:c { l_@@_tmpa_ \l_@@_given_unit_tl _ tl }  }
      }
  }
%    \end{macrocode}
% \end{macro}
%
% \begin{macro}{ \@@_calc_for_number_of_persons_and_store_in:NN }
%    \begin{macrocode}
\cs_new:Npn \@@_calc_for_number_of_persons_and_store_in:NN #1#2
  {
    \int_compare:nNnF 
      { \l_@@_wanted_number_of_persons_int }
      =
      { \l_@@_local_number_of_persons_int }
      {
        \@@_tl_set_fp_and_eval:Nn #1
          {
            \l_@@_wanted_number_of_persons_int /  
            \l_@@_local_number_of_persons_int * 
            #2
          }
      }
  }
%    \end{macrocode}
% \end{macro}
%
% \begin{macro}{\@@_check_number_for_rounding:n,\@@_check_number_after_dot_aux:ww}
%   Check if rounding is needed. The auxilary function splits the number into
%   the integer and non-integer part ($12.45$ to $12$ and $45$). 
%   
%  If the non-integer part exists its number of digits $n$ are counted checked
%  if it is smaller then $a$, which depends on |\l_@@_round_to_int_bool|.
%  
%    \begin{macrocode}
\cs_new:Npn \@@_check_number_for_rounding:n #1
  {
    \bool_set_false:N \l_@@_round_number_bool
    \@@_check_number_after_dot_aux:ww #1 . \q_recursion_tail .
    \q_recursion_stop
  }
\cs_new:Npn \@@_check_number_after_dot_aux:ww #1. #2 .
  {
    \quark_if_recursion_tail_stop_do:nn {#2}
      {
        \bool_lazy_or:nnF
          { \l_@@_significant_figures_positive_bool }
          { \l_@@_round_to_int_bool }
          { \bool_set_true:N \l_@@_round_number_bool }
      }
        \int_compare:nNnT 
          { \tl_count:n {#2} } >
          { 
            \bool_if:NTF \l_@@_round_to_int_bool
              { 0 }
              { \l_@@_significant_figures_int }
          }
          { \bool_set_true:N \l_@@_round_number_bool }
    \use_none_delimit_by_q_recursion_stop:w
  }
%    \end{macrocode}
% \end{macro}
%
% \begin{macro}{ \@@_round_calculated_input:Nn,\@@_round_calculated_input:NV }
%   After calculating the numbers are rounded (if needed, safed inside \cs{l_tmpa_bool}). 
%   We test at first  if
%   the input needs to be rounded by comparing the number of tokens after the 
%   decimal with the number of significant figures plus one. Afterwards the 
%   input is expanded and stored in \#1. If rounding needs to be done it happens now.
%    \begin{macrocode}
\cs_new:Npn \@@_round_calculated_input:Nn #1#2
  {
    \@@_check_number_for_rounding:n {#2}
    \tl_set:Nf #1 
      {
        \bool_if:NTF \l_@@_round_number_bool
          { \fp_eval:n { round ( \@@_rounding_function:n {#2} ) } }
          {#2}
     }
  }
\cs_generate_variant:Nn \@@_round_calculated_input:Nn { NV }
%    \end{macrocode}
% \end{macro}
%
%
%
%
%
%
%
%
% \begin{macro}{\@@_check_temperature_limit:N}
%   If \lstinline|check_temperature_bool| is set to true it  now checks
%   if the value is below the absolute temperature. \lstinline|\clist_if_in:nVT|
%   is needed due to \texttt{Re}, a \lstinline|\tl_if_in:nVT| would also check for \texttt{R}
%   and \texttt{e} instead of only \texttt{Re}.
%    \begin{macrocode}
\cs_new:Npn \@@_check_temperature_limit:N #1
  {
    \bool_if:NT \l_@@_check_temperature_bool
      {
        \seq_if_in:NVT \l_@@_temperatures_to_check_seq \l_@@_option_unit_tl
          { 
            \fp_compare:cNnT 
              { c_@@_ \l_@@_option_unit_tl _min_fp } > {#1}
              { 
                \msg_error:nnxx { cooking-units } 
                  { Temperature-too-low } 
                  { #1 \space \l_@@_option_unit_tl } 
                  {  
                    \fp_use:c { c_@@_ \l_@@_option_unit_tl _min_fp } 
                    \space \l_@@_option_unit_tl 
                  }
              }
          }
      }
  }
%    \end{macrocode}
% \end{macro}
%
%
%
%
%
%
%
%
%
%
%
%
%    \begin{macrocode}
\cs_new_nopar:Npn \@@_grab_arrows_for_safety_do_afterwards:nN #1#2
  {
    \str_if_eq:nnTF {#2} { > }
      {
        \tl_put_right:Nx \l_@@_tmpa_tl { \tl_to_str:N > }
        \exp_last_unbraced:NV #1 \l_@@_tmpa_tl
      }{ 
        \tl_put_right:Nn \l_@@_tmpa_tl {#2}
        \@@_grab_arrows_for_safety_do_afterwards:nN {#1} 
      }
  }
\cs_new_nopar:Npn \@@_if_arrow_grab_until_close_do:nnTF #1#2#3#4
  {
    \str_if_eq:nnTF {#1} { < }
      {  
        \tl_if_in:NoT \l_@@_input_allowed_special_signs_tl { < }
          { \msg_error:nn {cooking-units} { <-not-allowed-as-special-sign } }
        \tl_clear:N \l_@@_tmpa_tl
        \tl_put_right:Nx \l_@@_tmpa_tl { \tl_to_str:N < }
        \@@_grab_arrows_for_safety_do_afterwards:nN
          {#3} #2
      }
      {#4}
  }
%    \end{macrocode}
%
%
%
% \subsection {\mycs{cunum}}
%
%
% \begin{macro}{ \cunum }
% The main command of this package. 
%    \begin{macrocode}
\NewDocumentCommand \cunum { d<> O{} m O{} m }
  {
    \@@_if_arrow_grab_until_close_do:nnTF {#3} {#5}
      { \cunum }
      { 
        \group_begin:
        \@@_cunum:nnnnn {#1} {#2} {#3} {#4} {#5}
        \group_end:
      }
  }
%    \end{macrocode}
% \end{macro}
%
%
% \begin{macro}{ \@@_cunum }
%    \begin{macrocode}
\cs_new:Npn \@@_cunum:nnnnn #1#2#3#4#5
  {
    \@@_cunum_initialise:nnnnn {#1} {#2} {#3} {#4} {#5}
    \@@_parse_input:n {#3}
    \@@_print_correct_unit:
  }
%    \end{macrocode}
% \end{macro}
%
%
% \begin{macro}{ \@@_cunum_initialise:nnnn }
%   Lots of things to do in this function. It checks if the unit is defined or not,
%   sets the keys for converting the unit and sets the optional unit (the unit
%   our input is converted to) accordingly, sets the options again (to get the
%   option-specific options) and \dots\ yeah.
%   
%   First parses the change of unit ('set groups'), afterwards set 
%   the predefined options for the new unit. Afterwards 
%   set the 'normal' options given by the optional argument.
%
%    \begin{macrocode}
\cs_new_protected:Npn \@@_cunum_initialise:nnnnn #1#2#3#4#5
  {
    \@@_set_process_and_print_for_cunum:
    \tl_set:Nn \l_@@_phantom_tl {#4}
    \@@_initialise_default:n {#5}
    \@@_initialise_unit_change:n {#2}
    \@@_initialise_after_unit_change:nn {#1} {#2}
  }
%    \end{macrocode}
% \end{macro}
%
%
%
%
% \begin{macro}{ \@@_cunum_print_correct_unit: }
%   The invisible space is added by a \lstinline|\phantom|, afterwards
%   \lstinline|value_unit_space_tl| is used (which is set to
%   \lstinline|\thinspace| by default) and if either special signs or
%   fractions are parsed the input-unit is printed else the converted
%   unit is.
%    \begin{macrocode}
\cs_new:Npn \@@_cunum_print_correct_unit:
  {
    \tl_if_empty:NF \l_@@_phantom_tl { \phantom { \l_@@_phantom_tl } }
    \tl_use:N \l_@@_value_unit_space_tl
    \bool_lazy_any:nTF
      {
        { \l_@@_fraction_in_input_bool }
        { \l_@@_special_sign_bool }
        { \l_@@_draft_bool }
      }
      { \@@_translate:xx \l_@@_given_unit_tl \c_@@_postfix_unit_tl }
      { \@@_translate:xx \l_@@_option_unit_tl \c_@@_postfix_unit_tl }
  }
%    \end{macrocode}
% \end{macro}
%
%
% \section { cutext \& Cutext }
%
%
% A quite primitive implentation of \opt{cutext-to-cunum}, but
% sufficient for now.
%    \begin{macrocode}
\NewDocumentCommand \cutext { d<> O{} m m }
  {
    \@@_if_arrow_grab_until_close_do:nnTF {#3} {#4}
      { \cutext }
      { 
        \group_begin:
        \@@_cutext_initialise:nnn {#1} {#2} {#4}
        \bool_set_false:N \l_@@_cutext_uppercase_word_bool
        \@@_cutext:nnnn {#1} {#2} {#3} {#4}
        \group_end:
      }
  }
%    \end{macrocode}
%
%    \begin{macrocode}
\NewDocumentCommand \Cutext { d<> O{} m m }
  {
    \@@_if_arrow_grab_until_close_do:nnTF {#3} {#4}
      { \Cutext }
      { 
        \group_begin:
        \@@_cutext_initialise:nnn {#1} {#2} {#4}
        \bool_set_true:N \l_@@_cutext_uppercase_word_bool
        \@@_cutext:nnnn {#1} {#2} {#3} {#4}
        \group_end:
      }
  }
%    \end{macrocode}
%
%    \begin{macrocode}
\cs_new_protected:Npn \@@_cutext_initialise:nnn #1#2#3
  {
    \bool_set_true:N \l_@@_using_cutext_bool
    \@@_set_process_and_print_for_cutext:
    \@@_initialise_default:n {#3}
    \bool_if:NTF \l_@@_cutext_change_unit_bool
      { \@@_initialise_unit_change:n {#2} }
      { \tl_set_eq:NN \l_@@_option_unit_tl \l_@@_given_unit_tl }
    \@@_initialise_after_unit_change:nn {#1} {#2}
  }
%    \end{macrocode}
%
%    \begin{macrocode}
% #1: label, #2: Options, #3: Values, #4: unit
\cs_new:Npn \@@_cutext:nnnn #1#2#3#4
  {
    \bool_if:NTF \l_@@_cutext_to_cunum_bool
      {
        \bool_set_false:N \l_@@_using_cutext_bool
        \cunum <#1> [#2] {#3} {#4} 
      }{
        \bool_if:NTF \l_@@_cutext_old_bool
          {
            \tl_set_eq:NN \l_@@_option_unit_tl \l_@@_given_unit_tl
            \@@_old_cutext_default:nnn {#2} {#3} {#4}
          }{
            \@@_parse_input:n {#3}
          }
        \@@_print_correct_unit:
      }
  }
%    \end{macrocode}
%
%
%
% \begin{macro} {\@@_cutext_print_input:Nn , \@@_cutext_print_input:NV}
% \changes {2016/06/11} {0.99} {New.}
%    \begin{macrocode}
\cs_new_protected:Npn \@@_cutext_print_input:Nn #1#2
  {
    \bool_if:NTF \l_@@_lokal_numeral_bool
      {
        \@@_int_if_equal_one:nTF {#1}
          {
            \@@_translate_one_to_and_check_existance:Nx \l_@@_translation_tmpa_tl {#2}
            \bool_if:NTF \l_@@_cutext_uppercase_word_bool
              {
                \exp_args:Nx \tl_upper_case:n { \tl_head:V \l_@@_translation_tmpa_tl }
                \tl_tail:V \l_@@_translation_tmpa_tl
              }
              { \l_@@_translation_tmpa_tl }
          }{ 
            \bool_if:NTF \l_@@_cutext_uppercase_word_bool
              { \@@_print_Numeral:n {#1} }
              { \@@_print_numeral:n {#1} }
          }
      }
      { \@@_print_numerical_input:N #1 }
  }
\cs_generate_variant:Nn \@@_cutext_print_input:Nn { NV }
\cs_new_protected:Npn \@@_cutext_print_input:N #1
  { 
    \bool_lazy_any:nTF
      {
        { \l_@@_fraction_in_input_bool }
        { \l_@@_special_sign_bool }
        { \l_@@_draft_bool }
      }
      { \@@_cutext_print_input:NV #1 \l_@@_given_unit_tl }
      { \@@_cutext_print_input:NV #1 \l_@@_option_unit_tl }
  }
%    \end{macrocode}
% \end{macro}
%
%    \begin{macrocode}
\cs_new:Npn \@@_cutext_print_correct_unitname:
  {
    \l_@@_cutext_space_tl
    \bool_lazy_any:nTF
      {
        { \l_@@_fraction_in_input_bool }
        { \l_@@_special_sign_bool }
        { \l_@@_draft_bool }
      }
      { \@@_translate:xx \l_@@_given_unit_tl \c_@@_postfix_unitname_pl_tl }
      {
        \@@_fp_if_equal_one:nTF { \l_@@_cutext_last_value_tl }
          { \@@_translate:xx \l_@@_option_unit_tl \c_@@_postfix_unitname_tl }
          { \@@_translate:xx \l_@@_option_unit_tl \c_@@_postfix_unitname_pl_tl }
      }
  }
%    \end{macrocode}
%
%    \begin{macrocode}
\cs_new:Npn \@@_cutext_pre_process_input:NN #1#2
  {
    \@@_cutext_check_unitname_consequences:NN #1#2
  }
%    \end{macrocode}
% It doesn't matter if I check if the singular or the plural translation exists,
% as the plural one exists if the singular exists and vice versa.
%    \begin{macrocode}
\cs_new:Npn \@@_cutext_check_unitname_consequences:NN #1#2
  {
    \bool_lazy_any:nTF
      {
        { \l_@@_fraction_in_input_bool }
        { \l_@@_special_sign_bool }
        { \l_@@_draft_bool }
      }
      { \tl_set_eq:NN \l_@@_tmpb_tl \l_@@_given_unit_tl }
      { \tl_set_eq:NN \l_@@_tmpb_tl \l_@@_option_unit_tl } 
    \@@_unitname_get:NxF \l_@@_tmpa_tl \l_@@_tmpb_tl
      { 
        \msg_warning:nnx 
          { cooking-units } 
          { cutext-no-translation-available } 
          \l_@@_tmpb_tl
        \bool_set_false:N \l_@@_using_cutext_bool 
        \@@_set_process_and_print_for_cunum:
      }
  }
%    \end{macrocode}
%
%
%
%    \begin{macrocode}
\cs_new:Npn \@@_cutext_post_process_input:NN #1#2
  {
    \bool_if:NF \l_@@_special_sign_bool
      {
        \@@_cutext_post_process_input_aux:NN #1#2
      }
  }
%    \end{macrocode}
%    \begin{macrocode}
\cs_new:Npn \@@_cutext_post_process_input_aux:NN #1#2
  {
    \@@_bool_set_if_integers:NNN \l_@@_tmpa_bool #1 #2
    \bool_if:NTF \l_@@_range_in_input_bool
      { \tl_set_eq:NN \l_@@_cutext_last_value_tl #2 }
      { \tl_set_eq:NN \l_@@_cutext_last_value_tl #1 }
    \bool_if:NT \l_@@_lokal_numeral_bool
      { \@@_cutext_post_process_input_numerals:NN #1#2 }
  }
\cs_new:Npn \@@_cutext_post_process_input_numerals:NN #1#2
  {
    \bool_if:NTF \l_@@_tmpa_bool
      {
        \bool_if:NTF \l_@@_range_in_input_bool
          {
            \bool_lazy_and:nnF
              { \int_compare_p:nNn {#1} < { \l_@@_print_numerals_below_int } }
              { \int_compare_p:nNn {#2} < { \l_@@_print_numerals_below_int } }
              { \bool_set_false:N \l_@@_lokal_numeral_bool }
          }{
            \int_compare:nNnF {#1} < { \l_@@_print_numerals_below_int }
              { \bool_set_false:N \l_@@_lokal_numeral_bool } 
          }
      }
      { \bool_set_false:N \l_@@_lokal_numeral_bool }
  }
%    \end{macrocode}
%
%
%
%
%
%
%
%
%
% \section{cuam}
%
%    \begin{macrocode}
\tl_const:Nn \c_@@_cuam_marker_tl { @@_cunum }
\tl_new:c { l_@@_tmpa_ \c_@@_cuam_marker_tl  _ tl }
\tl_set:cn { l_@@_tmpa_ \c_@@_cuam_marker_tl  _ tl } { \c_one_fp }
\clist_new:c { l_@@_predefined_option_ \c_@@_cuam_marker_tl _clist }
%    \end{macrocode}
%
% Replaces and extends \mycs{cufrac}.
% \begin{macro}{\cuam}
%    \begin{macrocode}
\NewDocumentCommand \cuam { d<> O{} m }
  {
    \@@_if_arrow_grab_until_close_do:nnTF {#3} { }
      { \cuam }
      { 
        \group_begin:
        \@@_cuam_initialise:nn {#1} {#2}
        \@@_cuam:n {#3}
        \group_end:
      }
  }
%    \end{macrocode}
% \end{macro}
%
%    \begin{macrocode}
\cs_new:Npn \@@_cuam:n #1
  {
    \bool_if:NTF \l_@@_cuam_old_bool
        { \@@_cuam_old:n {#1} }
        { 
          \@@_parse_input:n {#1}  
          \@@_print_correct_unit:
        }
  }
%    \end{macrocode}
%
%    \begin{macrocode}
\cs_new_protected:Npn \@@_cuam_initialise:nn #1#2
  {
    \@@_set_process_and_print_for_cuam:
    \tl_set_eq:NN \l_@@_given_unit_tl \c_@@_cuam_marker_tl
    \tl_set_eq:NN \l_@@_option_unit_tl \c_@@_cuam_marker_tl
    \bool_set_false:N \l_@@_lokal_phrase_bool
    \@@_initialise_after_unit_change:nn {#1} {#2}
  }
%    \end{macrocode}
%
% Important thing is: This whole process must not acticate if a
% special sign is in the input. This is due to mathematical comparisons.
%    \begin{macrocode}
\cs_new_protected:Npn \@@_cuam_post_process_input:NN #1#2
  {
    \bool_if:NF \l_@@_special_sign_bool
      { \@@_cuam_post_process_input_aux:NN #1 #2 }
  }
%    \end{macrocode}
%
% Just a helper macro to tidy up the main macro.  The first thing to
% do is to check if the nummerical input consits of integers. This
% is done both both numerical values (if a range exists).
% Afterwards check if phrases are used and can be used at all. This needs
% to be done before the numeral check as it can change the input.
%    \begin{macrocode}
\cs_new:Npn \@@_cuam_post_process_input_aux:NN #1#2
  {
    \@@_bool_set_if_integers:NNN \l_@@_tmpa_bool #1 #2
    \bool_lazy_and:nnT 
      { \l_@@_use_phrases_bool } { \l_@@_tmpa_bool }
      {
        \@@_cuam_post_process_input_phrases:NN #1#2
      }
    \bool_if:NT \l_@@_lokal_numeral_bool
      {
        \@@_cuam_post_process_input_numerals:NN #1#2
      }
  }
%    \end{macrocode}
%
% This command is more or less straightforward. Does a list of phrases exist
% for the current language? If yes, parse the input and check if phrases are used.
% It is important for ranges to check both values and then decide if phrases
% are used or not.
%    \begin{macrocode}
\cs_new:Npn \@@_cuam_post_process_input_phrases:NN #1#2
  {
    \@@_get_phrase_for_current_langauge:NNT 
      \l_@@_phrase_prop 
      \l_@@_phrase_numbers_clist
      {
        \@@_cuam_post_process_input_phrases_aux:NNN 
          #1 \l_@@_tmpa_int \l_@@_phrase_number_tl
        \bool_if:NT \l_@@_lokal_phrase_bool
          {
            \bool_if:NTF \l_@@_range_in_input_bool
              {
                \@@_cuam_post_process_input_phrases_aux:NNN 
                  #2 \l_@@_tmpb_int \l_@@_tmpb_tl
                \bool_lazy_and:nnF
                  { \l_@@_lokal_phrase_bool }
                  { \tl_if_eq_p:NN \l_@@_phrase_number_tl \l_@@_tmpb_tl }
                  { \bool_set_false:N \l_@@_lokal_phrase_bool }
                \bool_if:NT \l_@@_lokal_phrase_bool
                  {
                    \tl_set:NV #1 \l_@@_tmpa_int
                    \tl_set:NV #2 \l_@@_tmpb_int
                    \@@_cuam_get_phrase_name:NVN 
                      \l_@@_phrase_phrase_tl \l_@@_phrase_number_tl \l_@@_tmpb_int
                  }
              }{ 
                \tl_set:NV #1 \l_@@_tmpa_int 
                \@@_cuam_get_phrase_name:NVN 
                  \l_@@_phrase_phrase_tl \l_@@_phrase_number_tl \l_@@_tmpa_int
              }
          }
      }
  }
%    \end{macrocode}
%
% Processing numerals. This function only happens if numerals are used
% and because numerals have to be integers I can set the numeral boolean
% false if there is no integer. 
%
% Furthermore, due to knowing that we have integers we can use
% |\int_compare_p:nNn| instead of the |fp|-equivalent.
%    \begin{macrocode}
\cs_new:Npn \@@_cuam_post_process_input_numerals:NN #1#2
  {
    \bool_if:NTF \l_@@_tmpa_bool
      {
        \bool_if:NTF \l_@@_range_in_input_bool
          {
            \bool_lazy_and:nnF
              { \int_compare_p:nNn {#1} < { \l_@@_print_numerals_below_int } }
              { \int_compare_p:nNn {#2} < { \l_@@_print_numerals_below_int } }
              { \bool_set_false:N \l_@@_lokal_numeral_bool }
          }{
            \int_compare:nNnF {#1} < { \l_@@_print_numerals_below_int }
              { \bool_set_false:N \l_@@_lokal_numeral_bool } 
          }
      }
      { \bool_set_false:N \l_@@_lokal_numeral_bool }
  }
%    \end{macrocode}
%
%
%    \begin{macrocode}
\cs_new:Npn \@@_cuam_post_process_input_phrases_aux:NNN #1#2#3
  {
    \bool_set_false:N \l_@@_lokal_phrase_bool
    \clist_map_inline:Nn \l_@@_phrase_numbers_clist
      {
        \int_compare:nNnF { \int_abs:n {##1} } > {#1}
          {
            \int_compare:nNnTF {##1} < { \c_zero }
              { \int_set_eq:NN \l_tmpa_int \c_one }
              { \int_set:Nn \l_tmpa_int { \int_div_truncate:nn {#1} {##1} } }
            \int_compare:nNnT { \int_abs:n {##1} * \l_tmpa_int } = {#1}
              {
                \int_set_eq:NN #2 \l_tmpa_int
                \tl_set:Nn #3 {##1}
                \bool_set_true:N \l_@@_lokal_phrase_bool
                \clist_map_break:
              }
          }
      }
  }
\cs_new:Npn \@@_cuam_get_phrase_name:NnN #1#2#3
  {
    \@@_int_if_equal_one:nTF {#3}
      { \prop_get:NnN \l_@@_phrase_prop {#2} #1 }
      { \prop_get:NnN \l_@@_phrase_prop { #2-pl } #1 }
  }
\cs_generate_variant:Nn \@@_cuam_get_phrase_name:NnN { NVN }
%    \end{macrocode}
%
% Note: THis command checks if \emph{both} a phrase and a numeral is 
% used to decide whetever or not to print numerals. This is due to
% the $1$ being dependent (in some languages) depending on the gender
% of the following word. Currently testing only |\l_@@_lokal_numeral_bool| doesn't
% work due to the   $1$ having a special parsing.
%    \begin{macrocode}
\cs_new_protected:Npn \@@_cuam_print_numerical_input:N #1
  {
    \bool_lazy_and:nnTF
      { \l_@@_lokal_phrase_bool } 
      { \l_@@_lokal_numeral_bool }
      {
        \@@_int_if_equal_one:nTF {#1}
          {
            \@@_translate_one_to_and_check_existance:Nx \l_@@_translation_tmpa_tl
              { \l_@@_phrase_number_tl -phrase-gender }
            \l_@@_translation_tmpa_tl 
          }
          { \exp_args:NV \@@_print_numeral:n #1 } 
      }{ \@@_print_numerical_input:N #1 }
  }
%    \end{macrocode}
%
%    \begin{macrocode}
\cs_new:Npn \@@_cuam_print_correct_unitphrase:
  {
    \bool_if:NT \l_@@_lokal_phrase_bool
      {
        \l_@@_cuphrase_space_tl 
        \l_@@_phrase_phrase_tl
      }
  }
%    \end{macrocode}
%
%
%
%
%
%
% \section {cukeys}
%
% \subsection {Define Keys}
%
% \begin{macro}{ \cudefinekeys }
%   Defining keys.
%    \begin{macrocode}
\NewDocumentCommand \cudefinekeys { m m }
  {
    \bool_set_false:N \l_@@_single_key_bool
    \@@_cukeys_define_keys_and_single_keys:nn {#1} {#2}
  }
%    \end{macrocode}
% \end{macro}
%
% \begin{macro}{ \cudefinesinglekey }
%   Again, but with the boolean set to true.
%    \begin{macrocode}
\NewDocumentCommand \cudefinesinglekey { m m }
  {
    \bool_set_true:N \l_@@_single_key_bool
    \@@_cukeys_define_keys_and_single_keys:nn {#1} {#2}
  }
%    \end{macrocode}
% \end{macro}
%
% \begin{macro}{ \@@_cukeys_define_keys_and_single_keys:nn }
%   First checks whetever the unit is defined or not, then clears some macros
%   which are needed later. Further procession depends on the boolean.
%   If more than one key is created, the value \lstinline|\l_@@_tmpa_fp| is set to one.
%   This is important for adding keys where \lstinline|\l_@@_tmpa_fp| is changed accordingly.
%    \begin{macrocode}
\cs_new:Npn \@@_cukeys_define_keys_and_single_keys:nn #1#2
  {
    \@@_error_if_unit_not_defined:n {#1}
    \tl_if_blank:nF {#2}
      {
        \seq_clear:N \l_@@_tmpa_seq
        \prop_clear:N \l_@@_tmpa_prop
        \bool_if:NTF \l_@@_single_key_bool
          {
            \@@_cukeys_parse_and_create_single_key:nn {#1} {#2}
          }{
            \fp_set_eq:NN \l_@@_tmpa_fp \c_one_fp
            \tl_set:Nn \l_@@_given_unit_tl {#1}
            \@@_cukeys_parse_and_create_keys:nn {#1} 
              { 
                {#1} { \c_one_fp } #2 
              } 
          }
      }
  }
%    \end{macrocode}
% \end{macro}
%
% \begin{macro}{ \@@_cukeys_parse_and_create_keys:nn }
%   A simple parsing function using quarks (which are pretty handy).
%   At first parses the input, then creates the property lists for each key
%   (containing all the values) and at last defines the keys.
%    \begin{macrocode}
\cs_new:Npn \@@_cukeys_parse_and_create_keys:nn #1#2
  {
    \@@_cukeys_parse_input:nn #2
    \q_recursion_tail \q_recursion_tail \q_recursion_stop
    \@@_cukeys_create_key_prop:n {#1}
    \clist_set_from_seq:NN \l_@@_tmpa_clist \l_@@_tmpa_seq
    \@@_cukeys_define_keys:V \l_@@_tmpa_clist
  }
%    \end{macrocode}
% \end{macro}
%
% \begin{macro}{ \@@_cukeys_parse_input:nn }
% Yeah \dots\ \lstinline|\l_@@_tmpa_clist| stores all the used unit-keys, while
% the property list saves the relation to each other. This cycle is repeated until an
% recursion tail is found. If you define a new key, \verb|\l_@@_tmpa_fp| is set to 
% 1 and changed later if a new key is added.
%    \begin{macrocode}
\cs_new:Npn \@@_cukeys_parse_input:nn #1#2
  {
    \quark_if_recursion_tail_stop:n {#1} 
    \quark_if_recursion_tail_stop_do:nn {#2} 
      { \msg_error:nn { cooking-units } { missing-argument } }
    \@@_error_if_unit_not_defined:n {#1}
    \seq_put_right:Nn \l_@@_tmpa_seq {#1}
    \prop_put:Nnx \l_@@_tmpa_prop {#1} { \fp_eval:n { (#2) / \l_@@_tmpa_fp } }
    \@@_cukeys_parse_input:nn
  }
%    \end{macrocode}
% \end{macro}
%
% \begin{macro}{ \@@_cukeys_create_key_prop:n }
%
% \changes {2016/09/22} {1.02f} {Save 'Liste', 'prop' and 'Erstes Ding' to \cs{l_tmpb_prop}  and set the unit-prop lists equal to \cs{l_tmpb_prop}.}
% \changes {2016/09/22} {1.02g} {Do not need \cs{tl_clear_new:c} as token list is already defined.}
%
% All linked unit-keys are stored within \lstinline|\l_@@_tmpa_clist| and are mapped
% one after another. At first a property list is created (or cleared), this property list
% stores the units linked to this unit, saves the created property list which contains the 
% numerical relation of each unit. \lstinline|Erstes Ding| (firsth thing) stores
% the first unit used for defining the keys, it is needed later for adding keys.
% 
% For each unit the other units are added in the cleared or newly created property list
% as keys with their value being the correct numerical relation. For example:
%
% \begin{align*}
%  \SI{1}{\kg}&=  \SI{1}{\kg} \\
%  \SI{1}{\kg}&=  \SI{100}{\deka\gram} \\
%  \SI{1}{\kg}&=  \SI{1000}{\g} \\
%  \SI{1}{\kg}&=  \SI{35.27399}{oz} 
% \end{align*}
%
% Therefore the property list for \lstinline|kg| contains the keys and values: 
% \texttt{kg}=\texttt{1}
% \texttt{dag}=\texttt{100}, \texttt{g}=\texttt{1000} and \texttt{oz}= \texttt{35.27399}.
%
% For the next unit (\texttt{dag} in this case) a new property list is created, relation above
% stored inside etc. The  condition now is that \texttt{dag}=\texttt{1}, therefore every number
% is divided the the number \texttt{100}:
%
% \begin{align*}
%  \SI{1}{\deka\gram}&=  \SI{0.01}{\kg} \\
%  \SI{1}{\deka\gram}&=  \SI{1}{\deka\gram} \\
%  \SI{1}{\deka\gram}&=  \SI{10}{\g} \\
%  \SI{1}{\deka\gram}&=  \SI{0.3527399}{\oz} 
% \end{align*}
%
% Same for \texttt{g}:
%
% \begin{align*}
%  \SI{1}{\g}&=  \SI{0.001}{\kg} \\
%  \SI{1}{\g}&=  \SI{0.1}{\deka\gram} \\
%  \SI{1}{\g}&=  \SI{1}{\g} \\
%  \SI{1}{\g}&=  \SI{0.03527399}{oz} 
% \end{align*}
%
% and \texttt{oz}
%
% \begin{align*}
%  \SI{1}{oz}&=  \SI{0.0283495}{\kg} \\
%  \SI{1}{oz}&=  \SI{2.83495}{\deka\gram} \\
%  \SI{1}{oz}&=  \SI{28.3495}{\g} \\
%  \SI{1}{oz}&=  \SI{1}{oz} 
% \end{align*}
%
%    \begin{macrocode}
\cs_new:Npn \@@_cukeys_create_key_prop:n #1
  {
    \prop_clear:N \l_@@_tmpb_prop
    \prop_put:NnV \l_@@_tmpb_prop { Liste } \l_@@_tmpa_seq
    \prop_put:NnV \l_@@_tmpb_prop { prop } \l_@@_tmpa_prop
    \prop_put:NnV \l_@@_tmpb_prop { Erstes Ding } \l_@@_given_unit_tl
    \seq_map_inline:Nn \l_@@_tmpa_seq
      {
        \prop_set_eq:cN { l_@@_cukeys_ ##1 _prop } \l_@@_tmpb_prop
        \tl_set_eq:cN   { l_@@_tmpa_ ##1 _tl } \c_one_fp
        \seq_map_inline:Nn \l_@@_tmpa_seq
          {
            \prop_put:cnx { l_@@_cukeys_ ##1 _prop }
              {####1}
              { 
                \fp_eval:n 
                  { 
                    ( \prop_item:Nn \l_@@_tmpa_prop {####1} ) / 
                    ( \prop_item:Nn \l_@@_tmpa_prop {##1} )
                  } 
              }
          }
      }      
  }
%    \end{macrocode}
% \end{macro}
%
% \begin{macro}{ \@@_cukeys_define_keys:n , \@@_cukeys_define_keys:V }
% Defining the keys: It maps through the list of unit-keys and creates a
% unit-key respectively.
%    \begin{macrocode}
\cs_new:Npn \@@_cukeys_define_keys:n #1
  {
   \seq_map_inline:Nn \l_@@_tmpa_seq
     {
       \seq_if_in:NnF \l_@@_list_of_defined_keys_seq {##1} 
         { \seq_put_right:Nn \l_@@_list_of_defined_keys_seq {##1} }
       \keys_define:nn { cooking-units }
         {
           ##1 .choices:Vn =
             \l_@@_tmpa_clist
             {
               \@@_cukeys_define_keys_and_single_key_aux:n {##1}
             } ,
           ##1 / unknown .code:n= 
             { 
               \seq_set_split:Nnn \l_tmpa_seq { , } {#1}
               \msg_error:nnxxx
                 { cooking-units } 
                 { key-choice-unknown }
                 {##1} 
                 {####1} 
                 { \seq_use:Nnnn \l_tmpa_seq { ',~ ' } { ',~ ' } { ' ~ and ~ ' } } 
             } ,
           ##1 .default:n = {##1} ,
           ##1 .groups:n = { change-unit }
         }
     }
  }
\cs_generate_variant:Nn \@@_cukeys_define_keys:n { V }
%    \end{macrocode}
% \end{macro}
%
% \begin{macro}{\@@_cukeys_define_keys_and_single_key_aux:n}
% 
%    \begin{macrocode}
\cs_new:Npn \@@_cukeys_define_keys_and_single_key_aux:n #1
  {
    \prop_get:cVc 
      { l_@@_cukeys_#1_prop } 
      \l_keys_choice_tl 
      { l_@@_tmpa_ #1_tl }
    \prop_put:NnV \l_@@_change_unit_prop {#1} \l_keys_choice_tl
  }
%    \end{macrocode}
% \end{macro}
%
%
%    \begin{macrocode}
\cs_new:Npn \@@_cukeys_parse_and_create_single_key:nn #1#2
  {
    \tl_set_rescan:Nnn \l_@@_tmpa_tl
      {
        \char_set_catcode_letter:N \# %
        \char_set_catcode_ignore:N\ %
      } { {#1} { \c_one_fp } #2 }
    \@@_cusinglekeys_parse_input:V \l_@@_tmpa_tl
    \@@_cusinglekeys_create_key_prop:n {#1}
    \clist_set_from_seq:NN \l_@@_tmpa_clist \l_@@_tmpa_seq
    \@@_cukeys_define_singlekey:nV {#1} \l_@@_tmpa_clist
  }
%    \end{macrocode}
%
%    \begin{macrocode}
\cs_new:Npn \@@_cusinglekeys_parse_input:n #1
  {
    \@@_cusinglekeys_parse_input_aux:nn #1
    \q_recursion_tail \q_recursion_tail \q_recursion_stop
  }
\cs_generate_variant:Nn \@@_cusinglekeys_parse_input:n { V }
%
\cs_new:Npn \@@_cusinglekeys_parse_input_aux:nn #1#2
  {
    \quark_if_recursion_tail_stop:n {#1} 
    \quark_if_recursion_tail_stop_do:nn {#2} 
      { \msg_error:nn { cooking-units } { missing-argument } }
    \@@_error_if_unit_not_defined:n {#1}
    \seq_put_right:Nn \l_@@_tmpa_seq {#1}
    \tl_set:Nn \l_@@_tmpa_tl {#2}
    \tl_replace_all:NVn \l_@@_tmpa_tl \c_@@_input_str_hash_one_tl { \l_@@_tmpa_fp }
    \prop_put:NnV \l_@@_tmpa_prop {#1} \l_@@_tmpa_tl
    \@@_cusinglekeys_parse_input_aux:nn
  }
%    \end{macrocode}
%
%    \begin{macrocode}
\cs_new:Npn \@@_cusinglekeys_create_key_prop:n #1
  {
    \tl_set_eq:cN { l_@@_tmpa_ #1 _tl } \c_one_fp
    \prop_set_eq:cN { l_@@_cukeys_ #1 _prop } \l_@@_tmpa_prop
    \prop_put:cnn { l_@@_cukeys_ #1 _prop } { Erstes Ding } {#1}
    \prop_put:cnV { l_@@_cukeys_ #1 _prop } { Liste } \l_@@_tmpa_seq
    \prop_put:cnV { l_@@_cukeys_ #1 _prop } { prop } \l_@@_tmpa_prop
  }
%    \end{macrocode}
%
%    \begin{macrocode}
\cs_new:Npn \@@_cukeys_define_singlekey:nn #1#2
  {
    \seq_if_in:NnF \l_@@_list_of_defined_keys_seq {#1} 
      {  \seq_put_right:Nn \l_@@_list_of_defined_keys_seq {#1} }
     \keys_define:nn { cooking-units }
         {
           #1 .choices:Vn =
             \l_@@_tmpa_clist
               {
                 \@@_cukeys_define_keys_and_single_key_aux:n {#1}
               } ,
           #1 / unknown .code:n= 
             { 
               \seq_set_split:Nnn \l_tmpa_seq { , } {#2}
               \msg_error:nnxxx
                 { cooking-units } 
                 { key-choice-unknown }
                 {#1} 
                 {##1} 
                 { \seq_use:Nnnn \l_tmpa_seq { ',' } { ',' } { ' ~ and ~ ' } }  
             } ,
           #1 .default:n = {#1} ,
           #1 .groups:n = { change-unit } ,
         }
  }
\cs_generate_variant:Nn \@@_cukeys_define_singlekey:nn { nV }
%    \end{macrocode}
%
%
% \section {Adding Keys}
%
% Question to me: Why do we need \cs{l_tmpa(b)_tl}?
% Maybe due to \cs{cuaddtokeys}?
%    \begin{macrocode}
\NewDocumentCommand \cuaddkeys { m m }
  {
    \bool_set_false:N \l_@@_single_key_bool
    \tl_set:Nn \l_@@_tmpa_tl {#1}
    \tl_set:Nn \l_@@_tmpb_tl {#2}
    \@@_cukeys_add_keys_or_single_keys:VV \l_@@_tmpa_tl \l_@@_tmpb_tl
  }
%    \end{macrocode}
%
%    \begin{macrocode}
\NewDocumentCommand \cuaddsinglekeys { m m }
  {
    \bool_set_true:N \l_@@_single_key_bool
    \tl_set:Nn \l_@@_tmpa_tl {#1}
    \tl_set:Nn \l_@@_tmpb_tl {#2}
    \@@_cukeys_add_keys_or_single_keys:VV \l_@@_tmpa_tl \l_@@_tmpb_tl
  }
%    \end{macrocode}
%
%    \begin{macrocode}
\NewDocumentCommand \cuaddtokeys { m m m }
  {
    \bool_set_false:N \l_@@_single_key_bool
    \tl_set:Nn \l_@@_tmpa_tl {#1}
    \tl_set:Nn \l_@@_tmpb_tl { {#2} { \fp_eval:n { \c_one_fp / (#3) } } }
    \@@_cukeys_add_keys_or_single_keys:VV \l_@@_tmpa_tl \l_@@_tmpb_tl
  }
%    \end{macrocode}
%
%
%
%    \begin{macrocode}
\cs_new:Npn \@@_cukeys_add_keys_or_single_keys:nn #1#2
  {
    \@@_error_if_unit_not_defined:n {#1}
    \seq_if_in:NnF \l_@@_list_of_defined_keys_seq {#1}
      { \msg_error:nnn { cooking-units } { Key-not-defined } {#1} }
    \tl_if_blank:nF {#2}
      {
        \@@_cukeys_add_keys_and_single_key_aux:n {#1}
        \bool_if:NTF \l_@@_single_key_bool
          { \@@_cukeys_parse_and_create_single_key:nn {#1} {#2} }
          { \@@_cukeys_parse_and_create_keys:nn {#1} {#2} }
      }
  }
\cs_generate_variant:Nn \@@_cukeys_add_keys_or_single_keys:nn { VV }
%    \end{macrocode}
%
%
%    \begin{macrocode}
\cs_new:Npn \@@_cukeys_add_keys_and_single_key_aux:n #1
  { 
    \prop_get:cnN { l_@@_cukeys_ #1 _prop } { Liste } \l_@@_tmpa_seq
    \prop_get:cnN { l_@@_cukeys_ #1 _prop } { prop } \l_@@_tmpa_prop
    \prop_get:cnN { l_@@_cukeys_ #1 _prop } { Erstes Ding } \l_@@_given_unit_tl
    \prop_get:cVN { l_@@_cukeys_ #1 _prop } \l_@@_given_unit_tl \l_@@_tmpa_fp
  }
%    \end{macrocode}
%
%
% \section {Creating New Units}
%
%
%\begin{macro}{ \declarecookingunit , \newcookingunit , \providecookingunit }
%    \begin{macrocode}
\NewDocumentCommand \declarecookingunit { o m }
  {
    \seq_if_in:NnTF \g_@@_list_of_defined_units_seq {#2} 
      { \msg_info:nnn { cooking-units } { redefine-unit } {#2} }
      { \@@_new_cooking_unit:nn {#1} {#2} }
    \@@_set_cooking_unit:nn {#1} {#2}
  }
\NewDocumentCommand \newcookingunit { o m }
  {
    \@@_new_cooking_unit:nn {#1} {#2}
    \@@_set_cooking_unit:nn {#1} {#2}
  }
\NewDocumentCommand \providecookingunit { o m }
  {
    \seq_if_in:NnF \g_@@_list_of_defined_units_seq {#2} 
      {
        \@@_new_cooking_unit:nn {#1} {#2} 
        \@@_set_cooking_unit:nn {#1} {#2}
      }
  }
%    \end{macrocode}
%\end{macro}
%
%    \begin{macrocode}
\cs_new:Npn \@@_new_cooking_unit:nn #1#2
  {
    \seq_if_in:NnTF \g_@@_list_of_defined_units_seq {#2} 
      { \msg_error:nnn { cooking-units } { unit-already-defined } {#2} }
      {
        \seq_put_right:Nn \g_@@_list_of_defined_units_seq {#2}
        \tl_new:c { l_@@_tmpa_  #2 _tl }
        \tl_set_eq:cN { l_@@_tmpa_  #2 _tl } \c_one_fp
        \clist_new:c { l_@@_predefined_option_#2_clist } 
        \keys_define:nn { cooking-units }
          {
            set-option-for-#2 .clist_set:c = { l_@@_predefined_option_#2_clist } ,
            add-option-for-#2 .code:n = 
              { \clist_put_right:cn { l_@@_predefined_option_#2_clist } {##1} },
          }
        \prop_new:c { l_@@_cukeys_ #2 _prop }
        \tl_new:c { l_@@_default_unit_ #2 _tl }
      }
  }
\cs_new:Npn \@@_set_cooking_unit:nn #1#2
  {
    \IfNoValueTF {#1}
      {
        \tl_set:cn { l_@@_default_unit_ #2 _tl } {#2}
        \@@_deftranslation_base:xxn {#2} \c_@@_postfix_unit_tl {#2}
      }{
        \tl_set:cn { l_@@_default_unit_ #2 _tl } {#1}
        \@@_deftranslation_base:xxn {#2} \c_@@_postfix_unit_tl {#1}
      }
    \@@_deftranslation_base:xxn {#2} \c_@@_postfix_unitname_tl { \q_@@_no_translation }
    \@@_deftranslation_base:xxn {#2} \c_@@_postfix_unitname_pl_tl { \q_@@_no_translation }
    \@@_deftranslation_base:xxn {#2} \c_@@_postfix_gender_tl { m }
  }
%    \end{macrocode}
%
%
%
% \section {Names}
%
% Stolen from \pkg{l3keys.dtx}
%    \begin{macrocode}
\tl_new:N \l_@@_sanitise_tl
\cs_new_protected:Npn \@@_sanitise_aux:w #1 \q_mark
  { \tl_set:Nn \l_@@_sanitise_tl {#1} }
\group_begin:
  \char_set_catcode_active:n { `\< }
  \char_set_catcode_active:n { `\> }
  \cs_new:Npn \@@_sanitize_open_arrow:
    {
      \exp_after:wN \@@_sanitize_open_arrow_auxi:w \l_@@_sanitise_tl
        \q_mark < \q_nil <
      \exp_after:wN \@@_sanitise_aux:w \l_@@_sanitise_tl
    }
  \cs_new_protected:Npn \@@_sanitize_open_arrow_auxi:w #1 <
    {
      \tl_set:Nn \l_@@_sanitise_tl {#1}
      \@@_sanitize_open_arrow_auxii:w
    }
  \cs_new_protected:Npn \@@_sanitize_open_arrow_auxii:w #1 <
    {
      \quark_if_nil:nF {#1}
        {
          \tl_set:Nx \l_@@_sanitise_tl
            {
              \exp_not:V \l_@@_sanitise_tl
              \token_to_str:N <
              \exp_not:n {#1}
            }
          \exp_after:wN \@@_sanitize_open_arrow_auxii:w
        }
    }
  \cs_new:Npn \@@_sanitize_close_arrow:
    {
      \exp_after:wN \@@_sanitize_close_arrow_auxi:w \l_@@_sanitise_tl
        \q_mark > \q_nil >
      \exp_after:wN \@@_sanitise_aux:w \l_@@_sanitise_tl
    }
  \cs_new_protected:Npn \@@_sanitize_close_arrow_auxi:w #1 >
    {
      \tl_set:Nn \l_@@_sanitise_tl {#1}
      \@@_sanitize_close_arrow_auxii:w
    }
  \cs_new_protected:Npn \@@_sanitize_close_arrow_auxii:w #1 >
    {
      \quark_if_nil:nF {#1}
        {
          \tl_set:Nx \l_@@_sanitise_tl
            {
              \exp_not:V \l_@@_sanitise_tl
              \token_to_str:N >
              \exp_not:n {#1}
            }
          \exp_after:wN \@@_sanitize_close_arrow_auxii:w
        }
    }
\group_end:
\cs_new_protected:Npn \@@_sanitize_arrows:n #1
  {
    \tl_set:Nn \l_@@_sanitise_tl {#1}
    \@@_sanitize_open_arrow:
    \@@_sanitize_close_arrow:
  }
%    \end{macrocode}
%
%
%    \begin{macrocode}
\NewDocumentCommand \cudefinename { m m }
  {
    \tl_set:Nn \l_@@_language_tl {#1}
    \@@_sanitize_arrows:n {#2}
    \exp_last_unbraced:NV
    \@@_cuname_parse_input:n \l_@@_sanitise_tl
    \q_recursion_tail \q_recursion_tail \q_recursion_stop
  }
%    \end{macrocode}
%
%
%    \begin{macrocode}
\cs_new:Npn \@@_cuname_parse_input:n #1
  {
    \peek_meaning_ignore_spaces:NTF [
      { 
        \@@_cuname_parse_unit_symbol:nw {#1}
      }{ 
        \clist_if_in:NnTF \g_@@_allowed_special_keys_clist {#1} 
          { \@@_cuname_parse_input_aux:nn {#1} }
          { \@@_cuname_parse_unit_symbol:nw {#1} [ \q_no_value ] }
      }
  }
%    \end{macrocode}
%
%    \begin{macrocode}
\cs_new:Npn \@@_cuname_parse_unit_symbol:nw #1 [#2]
  {
    \quark_if_recursion_tail_stop:n {#1} 
    \quark_if_recursion_tail_stop_do:nn {#2} 
      { \msg_error:nn { cooking-units } { missing-argument } }
    \@@_error_if_unit_not_defined:n {#1}
    \quark_if_no_value:nTF {#2}
      {
        \@@_deftranslation_to:Vxxv 
          \l_@@_language_tl {#1} 
          \c_@@_postfix_unit_tl 
          { l_@@_default_unit_ #1 _tl }
      }{ 
        \@@_deftranslation_to:Vxxn
          \l_@@_language_tl {#1}
          \c_@@_postfix_unit_tl {#2} 
      }
    \@@_cuname_parse_input_aux:nn {#1}
  }
%    \end{macrocode}
%
%
%    \begin{macrocode}
\cs_new:Npn \@@_cuname_parse_input_aux:nn #1#2
  {
    \quark_if_recursion_tail_stop:n {#1} 
    \quark_if_recursion_tail_stop_do:nn {#2} 
      { \msg_error:nn { cooking-units } { missing-argument } }
    \clist_if_in:NnF \g_@@_allowed_special_keys_clist {#1} 
      { \@@_error_if_unit_not_defined:n {#1} }
    \@@_deftranslation_to:Vxxn 
      \l_@@_language_tl {#1} 
      \c_@@_postfix_unitname_tl {#2}
    \peek_meaning_ignore_spaces:NTF [
      { \@@_cuname_parse_bracket:nw {#1} }
      { \@@_cuname_parse_bracket:nw {#1} [#2] }
  }
%    \end{macrocode}
%
%    \begin{macrocode}
\cs_new:Npn \@@_cuname_parse_bracket:nw #1 [#2]
  {
    \clist_if_in:NnF \g_@@_allowed_special_keys_clist {#1} 
      { 
        \@@_deftranslation_to:Vxxn \l_@@_language_tl {#1} 
          \c_@@_postfix_unitname_pl_tl {#2}
      }
    \peek_meaning_ignore_spaces:NTF <
      { \@@_cuname_parse_gender:nw {#1} }
      { \@@_cuname_parse_gender:nw {#1} <m> }
  }
%    \end{macrocode}
%
%    \begin{macrocode}
\cs_new:Npn \@@_cuname_parse_gender:nw #1 <#2>
  {
    \@@_check_if_correct_gender_input:n {#2}
    \@@_deftranslation_to:Vxxn 
      \l_@@_language_tl {#1} 
      \c_@@_postfix_gender_tl {#2}
    \@@_cuname_parse_input:n
  }
%    \end{macrocode}
%
%
%
% \subsection {cudefinesymbol}
%
%
%    \begin{macrocode}
\NewDocumentCommand \cudefinesymbol { m m }
  {
    \tl_set:Nn \l_@@_language_tl {#1}
    \@@_cuprint_define_printed_unit:nn #2
    \q_recursion_tail \q_recursion_tail \q_recursion_stop
  }
%    \end{macrocode}
%
%    \begin{macrocode}
\cs_new:Npn \@@_cuprint_define_printed_unit:nn #1#2
  {
    \quark_if_recursion_tail_stop:n {#1} 
    \quark_if_recursion_tail_stop_do:nn {#2} 
      { \msg_error:nn { cooking-units } { missing-argument } }
    \clist_if_in:NnTF \g_@@_allowed_special_keys_clist {#1} 
      {
        \@@_deftranslation_to:Vxxn \l_@@_language_tl {#1} 
          \c_@@_postfix_unitname_tl {#2}
      }{ 
        \@@_error_if_unit_not_defined:n {#1} 
        \@@_deftranslation_to:Vxxn 
          \l_@@_language_tl {#1} 
          \c_@@_postfix_unit_tl {#2}
      }
    \@@_cuprint_define_printed_unit:nn
  }  
%    \end{macrocode}
%
%
%
% \subsection {Phrases}
%
%
%    \begin{macrocode}
\@@_newtranslation_base:nVn { phrase-list } \c_@@_postfix_phrase_tl { \q_@@_no_translation }
%    \end{macrocode}
%
%    \begin{macrocode}
\prg_new_conditional:Npnn \@@_phrase_list_get_for:NN #1#2 { TF , T , F }
  {
    \@@_translate_let:VNxx #2 #1 { phrase-list } \c_@@_postfix_phrase_tl
    \tl_if_eq:NNTF #1 \q_@@_no_translation
      { \prg_return_false: }
      { \prg_return_true: }
  } 
\prg_new_conditional:Npnn \@@_get_phrase_for_current_langauge:NN #1#2 { T }
  {
    \@@_translate_let:Nxx #1 { phrase-list } \c_@@_postfix_phrase_tl
    \tl_if_eq:NNTF #1 \q_@@_no_translation
      { \prg_return_false: }
      { 
        \@@_translate_let:Nxx #2 { phrase-list-list } \c_@@_postfix_phrase_tl
        \prg_return_true: 
      }
  } 
%    \end{macrocode}
%    \begin{macrocode}
\NewDocumentCommand \cudefinephrase { m m }
  {
    \@@_cuphrase:nn {#1} {#2}
  }
\cs_new:Npn \@@_cuphrase:nn #1#2
  {
    \tl_set:Nn \l_@@_language_tl {#1}
%    \end{macrocode}
%    \begin{macrocode}
      \@@_phrase_list_get_for:NNTF \l_@@_phrase_prop \l_@@_language_tl
        {
          \@@_translate_let:VNxx \l_@@_language_tl \l_@@_phrase_numbers_clist 
            { phrase-list-list } \c_@@_postfix_phrase_tl
        }{
          \prop_clear:N \l_@@_phrase_prop
          \clist_clear:N \l_@@_phrase_numbers_clist
        }
%    \end{macrocode}
%    \begin{macrocode}
      \@@_sanitize_arrows:n {#2}
      \exp_last_unbraced:NV
      \@@_cuphrase_parse:n \l_@@_sanitise_tl
        \q_recursion_tail \q_recursion_tail \q_recursion_stop
%    \end{macrocode}
%    \begin{macrocode}
    \clist_sort:Nn \l_@@_phrase_numbers_clist
      {
        \int_compare:nNnTF { \int_abs:n {##1} } < { \int_abs:n {##2} }
          { \sort_return_swapped: }
          { 
            \int_compare:nNnTF { \int_abs:n {##1} } = { \int_abs:n {##2} }
              {
                \int_compare:nNnTF {##1} < {##2}
                  { \sort_return_same: }
                  { \sort_return_swapped: }
              }{ \sort_return_same: }
          }
      }
%    \end{macrocode}
%    \begin{macrocode}
    \@@_deftranslation_to:VxxV
      \l_@@_language_tl { phrase-list } 
      \c_@@_postfix_phrase_tl 
      \l_@@_phrase_prop
    \@@_deftranslation_to:VxxV 
      \l_@@_language_tl { phrase-list-list } 
      \c_@@_postfix_phrase_tl \l_@@_phrase_numbers_clist
  }  
%    \end{macrocode}
%
%
%    \begin{macrocode}
\cs_new:Npn \@@_cuphrase_parse:n #1
  {
    \quark_if_recursion_tail_stop:n {#1}
    \@@_if_integer:nF {#1} 
      { \msg_error:nnn { cooking-units} { phrase-unit-not-an-integer } {#1} }
    \peek_meaning_remove_ignore_spaces:NTF *
      {
        \int_set:Nn \l_@@_tmpa_int {-#1}
        \@@_cuphrase_parse_normal:Vn \l_@@_tmpa_int
      }{
        \int_set:Nn \l_@@_tmpa_int {#1}
        \@@_cuphrase_parse_normal:Vn \l_@@_tmpa_int
      }
  }
%    \end{macrocode}
%
%    \begin{macrocode}
\cs_new:Npn \@@_cuphrase_parse_normal:nn #1#2
  {
    \quark_if_recursion_tail_stop_do:nn {#2} 
      { \msg_error:nn { cooking-units } { missing-argument } }
    \prop_put:Nnn \l_@@_phrase_prop {#1} {#2}
    \clist_if_in:NnF \l_@@_phrase_numbers_clist {#1}
      { \clist_put_right:Nn \l_@@_phrase_numbers_clist {#1} }
    \peek_meaning_ignore_spaces:NTF [
      {
        \@@_chuphrase_parse_plural:nw {#1}
      }{
        \@@_chuphrase_parse_plural:nw {#1} [#2]
      }
  }
\cs_generate_variant:Nn \@@_cuphrase_parse_normal:nn { V }
%    \end{macrocode}
%
%    \begin{macrocode}
\cs_new:Npn \@@_chuphrase_parse_plural:nw #1 [#2]
  {
    \prop_put:Nnn \l_@@_phrase_prop { #1-pl } {#2}
    \peek_meaning_ignore_spaces:NTF <
      {
        \@@_chuphrase_parse_gender:nw {#1}
      }{
        \@@_chuphrase_parse_gender:nw {#1} < m >
      }
  }
%    \end{macrocode}
%
%    \begin{macrocode}
\cs_new:Npn \@@_chuphrase_parse_gender:nw #1 <#2>
  {
    \@@_check_if_correct_gender_input:n {#2}
    \@@_deftranslation_to:Vxxn 
      \l_@@_language_tl { #1-phrase-gender } 
      \c_@@_postfix_gender_tl {#2}
    \@@_cuphrase_parse:n
  }
%    \end{macrocode}
%
%
%
%
% \subsection{Old stuff}
%
% \begin{macro} { \@@_old_cutext_default:nnn }
% \changes {2016/06/11} {0.98a} { New command }
%    \begin{macrocode}
\cs_new:Npn \@@_old_cutext_default:nnn #1#2#3
  {
    \bool_if:NTF \l_@@_draft_bool
      {#2}
      {
        \tl_set:Nn \l_@@_cutext_last_value_tl {#2}
        \tl_if_in:NVTF \l_@@_cutext_last_value_tl \l_@@_input_range_sign_tl
          { 
            \tl_replace_once:NVn \l_@@_cutext_last_value_tl 
              \l_@@_input_range_sign_tl { \q_@@_range } 
            \@@_old_cutext_parse_range:Vn \l_@@_cutext_last_value_tl {#3} 
          }{
            \@@_old_cutext_print_input:Nn \l_@@_cutext_last_value_tl {#3}
          }
      }
  }
%    \end{macrocode}
%    \begin{macrocode}
\cs_new_protected:Npn \@@_old_cutext_print_input:Nn #1#2
  {
    \@@_old_cutext_if_numeral_is_int_and_parse_and_smaller_then_print_numerals:NTF #1
      {
        \@@_int_if_equal_one:nTF {#1}
          {
            \@@_translate_one_to_and_check_existance:Nx \l_@@_translation_tmpa_tl {#2}
            \bool_if:NTF \l_@@_cutext_uppercase_word_bool
              {
                \exp_args:Nx \tl_upper_case:n { \tl_head:V \l_@@_translation_tmpa_tl }
                \tl_tail:V \l_@@_translation_tmpa_tl
              }
              { \l_@@_translation_tmpa_tl }
          }{ 
            \bool_if:NTF \l_@@_cutext_uppercase_word_bool
              { \@@_print_Numeral:n {#1} }
              { \@@_print_numeral:n {#1} }
          }
      }
      { \@@_print_numerical_input:N #1 }
  }
%    \end{macrocode}
% \end{macro}
%
% \begin{macro} {\@@_old_cutext_if_numeral_is_int_and_parse_and_smaller_then_print_numerals:N}
%    \begin{macrocode}
\prg_new_conditional:Npnn \@@_old_cutext_if_numeral_is_int_and_parse_and_smaller_then_print_numerals:N #1 { TF }
  {
    \bool_if:NTF \l_@@_lokal_numeral_bool
      {
        \@@_if_integer:VTF #1
          {
            \int_compare:nNnTF {#1} < { \l_@@_print_numerals_below_int }
              { \prg_return_true: }
              { \prg_return_false: }
          }{ \prg_return_false: }
      }{ \prg_return_false: }
  }
%    \end{macrocode}
% \end{macro}
%
%    \begin{macrocode}
\cs_new:Npn \@@_old_cutext_parse_range:nn #1 #2
  {
    \@@_old_cutext_parse_range_aux:nww {#2} #1 \q_stop
  }
\cs_generate_variant:Nn \@@_old_cutext_parse_range:nn { V }
\cs_new:Npn \@@_old_cutext_parse_range_aux:nww #1 #2 \q_@@_range #3 \q_stop
  {
    \tl_set:Nn \l_@@_tmpa_tl {#2}
    \tl_set:Nn \l_@@_cutext_last_value_tl {#3}
    \@@_old_cutext_print_input:Nn \l_@@_tmpa_tl {#1}
    \tl_use:N \l_@@_cutext_range_sign_tl
    \@@_old_cutext_print_input:Nn \l_@@_cutext_last_value_tl {#1}
  }
%    \end{macrocode} 
%
%
% \begin{macro}{\@@_cuam_old:n}
%    \begin{macrocode}
\cs_new:Npn \@@_cuam_old:n #1
  {
    \tl_set_rescan:Nnn \l_@@_tmpa_tl
      {
        \char_set_catcode_letter:N \_ %
        \char_set_catcode_ignore:N\ %
      } {#1}
    \bool_if:NTF \l_@@_draft_bool
      { \l_@@_tmpa_tl }
      { \@@_cuam_old_parse:V \l_@@_tmpa_tl }
  }
%    \end{macrocode}
% \end{macro}
%
%    \begin{macrocode}
\cs_new_protected:Npn \@@_cuam_old_parse:n #1
  {
    \tl_if_in:nVTF {#1} \l_@@_input_range_sign_tl
      {
        \tl_set:Nn \l_@@_tmpa_tl {#1}
        \tl_replace_once:NVn \l_@@_tmpa_tl \l_@@_input_range_sign_tl { \q_@@_range }
        \@@_cuam_old_parse_range:V \l_@@_tmpa_tl
      }{
        \tl_if_in:nnTF {#1} { / }
          {
            \tl_if_in:nnTF {#1} { _ }
              { \@@_cuam_old_parse_mixed_frac:www #1 \q_stop }
              { \@@_cuam_old_parse_frac:ww #1 \q_stop }
          }{ 
            \tl_if_in:nnTF {#1} { _ }
              { \msg_error:nnn { cooking-units } { missing-slash } {#1} }
              { \@@_cuam_old_parse_scale:n {#1} }
          }
      }
  }
\cs_generate_variant:Nn \@@_cuam_old_parse:n { V }
%    \end{macrocode}
%
%    \begin{macrocode}
\cs_new:Npn \@@_cuam_old_parse_range:n #1
  {
    \@@_cuam_old_parse_range_aux:ww #1 \q_nil
  }
\cs_generate_variant:Nn \@@_cuam_old_parse_range:n { V }
\cs_new:Npn \@@_cuam_old_parse_range_aux:ww #1 \q_@@_range #2 \q_nil
  {
    #1 \l_@@_cunum_range_sign_tl #2
  }
%    \end{macrocode}
%
%
%    \begin{macrocode}
\cs_new:Npn \@@_cuam_old_parse_scale:n #1 {#1}
%    \end{macrocode}
%
%    \begin{macrocode}
\cs_new:Npn \@@_cuam_old_parse_frac:ww #1/#2 \q_stop
  { \@@_frac:nn {#1} {#2} }
%    \end{macrocode}
%
%    \begin{macrocode}
\cs_new:Npn \@@_cuam_old_parse_mixed_frac:www #1_#2/#3 \q_stop
  { 
    #1 
    \hbox_to_wd:nn { \l_@@_mixed_frac_dim } { } 
    \@@_frac:nn {#2} {#3} 
  }
%    \end{macrocode}
%
%
% \subsubsection {cufrac}
%
% Obsolete.
%    \begin{macrocode}
\NewDocumentCommand \cufrac { O{} m }
  {
    \msg_error:nnnn { cooking-units } { obsolete-command } { \cufrac } { \cuam }
    \group_begin:
    \tl_if_empty:nF {#1}
      { \keys_set:nn { cooking-units } {#1} }
    \@@_cufrac:n {#2}
    \group_end:
  }
\cs_new:Npn \@@_cufrac:n #1
  {
    \tl_set_rescan:Nnn \l_@@_tmpa_tl
      {
        \char_set_catcode_letter:N \_ %
        \char_set_catcode_ignore:N\ %
      } {#1}
    \@@_cufrac_parse:V \l_@@_tmpa_tl
  }
\cs_new:Npn \@@_cufrac_parse:n #1
  {
    \tl_if_in:nnTF {#1} { / }
      {
        \tl_if_in:nnTF {#1} { _ }
          { \@@_cufrac_parse_mixed_frac:www #1 \q_stop }
          { \@@_cufrac_parse_frac:ww #1 \q_stop }
      }{ 
        \tl_if_in:nnTF {#1} { _ }
          { \msg_error:nnn { cooking-units } { missing-slash } {#1} }
          { \@@_cufrac_parse_scale:n {#1} }
      }
  }
\cs_generate_variant:Nn \@@_cufrac_parse:n { V }
\cs_new:Npn \@@_cufrac_parse_scale:n #1 {#1}
\cs_new:Npn \@@_cufrac_parse_frac:ww #1/#2 \q_stop
  { \@@_frac:nn {#1} {#2} }
\cs_new:Npn \@@_cufrac_parse_mixed_frac:www #1_#2/#3 \q_stop
  { 
    #1 
    \hbox_to_wd:nn { \l_@@_mixed_frac_dim } { } 
    \@@_frac:nn {#2} {#3} 
  }
%    \end{macrocode}
%
%
%
%
% \subsection {cusetup}
%
%    \begin{macrocode}
\NewDocumentCommand \cusetup { m }
  {
    \keys_set:nn { cooking-units } {#1}
  }
%    \end{macrocode}
%
%
%
% \subsection{Definitions et all}
%
%    \begin{macrocode}
\newcookingunit { kg }
\newcookingunit { dag }
\newcookingunit { g }
\newcookingunit { oz }
\newcookingunit { lb }
\newcookingunit { stick }
\newcookingunit [ \ensuremath{ \@@_frac:nn { eV } { c^2 } } ] { eVc-2 }
%    \end{macrocode}
%
%    \begin{macrocode}
\newcookingunit { K }
\newcookingunit [ \ensuremath{ {} ^ { \circ } } \kern-\scriptspace C ]  { C }
\newcookingunit [ \ensuremath{ {} ^ { \circ } } \kern-\scriptspace F ] { F }
\newcookingunit [ \ensuremath{ {} ^ { \circ } } \kern-\scriptspace R\'{e} ] { Re }
%    \end{macrocode}
%
%    \begin{macrocode}
\newcookingunit { d }
\newcookingunit { h }
\newcookingunit { min }
\newcookingunit { s }
\newcookingunit [ \ensuremath{ \@@_frac:nn { \hbar } { eV } } ] { hbareV-1 }
%    \end{macrocode}
%
%    \begin{macrocode}
\newcookingunit { m }
\newcookingunit { cm }
\newcookingunit { dm }
\newcookingunit { mm }
\newcookingunit { in }
\newcookingunit [ \ensuremath{ \@@_frac:nn { c\hbar } { eV } } ] { chbareV-1 }
%    \end{macrocode}
%
%    \begin{macrocode}
\newcookingunit { l }
\newcookingunit { dl }
\newcookingunit { cl }
\newcookingunit { ml }
\newcookingunit [ \ensuremath { \@@_frac:nn { c^3 \hbar^3 } { eV^3 } } ] { (chbareV-1)3 }
%    \end{macrocode}
%
%    \begin{macrocode}
\newcookingunit { cal }
\newcookingunit { kcal }
\newcookingunit { J }
\newcookingunit { kJ }
\newcookingunit { eV }
%    \end{macrocode}
%
%    \begin{macrocode}
\newcookingunit [ pinch ] { pn }
\newcookingunit { EL }
\newcookingunit { TL }
\newcookingunit [ ssp. ] { ssp } %% saltspoonful
\newcookingunit [ csp. ] { csp } %% coffeespoonful
\newcookingunit [ dsp. ] { dsp }
\newcookingunit [ Msp. ] { Msp }
%    \end{macrocode}
%
%    \begin{macrocode}
\DeclareLanguageAlias { AmericanEnglish } { American }
%    \end{macrocode}
%
%    \begin{macrocode}
\cudefinename { German }
  {
    { kg } { Kilogramm } < n >
    { dag } { Dekagramm } < n >
    { g } { Gramm } < n >
    { oz } { Unze } < f >
    { lb } { Pfund } < n >
%    \end{macrocode}
%    \begin{macrocode}
    { d } { Tag } [ Tage ]
    { h } { Stunde } [ Stunden ] < f >
    { min } { Minute } [ Minuten ] < f >
    { s } { Sekunde } [ Sekunden ] < f >
%    \end{macrocode}
%    \begin{macrocode}
    { C } { Grad \space Celsius } 
    { K } { Kelvin } < n >
    { F } { Grad \space Fahrenheit }
    { Re } { Grad \space R\'{e}amur }
%    \end{macrocode}
%    \begin{macrocode}
    { m } { Meter } < n >
    { dm } { Dezimeter } < n >
    { cm } { Centimeter } < n >
    { mm } { Millimeter } < n >
    { in } { Zoll } 
%    \end{macrocode}
%    \begin{macrocode}
    { l } [ l ] { Liter }
    { dl } { Deziliter }
    { cl } { Centiliter }
    { ml } { Milliliter }
%    \end{macrocode}
%    \begin{macrocode}
    { cal } { Kalorie } [ Kalorien ] < f >
    { kcal } { Kilokalorie } [ Kilokalorien ] < f >
    { J } { Joule }
    { kJ } { Kilojoule }
    { eV } { Elektronenvolt } < n > 
%    \end{macrocode}
%    \begin{macrocode}
    { Msp } [ Msp. ] { Messerspitze } [ Messerspitzen ] < f >
    { pn } [ Prise ] { Prise } [ Prisen ] < f >
    { EL } [ EL ] { Essl{\"o}ffel }
    { TL } [ TL ] { Teel{\"o}ffel }
    { csp } [ KL ] { Mokkal{\"o}ffel }
%    \end{macrocode}
%    \begin{macrocode}
    { decimal-mark } { , }
    { one (m) } { ein }
    { one (f) } { eine }
    { one (n) } { ein }
  }
%    \end{macrocode}
%
%
%    \begin{macrocode}
\cudefinename { English }
  {
    { kg } { kilogramme }
    { dag } { decagramme }
    { g } { gramme }
    { oz } { ounce } 
    { lb } { pound } [ pounds ]
    { stick } { stick } [ sticks ]
%    \end{macrocode}
%    \begin{macrocode}
    { d } { day } [ days ]
    { h } { hour } [ hours ]
    { min } { minute } [ minutes ]
    { s } { second } [ seconds ]
%    \end{macrocode}
%    \begin{macrocode}
    { C } { degree \space Celsius }  [ degrees \space Celsius ]
    { F } { degree \space Fahrenheit } [ degrees \space Fahrenheit ]
    { K } { kelvin }
    { Re } { degree \space R\'{e}aumur } [ degrees \space R\'{e}aumur ]
%    \end{macrocode}
%    \begin{macrocode}
    { m } { metre } [ metres ]
    { dm } { decimetre } [ decimetres ]
    { cm } { centimetre } [ centimetres ]
    { mm } { millimitre } [ millimitres ]
    { in } { inch } [ inches ]
%    \end{macrocode}
%    \begin{macrocode}
    { l } [ \ensuremath { \ell } ] { litre } [ litres ]
    { dl } { decilitre } [ decilitres ]
    { cl } { centilitre } [ centilitres ]
    { ml } { millilitre } [ millilitres ]
%    \end{macrocode}
%    \begin{macrocode}
    { cal } { calorie } [ calories ]
    { kcal } { kilocalorie } [ kilocalories ]
    { J } { joule } [ joules ]
    { kJ } { kilojoule } [ kilojoules ]
    { eV } { electron \space volt } 
%    \end{macrocode}
%    \begin{macrocode}
%    { Msp } [ pinch ] { pinch } [ pinches ]
%    { Msp } { Messerspitze } [ Messerspitzen ] <f>
    { pn } [ pinch ] { pinch } [ pinches ]
    { EL } [ tbsp. ] { tablespoon } [ tablespoons ]
    { TL } [ tsp. ] { teaspoon } [ teaspoons ]
    { dsp } { dessertspoonful } 
    { csp } { coffeespoonful } 
    { ssp } { saltspoonful } 
%    \end{macrocode}
%    \begin{macrocode}
    { decimal-mark } { . }
    { one (m) } { one }
    { one (f) } { one }
    { one (n) } { one }
  }
%    \end{macrocode}
%
%    \begin{macrocode}
\cudefinename { AmericanEnglish }
  {
    { kg } { kilogram }
    { dag } { decagram }
    { g } { gram }
    { oz } { ounce } 
%    \end{macrocode}
%    \begin{macrocode}
    { m } { meter } [ meters ]
    { dm } { decimeter } [ decimeters ]
    { cm } { centimeter } [ centimeters ]
    { mm } { millimiter } [ millimiters ]
    { in } { inch } [ inches ]
%    \end{macrocode}
%    \begin{macrocode}
    { l } [ \ensuremath { \ell } ] { liter } [ liters ]
    { dl } { deciliter } [ deciliters ]
    { cl } { centiliter } [ centiliters ]
    { ml } { milliliter } [ milliliters ]
%    \end{macrocode}
%    \begin{macrocode}
%    { Msp } { Messerspitze } [ Messerspitzen ] <f>
    { pn } [ pn. ] { pinch } [ pinches ]
  }
%    \end{macrocode}
%
%
%    \begin{macrocode}
\cudefinename { French }
  {
    { kg } { kilogramme } [ kilogrammes ]
    { dag } { d\'{e}cagramme } [ d\'{e}cagrammes]
    { g } { gramme } [ gramme ]
    { oz } { once } < f >
    { lb } { livre } [ livres ] < f >
%    \end{macrocode}
%    \begin{macrocode}
    { d } { jour } [ jours ]
    { h } { heure } [ heures ] < f >
    { min } { minute } [ minutes ] < f >
    { s } { seconde } [ secondes ] < f >
%    \end{macrocode}
%    \begin{macrocode}
    { C } { degr\'{e} \space Celsius }  [ degr\'{e}s \space Celsius ]
    { K } { degr\'{e} \space Fahrenheit } [ degr\'{e}s \space Fahrenheit ]
    { F } { kelvin } [ kelvins ]
    { Re } { \'{e}chelle \space R\'{e}aumur } [ degr\'{e}s \space R\'{e}aumur ]
%    \end{macrocode}
%    \begin{macrocode}
    { m } { m\`{e}tre } [ m\`{e}tres ]
    { dm } { d\'{e}cim\`{e}tre } [ d\'{e}cim\`{e}tres ]
    { cm } { centim\`{e}tre } [ centim\`{e}tres ]
    { mm } { millim\`{e}tre } [ millim\`{e}tres ]
    { in } [ po ] { pouce } [ pouces ]
%    \end{macrocode}
%    \begin{macrocode}
    { l } [ L ] { litre } [ litres ]
    { dl } [ dL ] { d\'{e}cilitre } [ d\'{e}cilitres ]
    { cl } [ cL ] { centilitre } [ centilitres ]
    { ml } [ mL ] { millilitre } [ millilitres ]
%    \end{macrocode}
%    \begin{macrocode}
    { cal } { calorie } [ calorie ]
    { kcal } { kilocalorie } [ kilocalories ]
    { J } { joule } [ joules ]
    { kJ } { kilojoule } [ kilojoules ]
    { eV } { \'{e}lectron-volt } [ \'{e}lectron-volts ]
%    \end{macrocode}
%    \begin{macrocode}
    { pn } { pinc\'{e}e } < f > 
    { EL } { cuill\'{e}re  \space \`{a} \space  soupe } < f >
    { TL } { cuill\'{e}re \space \`{a} \space caf\'{e} } < f >
%    \end{macrocode}
%    \begin{macrocode}
    { decimal-mark } { . }
    { one (m) } { un }
    { one (f) } { une }
    { one (n) } { un }
  } 
%    \end{macrocode}
%
%
%
%    \begin{macrocode}
\cudefinekeys { kg }
  {
    { dag }{ 100 }
    { g }  { 1000 }
    { oz } { 35.27399 } 
    { lb } { 2.2046226 } %% 2.204 622 6
    { stick } { 8.81849 }
    { eVc-2 } { 560958865.0e+27 } %% 560958865.0 +- 3.5 e+27
  }
%    \end{macrocode}
%    \begin{macrocode}
\cudefinekeys { d }
  {
    { h }  { 24 }
    { min }{ 1440 }
    { s }  { 86400 }
    { hbareV-1 } { 151926746.1e+7 * 86400 } %% 151926746.1 +- 2.1 e+7
  }
%\cuaddtokeys { s } { hbareV-1 } { 6.582119514e-16 }
%    \end{macrocode}
%    \begin{macrocode}
\cudefinekeys { m }
  {
    { dm } { 10 }
    { cm } { 100 }
    { mm } { 1000 }
    { in } { 39.370079 }
    { chbareV-1 } { 5067730.759 } %% 5067730.759 +- 0.070
  }
%\cuaddtokeys { m } { chbareV-1 } { 1.97326972e-7 }
%    \end{macrocode}
%    \begin{macrocode}
\cudefinekeys { l }
  {
    { dl } { 10 }
    { cl } { 100 } 
    { ml }{ 1000 } 
    { (chbareV-1)3 } { 130148929.5e+12 * 1e-3 } %% (130148929.5 +- 5.4 e+12)*1e-3
  }
%    \end{macrocode}
%    \begin{macrocode}
\cudefinekeys { J }
  {
    { kJ }{ 1e-3 }
    { cal }{ 0.2388459 }
    { kcal }{ 0.2388459e-3 }
    { eV }{ 624150912.6e+10 } %% 624150912.6 +- 3.9 e+10
  }
%\cuaddtokeys { J } { eV } { 1.6021766208e-19 }
%    \end{macrocode}
%    \begin{macrocode}
\fp_const:Nn \c_@@_kb_eV_fp {  8.617 330 3 e-5  }
\cudefinesinglekey { C }
  {
    { K } { #1 + 273.15 }
    { F } { #1 * 1.8 + 32 }
    { Re } { #1 * 0.8 }
    { eV } { ( #1 + 273.15 ) * \c_@@_kb_eV_fp }
  }
\cudefinesinglekey { F }
  {
    { C } { ( #1 - 32 ) *  5/9 }
    { K } { ( #1 + 459.67 ) *  5/9 }
    { Re } { ( #1 - 32 ) * 4/9 }
    { eV } {( #1 + 459.67 ) *  5/9 * \c_@@_kb_eV_fp }
  }
\cudefinesinglekey { K } 
  {
    { C } { #1 - 273.15 }
    { F } { #1 * 1.8 - 459.67 }
    { Re } { ( #1 - 273.15 ) * 0.8 }
    { eV } { #1 * \c_@@_kb_eV_fp }
  }
\cudefinesinglekey { Re }
  {
    { K } { #1 * 1.25 + 273.15 }
    { C } { #1  * 1.25 }
    { F } { #1 * 2.25 + 32 }
    { eV } { ( #1 * 1.25 + 273.15 ) * \c_@@_kb_eV_fp }    
  }
%    \end{macrocode}
%
%
%    \begin{macrocode}
\cudefinephrase { German } 
  {  
%    { 6 } * { halbes \  Dutzend } < n >
%    { 6 }  { efkjwefjkl \  Dutzend } < n >
    { 12 } { Dutzend } < n >
%    { 60 } { Schock } < n >
%    { 1728 } { Gro{\ss}gros } < n >
%    { 144 } { Gros } < n >
  }
%    \end{macrocode}
%
%
% \subsection{Finish}
%
%    \begin{macrocode}
\cusetup 
  { 
    set-option-for-F = { round-to-int = true } ,
    set-option-for-C = { round-to-int = true } ,
    set-option-for-K = { round-to-int = true } ,
    set-option-for-Re = { round-to-int = true } ,
    add-temperature-to-check =
      {
        K = 0,
        C = -273.15 ,
        F = -459.67 ,
        Re = -218.52
      } ,
  }
%    \end{macrocode}
%
%
%
%
%
%
%
%
%
%
%    \begin{macrocode}
%</package>      
%    \end{macrocode}
%
% \end{implementation}
%
%
% \endinput
%
%
%
%
% Local Variables:
% mode: doctex
% TeX-master: t
% End: